\documentclass{ata-calico}
\usepackage{indentfirst}
\usepackage{enumerate}
\pagenumbering{arabic}
\begin{document}

\maketitle

\pauta{Informes e Repasses}
\begin{itemize}
\item Hans: Tá rolando a greve da Comcap por causa do pacotaço do Gean. Ele passou nessa última terça-feira por 13 a 9, mas os trabalhadores ainda tão resistindo. Vai estar acontecendo vigília esse fim de semana inteiro e no domingo as 11h vai estar acontecendo ato em apoio também.
\end{itemize}

\pauta{Atividades Complementares}
\begin{itemize}
\item O NDE (Núcleo Docente Estruturante) elaborou uma normativa sobre as atividades complementares (nº001/CCO/2021 de 01 de fevereiro de 2021), que agora estão no currículo. O documento descreve todas as atividades, alocando horas para cada e separando em categorias.
\item Hans: Esse documento é complicado, porque quando eles colocam que algo empresarial conta como atividade complementar, vai contra o tripé pesquisa-ensino-extensão da universidade. Acho que devemos estar contra esse documento, principalmente esses pontos.
\item Paloma: É difícil dizer pra tirar os estágios das atividades complementares, querendo ou não é uma parte do sistema atual e muito estudante já tem que trabalhar para se manter na universidade, parece injusto tirar.
\item Matheus: Na UDESC, existia um sucateamento muito grande dessas horas complementares. Como era obrigatório, as pessoas iam na forma mais fácil de cumprir as horas. É algo que precisa ter cuidado.
\item Hans: A gente estar colocando empresa júnior e atividades empresariais como extensão, vai estar sucateando a extensão na universidade.
\item João TIZ: O documento já separa a vivência empresarial e a extensão. Se aceitarmos esse documento, é importante reforçar bastante essa separação de que extensão não tem a ver com empresa. Em outro ponto, aqui na UFSC a física tem muitas horas. É importante cuidar pra não sobrecarregar essas horas. Mas de forma geral isso pode ajudar a ter mais engajamento no curso de forma geral.
\item Existe uma falta de espaço para encaixar atividades artísticas, corporais e outras atividades do tipo, seria interessante ver de ter abertura para isso nas atividades complementares.
\item Cauê: É preciso pensar qual o propósito dessas atividades complementares, qual o objetivo com essa formulação de atividades. Tem que ver o que cabe e o que não cabe na formação de profissionais e cidadãos. É importante também demarcar certinho tudo para que depois não fique essa disputa com a coordenação de atividades complementares.
\item Tiz: Seria importante ter cursos de idiomas no documento, ter esse incentivo que seria algo bem importante na nossa formação profissional. Estava pensando em com horas máximas colocar o equivalente a dois semestres de curso. Uma coisa interessante na física também é que o excedente de horas de optativas conta para as atividades complementares.
\end{itemize}

\textbf{Encaminhamento:} Colocar no documento também cursos de idiomas.

\textbf{Encaminhamento:} Colocar no documento a questão de optativas que excedem o máximo também contar para atividades complementares.

\pauta{Apadrinhamento de Calouros}
\begin{itemize}
\item Paloma: A questão do apadrinhamento é difícil porque tem toda a preocupação de não colocar pessoas escrotas como padrinhos, para que os calouros possam se sentir a vontade com os veteranos e ser algo mais saudável. Seria interessante tirar em reunião mesmo quem seria padrinho.
\item Matheus: Seria um padrinho para um grupo de calouros ou 1 pra 1?
\item André: Acho ruim tirar todos em reunião, deixar aberto depois pra pessoas que participam do CALICO, das reuniões.
\item Matheus: Seria interessante serem duplas de padrinhos, para que caso um dos padrinhos ter um problema exista outra que os calouros podem contar com. Seria bom ter um acompanhamento também, pra ver se ta tudo rolando certo.
\item Paloma: Uma ideia é já tirar padrinhos aqui pra não deixar os calouros sem nenhum padrinho, ai divulgar que estamos procurando padrinhos e segunda quando começar as atividades já apadrinhar.
\item Paloma: A ideia é mais os calouros terem alguém que confiem pra pedir ajuda e criar um vinculo.
\item André: Em semestres anteriores, percebi que conversava com os calouros em bar e outros eventos. Ai mesmo depois eles procuravam pra tirar dúvida e pedir ajuda. A ideia do apadrinhamento seria meio nessa linha.
\item Matheus: Eu não sei como a gente levantar um critério geral assim para os padrinhos. A maior preocupação que a gente tem é ter a possibilidade de ter um padrinho com comportamento tóxico, que vai de acordo com a toxicidade do meio acadêmico e que tenha comportamento meio violento/preconceituoso, que não dê suporte. É complicado traçar um critério objetivo para definir quem confiaríamos.
\item André: Acho importante deixar bem explícito os pontos de porque não aceitaríamos alguém. É meio estranho deixar isso fechado pra gestão quando estamos sempre tentando chamar mais gente pro CALICO.
\item Matheus: É uma situação bem complicada, correr o risco de colocar alguém com comportamento tóxico que acabe prejudicando os calouros. Devemos ser o mais abertos e transparentes possível, mas também dar uma controlada.
\end{itemize}

\textbf{Encaminhamento:} Tirar alguns padrinhos em reunião, depois divulgar e abrir para mais pessoas se inscreverem e dar uma sondada em quem se inscrever. No dia 01/02 separar certinho padrinhos com calouros.

Ficaram como padrinhos inicialmente: Teo, Naiara, Helena, Arthur, Matheus, André, Hans, João TIZ.

\presentes{Paloma Cione, Helena Aires, Matheus Roque, Naiara Sachetti, Arthur Pickcius, Teo Gallarza, João Paulo T.I.Zanette, André Régis, Cristian Alchini, Hans Buss, Nicole Schmidt, Vitor Luiz da Silva, Cauê Baasch, Mauricio Konrath, José Daniel do Prado}

\end{document}
