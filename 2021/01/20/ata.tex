\documentclass{ata-calico}
\usepackage{indentfirst}
\usepackage{enumerate}
\pagenumbering{arabic}
\begin{document}

\maketitle

\pauta{Planejamento Estratégico}

\begin{itemize}

\item Objetivo: Pensar coletivamente o que queremos pro Calico: eventos, atuação, coisas que ficaram para trás, etc

\item Não tivemos propostas específicas de pauta, então vamos seguir as do grupo do telegram:
\begin{enumerate}
\item reorganização do calico: teoricamente não podemos fazer em PE, teria que ser em reunião aberta; 
    pensar em possíveis mudanças em representações, mudanças que não precisem de reunião aberta, etc.
    obs: cargos antigos foram eleitas fixas, e cargos como presidente, vice, etc, não podem ser reatribuidas, mas
    mudanças podem ser feitas de forma extra-oficial
\item Calourada
\item Eventos ao longo do semestre
\end{enumerate}
\item Caue: representante do conselho do CTC, diz sobre a renovação da representatividade no CETEC (possivelmente nomear mais alguem de CCO, já que os outros cursos não são tão assíduos em reuniões). Sobre reorganização, os cargos são mais simbólicos, e não existe tanta necessidade de alterar, exceto tesoureiro.

\item  Paloma: não precisamos nos preocupar com o colegiado de curso, já que as nomeações são recentes, e as de departamento quase não tem reuniões e tá meio morto.

\item Luis: não sabemos qual a data de duração do mandato do departamento, mas deve ser um ano e podemos discutir proximas ações.
No conselho do CTC, ele e Cauê podem continuar, e concorda com Caue sobre as cadeiras. Talvez tirar duas cadeiras de titular e uma de suplente. A nomeação não é por curso.  Com relação a outros cargos: colegiados de curso e departamento, concorda com a Paloma. O que mais funciona no departamento são as
câmaras, que também estão paradas. Sugeriu destacar mais uma pessoa para se responsabilizar pelo CETEC, pode ser em uma reunião futura.

\item Mikael: não se importa de deixar a representação do CETEC, e ofereceu alguem a pegar a vaga.

\item Luis: Podemos ter essa discussão em outra reunião aberta, mais pra continuar pensando.

\item Caue: Temos que seguir o planejamento de 2020, a apresentação da chapa, e as pendencias que ficaram dos anos passados. Disputar no colegiado a questão de evasão que está em discussão no colegiado.
A reforma do projeto pedagógico e currículo e formas de pressionar.  Inclusão da disciplina de libras, que já está sendo cobrada no colegiado há alguns anos. Além disso, de pendencias em particular quanto ao colegiado de curso, temos a curricularização da extensão, que está parado no colegiado, e as regulamentação de atividades complementares.

\item Luis: provavelmente o mais simples é a inclusão da disciplina de libras, algo simples e que não deve dar muito trabalho pro NDE.
Inclusive, verificar se foram pedidas as duas disciplinas de libras (I e II).
Sobre a reforma do projeto pedagógico, sugeriu realizar uma reunião aberta para esclarecer a quem interessar o que é, e por que é importante,
aproveitar pra ressuscitar as redes sociais, e  tentar chamar mais pessoas para participar. Provavelmente vai se precisar de uma corrida no começo
do semestre para tentar atrair calouros e dar tempo de discutir antes.

\item Paloma: vai ter reunião do colegiado dia 01/02, e não foi feita a avaliação do ensino remoto, que precisamos discutir no Calico e colegiado.

\item Luis: difícil incluir o tópico na próxima reunião, mas temos os acúmulos do ano passado para levar pro colegiado.

\item Luis: Em algum momento vai chegar no conselho do CTC a questão do Colle. Foi tirado um grupo pra levantar alguns pontos pra levar essa discussão adiante.
Se alguém do Calico quiser entrar no grupo pra dinamizar, seria interessante. Nos prepararmos pra marcar presença desse espaço. \newline

\textbf{Encaminhamentos:}
\begin{enumerate}
\item reprovações e evasão

\item reforma do PPC

\item curricularização da extensão

\item atividades complementares

\item cobrar libras I e II de novo

\end{enumerate}
\end{itemize}
\pauta{Calourada:}
\begin{itemize}
\item Luis: Pensar em formas de incluir os calouros na vida universitária, mostrar que o Calico está la pelos estudantes, fazendo algo mais divertido.
\begin{enumerate}
\item Planejamento do manual do calouro
\item Pedir um espaço nas turmas de introdução
\item Procurar mais formas de contato com os calouros e divulgação do manual dos calouros
\end{enumerate}

\item : pensar em formas mais práticas de incluir os calouros. Fazer noites de jogos com mais frequência. 

\item Cauê: adaptar o manual do calouro pra evitar falar muito do campus, etc. Concorda com os pontos anteriores, pensar em formas de integrar as pessoas que não vão 
se conhecer de outras formas. Reunir o curso como um todo. Talvez trazer de volta os JOgos SEdentarios, pensando em humanizar um pouco as coisas e todo mundo se conhecer,
fazer mais eventos pelo Calico como lives, chamar professores pra falar sobre um assunto. Trazer mais coesão pro curso, com pessoas de vários semestres e professores, etc.

\item Roque: concorda com várias coisas do Cauê. Buscar não deixar as pessoas se sentirem sozinhas. Uma ideia que parece boa considerando o ERE definir tutores pra dar um apoio aos calouros,
conversar e acompanhar os calouros, talvez até levar coisas políticas da faculdade. No manual do calouro, colocar uma seção específica para o ERE, com dicas de saúde física e mental.

\item Hans: Gosta da ideia de apadrinhamento dos calouros. Não acha que lives são muito interessantes, por desviar do costume do Cauê. Fazer uma apresentação do Calico e do movimento estudantil.

\item Curupira: gosta da ideia de apadrinhamento, mas nunca viu dar certo. Sobre juntar a galera no discord, etc, acha importante juntar todos os eventos em um lugar só, pra acostumar as 
pessoas a frequentarem um lugar de CCO onde as pessoas podem interagir. Se ofereceu pra organizar um servidor com as coisas de cco pras pessoas poderem entrar e jogar.

\item Roque: gosta da ideia de centralizar as paradas, sobre apadrinhamento, realmente é difícil, e é importante pensarmos em formas de fazer funcionar, e usar o apadrinhamento como forma de
substituto pra interação dos calouros com os veteranos no bar. Sobre Hans não gostar das lives, Roque acha que é uma boa ideia fazer algo legal, pra incentivar alguma cultura entre os  participantes, quão poucos sejam.

\item Luis: sobre apadrinhamento, é importante tomar cuidado com pessoas tóxicas que existem no curso, e como um possível apadrinhamento pode influenciar negativamente a percepção do curso. Lembrar dos problemas que ocorreram anteriormente nos grupos de wpps, etc. Então, cuidar com operacionalização.
Sobre manual do calouro, entrar em contato com alunos de outros cursos que possam ajudar nas recomendações de dicas de exercícios/saúde.
Sobre lives, acha que vale a pena tentar e ver a resposta antes de desistir.

\item Hans: a ideia é mais fazer uma roda de conversa do que uma live, considerando a redução de participação em lives.

\item Paloma: Concorda com Luis, é difícil confiar no apadrinhamento.

\item Luis: concorda com Hans, sugerindo fazer tanto eventos como live. 

\item Paloma: Sobre apadrinhamento. Os veteranos agora são os 19.2, e não sabemos se eles já planejaram algo. Acha que o apadrinhamento vai ser ruim porque não tem como abrir pra qualquer um que quiser, nem negar todas as pessoas

\item Roque: acha válido negar pessoas e deixar só quem estiver organizando. Apadrinhar grupos de alunos, lurkar nos grupos e tentar encontrar interesses comuns pra juntar grupos de interesse comum. Lembrando dos eventos antigos.

\item Luis: tem ressalvas quanto ao apadrinhamento, mas acha que é questão de operacionalização. Por exemplo, considerar presenças em reuniões do calico pra aprovar alguem a apadrinhar. Ou tirar 2 ou 3  pessoas pra apadrinhar grupos maiores. Não gosta muito de dividir os calouros entre interesses, e não gosta de forms. Sugeriu ir por sorteio, mesmo.

\item Paloma: não tem mais convicção do que falou. Não acha presença em reuniões um bom critério pra aceitar padrinhos.

\item Luis: acha importante que exista uma certa "cobrança" fazendo o mínimo vindo numa reunião do Calico. Seria um jeito de contar a presença e participação das pessoas no curso.

\item Paloma: para evitar disputas e trazer mais transparência, não gosta de selecionar por baixo dos panos as pessoas baseadas na participação em reuniões do Calico. Acha que vai contra a postura da chapa, e
não é um bom critério pra selecionar os padrinhos.

\item Roque: concorda que é um critério meio estranho, mas estamos procurando um critério que garanta que o apadrinhamento seja seguro e positivo, de forma que possamos confiar nos padrinhos escolhidos.

\item Luis: levar pra uma reunião aberta pra discutir coisas da calourada, inclusive o critério de apadrinhamento

Curupira: tem 3 coisas pra falar:
1. quando montamos um grupo, a melhor maneira é isolar ele do resto do mundo. É inevitável surgirem certas barreiras, e é importante ser transparente.
2. o Calico pode ter um nível de segmentação e seguir os princípios dele. não acha um problema definir um apadrinhamento do calico em específico, permitindo a existencia de alguma outra proposta de apadrinhamento.
3. esqueceu, vai voltar quando lembrar.

\item Paloma: na reunião sobre apadrinhamento, tirar as pessoas que serão padrinhas como forma de seleção e divulgação. Levar pra reunião tanto a proposta como o critério como forma de juntar os demais veteranos 

\item Curupira: ao invés de usar as presenças antigas como seleção, usar no lugar a presença em reuniões futuras. Fazer reuniões semanais para acompanhar os grupos de calouros.

\item Roque: concorda com o Curupira sobre reuniões de acompanhamento dos padrinhos. Acha importante ter essa questão do apadrinhamento pronta antes de começarem as aulas, e já começar o semestre com o apadrinhamento funcionando.

\item Curupira: seria bom ter antes, mas é meio que a cultura da ufsc fazer esses primeiros contatos durante a primeira semana de aula.

\item Luis: acha que o ideal é fazer a reunião aberta no pré inicio de semestre pra definir o critério.\newline

\textbf{Encaminhamentos:}
\begin{enumerate}
\item Apadrinhamento
\item Manual do calouro com questões do ERE, saúdo física e mental.
\item Discord: acha que deve ser do calico, pode ficar com o curupira.
\item Lives e mesa redonda
\item  Rodas de conversa
\item Evento específico do CALICO
\end{enumerate}


\item Roque: manter eventos, lives, mesas redondas, noites de jogos, etc no discord do Calico. 
Não ter o discord como um espaço formal, mas sim um lugar onde as pessoas possam frequentar.

Curupira: sobre cola, não tem como montar um servidor que não permita compartilhamento de informações.

Roque: questionamento se a questão de colas cai sobre o Calico. Luis acredita que sim, e é melhor evitar qualquer coisa.

\item Luis: sugeriu investir no discord do CALICO, e se não der certo, passar para o fórum do moodle.

\textbf{Caue: proposta de calourada}
\begin{itemize}
\item Segunda 17h (POO termina 16h): videoconferência pra todo mundo se apresentar, junto com veteranos diretos e CA
\item Segunda 18h: Noite de Jogos
\item  Terça 18h: Tour virtual da UFSC e apresentação das nossas entidades representativas e ME
\item Quarta 13h30: apresentação breve do CA e curso na aula de Intro
\item  Quinta 18h: live (mesa redonda) no Discord com profissionais da saúde física e mental
\item Sexta 18h: JoSé (campeonato de LoL? outros jogos?)
\end{itemize}

\item Curupira: acha bom encher os alunos de evento desde o começo, e não deixar os eventos de pra depois de quarta.
Sentiu falta de oficinas e minicursos, e elogiou a seccom.

\item Luis: também acha importante manter os eventos desde o início. Propôs mudar alguns dos pontos de quarta pra segunda, possivelmente na aula de POO.

\item Curupira: acha bom manter coisas dentro da aula pra dar uma validação pro CA na visão dos calouros.

\item Luis: parafraseando Caue no chat "em intro a gente fala das coisas burocráticas, prae, colegiados, etc"
Conversar com os professores de POO pra convidar a galera pros eventos de segunda, etc.

\item Paloma: chamar nas duas turmas de calouro, e não na de veteranos.
\end{itemize}

\pauta{Eventos ao longo do semestre}
\textbf{Luis: eventos de formação:}
\begin{itemize}
\item universidade popular
\item precarização do trabalho
\item modelos de universidade
\item outras discussões possíveis ao longo do semestre, precarização do ensino, modelo de universidade, etc.
\end{itemize}

\textbf{Hans: formações em 3 tópicos:}
\begin{itemize}
\item precarização e direitos trabalhistas, sindicalização
\item formação política, como fazer trabalho de base, etc
\item formações sobre curso, pontos que não sao estudados na graduação, etc
\end{itemize}

\begin{itemize}
\item Caue: colocar nossos colegas pra apresentar coisas interessantes, uma seccom junior, 
apresentar IC, algo entre os estudantes mesmo.

\item Paloma: Sobre a formação do trabalho de base, acha uma ideia boa, mas não acha que é papel do centro academico, e não conversa com a base do curso


\item Luis: as formações podem ser em formatos diferentes, desde rodas de conversas até reuniões semanais pra conversa nas linhas definidas, seja em modelos de universidade, etc.

\item Matrix: também não acho que é um papel do CA (formação do trabalho de base)

\item Hans: discorda do Matrix, acha que é importante fazer formação política, e que é especificamente o trabalho do CA.
Sugeriu discutir um livro sobre trabalho de base. Fazer as coisas de forma orgânica, etc. Mostrou preocupação com a continuidade da identidade e papel políticos do Calico depois que os membros mais políticos saírem da entidade.

\item Roque: Só falei sobre achar importante tentar fazer a formação política em conjunto com assuntos que sejam mais 
palpáveis/interessantes, tentando falar de coisas sobre a estrutura da universidade ou sobre a questão do trabalho
Ah, e que acho bem interessante a ideia do Cauê de fazer eventos sobre TCCs, ICs, estágio, etc é bem legal até 
pra desmistificar muita coisa.

\item Luis: acha importante fazer o trabalho de base, de forma não tão institucional. Informar o papel do centro acadêmico, de formas mais palatáveis pro estudante médio de computação.

\item Matrix: Sobre as formações, colocar elas como grupos de estudos, rodas de conversas, e fazer com que esses eventos virem materiais de estudo (feat Paloma)

\textbf{Sobre frequência de reuniões:}
\item Frequência de Reuniões: Paloma diz que deve ser mais dinâmico, com por exemplo 2 no mês, sem dias certos.

\item Curupira/Hans: Dizem que é bom manter um ritmo mais semanal, ou quinzenal, pra manter o sentimento de reunião

\item No momento, as reuniões serão quinzenais e poderemos adaptar melhor com o tempo.

\end{itemize}

\presentes{Paloma Cione, Matheus Roque, Arthur Pickcius, Luis Oswaldo, André Regis, Hans Buss, Mikael Saraiva, Pedro Aquinom, Cauê Baasch, Bruno por alguns minutos}

\end{document}
