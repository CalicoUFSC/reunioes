\documentclass{ata-calico}
\usepackage{indentfirst}
\usepackage{enumerate}
\pagenumbering{arabic}
\begin{document}

\maketitle

\pauta{Discussões sobre as pautas do CEB}

\begin{itemize}

    \item Paloma: Na conjuntura nacional foi falado sobre a situação da saúde. Foi pensando em fazer post e eventos, como dia 30 vai ter o dia nacional da mobilização. Seria bom montar um comite para organizar a base pra esses eventos. Tem gente fazendo estágio no HU e não tão sendo vacinados.

    \item Hans: A saúde tá em colapso. Bolsonaro não está dando suporte. No HU, só tem medicação para mais 4 entubações por exemplo. Acha que um bom ponto pra se levar ao comitê é quebra de patente da vacina pra ajudar produzir em maior escala.
    
    \item Oswaldo: Fiocruz começou a produção de vacinas. Acha que deveríamos pressionar para agilizar o processo de vacinação. Não adianta pedir só lockdown porque os necessitados morreriam de fome. Até os banqueiros estão achando melhor o lockdown. O problema não é só na saúde, mas na educação também que sofre tentativas de desmonte. Parece um bom momento pra greve, dado que isso implica em os estudantes ficarem em casa seguro.
    
    \item Caue: Bolsonaro tem fortes tendencias facistas. Precisamos denunciar os casos de perseguição política. A maioria dos setores está adotando a postura para combater a pandemia, mas não é o caso do governo federal. Movimento estudantil da UFSC está parado, precisamos do comitê para ajudar a vacinação dos estudantes.
    
    \item Oswaldo: O quanto o Calico tem interesse em participar do comitê? Com o tempo as pessoas vão perdendo o interesse e os problemas acabam caindo sobre poucas pessoas. Devemos ter suporte técnico, afinal estamos numa universidade federal. Apoio a nossa participação.
    
    \item Curupira: Qual o escopo do comitê de vacinação e que propos?
    
    \item Hans: O comitê foi proposto no último DCE, APUFSC E APGU. Está para ser definido, se vai ser apenas da UFSC, se vai ser municipal ou qualquer outro escopo. Acha importante a nossa participação, já que é a vida do estudante que está em jogo aqui.
    
    \item Oswaldo: A atuação do comitê será online, por isso o escopo servirá mais para definir a direção das exigências. A princípio deve ser municipal e expandir conforme necessidade. Acha q esse comitê deveria ser composto por entidades, por questões de organização, mas deveria ser aberto pra toda comunidade, pois é importante a pluraridade de pessoas.
    
    \item Hans: Duas discordâncias. Não deveria ser apenas online, dado que a nossa classe não está tendo direito realmente ao isolamento social. O comitê não deveria ser fechado à entidades, pois é mais fácil a aderência de estudantes que não estão vinculados às entidades.
    
    \item Cauê: No geral concorda. Quer discutir mais sobre a educação. Deveríamos focar mais em como os estudantes estão sendo afetados. Os estudantes vulneráveis estão em condições cada vez mais precárias, sem o que comer e onde morar. Concorda com o Hans, deveríamos considerar ações presenciais, não descartar prontamente, dado que muitos não conseguem se encontrar em situação de isolamento. Acha importante evitar fechar a organização para que o máximo de pessoas possam participar.
    
    \item Hans: Deveríamos focar o esforço no comitê de vacinação, pois engloba muito mais estudantes.
    
    \item Oswaldo: Pede desculpas pelo que foi dito, não acha que deva ser inteiramente digital, mas que será em sua maioria. Apesar do nosso CA não estar tão bem quanto gostaríamos, ainda está funcionando, ao contrário de outros que pararam nesse período, então deve ser interessante deixar o comitê mais aberto. Calico tem condições de participar desses dois assuntos, vacinação e apoio aos estudantes vulneráveis.
    
    \item Caue: A ideia desse segundo assunto é para ajudar os estudantes em carência, pra ajudar o DCE que está fragilizado. Deveríamos colocar o dia 30 na agenda pra fazer um evento grande.
    
    \item Oswaldo: Cauê, concretiza isso por favor.
    
    \item Cauê: DCE tem que construir o dia 30.
    
    \item Oswaldo: Vamos votar sobre o GT da permanência dos estudantes. Hans tem uma fala contra, Cauê uma a favor
    
    \item Hans: Local para discutir é o DCE, propor uma discussão a mais é dobrar o  trabalho. A comunicação do DCE já tá parada. O comitê de vacinação deve juntar mais gente.
    
    \item Cauê: Temos condições de fazer mais que o comitê da vacina. Tem muita coisa que deveria ser do DCE, essa é uma delas. Precisamos encontrar gente para ajudar a produzir material, isso não deveria ser feito apenas em reuniões.
    
    1 contra,
    4 a favor,
    2 abstenções.
    
    \item Oswaldo: Alguém tem algo contra o dia 30?
    
    \item Curupira: Gostaria de saber mais ou menos o que se esperar desse tal evento do dia 30.
    
    \item Hans: A maioria dos atos são virtuais, mas podemos fazer algo presencial. Está sendo proposto pela UNE.
    
    Votação a respeito do dia 30 é unanime a favor.
    
    Hans propões de os CAs redigirem uma nota política do CEB.
    
    Ninguém tem desacordo.

\end{itemize}

\pauta{Divisão de cadeiras do Conselho do CTC}

\begin{itemize}

    \item Oswaldo: Conselho do CTC está acima da direção do CTC. O CETEC que escolhe os representantes, não os CAs diretamente, assim não é necessário ou possível que tenha uma divisão igual entre os CAs. No final do semestre geralmente falta tanta gente a ponto de vagar uma cadeira para sempre.

    \item Paloma: Falta discussão sobre representações falhas no CETEC.
    
    \item Cauê: Pra muita gente o importante é ter diversidade de cursos nas representações. Mas não acha que é o suficiente. Precisamos de gente eficiente para problemas mais difíceis como o do Colle. Precisamos chegar no conselho com uma decisão em unidade. Deveríamos pensar em quem vai ocupar sem levar em consideração o curso.
    
    \item Oswaldo: Ter um mínimo de cadeiras para os centros interessados (no caso 1), talvez um máximo (provavelmente 3) e o resto deveria ser votado.
    
    \item Curupira: Porque 3? É porque já era assim antes? Vocês acham que esse número é suficiente?
    
    \item Oswaldo: É um número arbitrário, mas sim parece ser uma quantidade boa para não virar o monopólio de um CA, mas caso sobre cadeiras que não exista limite.

\end{itemize}

\textbf{Encaminhamentos:} A favor do GT sobre permanência estudantil;\\

\textbf{Encaminhamentos:} A favor de um evento no dia 30 de março;\\

\textbf{Encaminhamentos:} Os CAs redigirem uma nota política do CEB;\\

\presentes{Cauê Baasch de Souza, Matheus Dhanyel Cândido Roque, Julien Hervot de Mattos Vaz, Paloma Cione, Teo Haeser Gallarza, Helena Aires, Nicole Schmidt, Lucas Verdade Godoy, André William Régis, Luis Oswaldo dos Santos Ganoza}

\end{document}
