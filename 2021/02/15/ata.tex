\documentclass{ata-calico}
\usepackage{indentfirst}
\usepackage{enumerate}
\pagenumbering{arabic}
\begin{document}

\maketitle

\pauta{Informes e Repasses}

\begin{itemize}
\item Oswaldo: Teremos horas complementares obrigatória para ingressantes a partir de 2019, documento que descreve tudo que é contabilizado. Fizemos uma avaliação do ensino remoto a pedido do colegiado do curso.

\item Cauê: Próximo encontro do conselho contém apenas assuntos burocráticos, nada de Colle. Um antigo aluno formado, Ranieri, trouxe a nós uma proposta interessante de dar seguimento nos conselhos futuros.
\end{itemize}

\pauta{Dicussão de Conjuntura}

\begin{itemize}
\item Oswaldo: Daremos seguimento nas discussões referentes a conjuntura.

\item Paloma: Estamos enfrentando vários problemas em Florianópolis de saneamento com essa chuvas. Muitas pessoas estão indignadas com o prefeito Gean, principalmente no bairro onde moro onde tudo foi alagado.

\item Cauê: Essa indignação é importante, mas é muito passageira e a maioria esquece até a hora da eleição. Pessoas que participaram ativamente nesse processo de eleição se decepcionaram com a vitória do Gean, mas não é surpresa dado ao investimento feito pela campanha do mesmo. O Gean conseguiu passar por cima da COMCAP e dos processos democráticos envolvidos, ocorrendo com apressadamente para não abrir espaço para debates ou oposição.

\item Hans: Logo depois das eleições tivemos outros problemas, como o aumento das tarifas, não tivemos discussão a respeito disso em entidade alguma. Podemos ver pela greve à COMCAP como os sindicatos foram atacados. Tratando assuntos internacionais, Bolsonaro quer aprovar porte de arma de até seis armas por pessoa.

\item Cauê: A respeito dos problemas nacionais são bem atípicos. O governo federal se mostra muito incompetente em relação a pandemia e tentaram aproveitar ao máximo para passar medidas duvidosas durante esse período conturbado. Mas o Bolsonaro não é incompetente, ele se mostra bastante habilidoso na política quando os interesses do capital estão alinhados. No início, usaram a abordagem de que se as pessoas parassem passariam dificuldades, o que não é completamente errado, mas deveríamos exigir do governo que bancassem essas pessoas vulneráveis.

\item Cauê: O estado banca muito o capital, como quando cobre os gastos de falências, e isso diminui cada vez mais o apoio que poderia ser dado ao povo. A esquerda do Brasil é muito desorganizada, mas acho que a maior causa disso não é falta de comunicação, é uma falta de estratégia. Não podemos esperar dos setores de direita que derrubem o governo Bolsonaro que tem destruído nossa instituição, o meio ambiente... É importante discutir esse tipo de coisa por aqui, pois os estudantes são a base mais organizada e precisamos desse debate.
\end{itemize}

\pauta{Indicação para o Colegiado de Departamento e Conselho do CTC}
\begin{itemize}
\item Oswaldo: Um breve resumo dos colegiados, são os órgãos que tomam as decisões para aquela instância em específica. Por exemplo, existe o chefe de departamento que pode tomar várias decisões, mas elas podem ser revogadas pelo colegiado do departamento. Segundo a resolução nº 017/CUn/1997, nós temos direito a um quinto dos professores como representantes discentes, no colegiado de departamento isso significa cinco estudantes. Logo vencerá as nomeações do colegiado, que duram, se não me engano, um ano e sempre trazemos para ser debatido no CA em reuniões abertas os representantes para serem elegidos. Os representantes não atuam de forma autônoma, todas decisões ou discussões levadas aos colegiados são discutidas em reuniões abertas em algum momento do futuro ou anterior as reuniões. Estamos aqui para discutir as decisões dos antigos representantes e quem deveria ser os próximos ou se deveriam ser mantidos os mesmos.

\item Paloma: Quem são os 5 representantes atualmente no departamento?

\item Oswaldo: Eu, tu, Helena, Djalma e Diogo.

\item Paloma: Bom, eu posso continuar. Posso ver com o Djalma pra continuar, mas acho que ele vai. Temos que encontrar alguém para substituir o Diogo.

\item Oswaldo: Só para esclarecer, achávamos que o colegiado do departamento seria pior, já que é 50 professores com apenas cinco representantes discentes, mas são poucas reuniões, ano passado acho que foi só uma reunião. Estou nos de curso, departamento e centro, e acredito que deveria continuar lá justamente devido a minha experiência. Tento deixar os outros representantes falarem mais para se familiarizarem e entenderem o gingado entre ser firme e irritar os professores, principalmente nos de curso onde têm menos pessoas. Como quando eu e o Cauê tivemos que defender um aluno que estava para jubilar e o professor Cancian estava furioso com ele. Discutimos por um tempo com os professores e conseguimos mais tempo para o tal aluno, que pelo jeito se formou. Com respeito aos outros representantes, acho que são bons só falta um pouco de ímpeto. Estamos batalhando para ocuparmos todos as cadeiras de discentes no colegiado do departamento, onde as duas vagas de SIN não é ocupada há um tempo porque o curso deles não indica ninguém. Pergunto por voluntários, temos duas vagas no departamento.

\item Cauê: Representação do curso está tudo certo?

\item Oswaldo: Sim, só falta gente para o do departamento e do centro.

\item Cauê: Certo, então em relação ao colegiado do CTC a nomeação é feita pelo CETEC, não só por nós no curso. Temos uma vaga de titular, onde o Oswaldo ocupa, e outra de suplente ocupada por mim. Não sei se vão precisar de mais porque as vezes os outros cursos param de participar. Eu tenho disposição de continuar como representante, mas também estou disposto a abrir mão da vaga para outro interessado, mas que tenha disposição de brigar porque é um espaço mais polêmico. Confesso que teve um incidente onde eu fui atacado especificamente devido a falas minhas que me arrependo, onde acabei generalizando alguns pontos demais e ofendido professores que poderiam ter sido nossos aliados, peço desculpas, estou redigindo uma carta de retratação para o próximo encontro e pretendo ser mais cauteloso.

\item Oswaldo: Importante salientar que o Cauê não estava num espaço onde ele estava como representante, foi num chat de Youtube com a conta pessoal dele. Eu coloco a minha vaga disposição de quem quiser no Conselho de Unidade junto com o Cauê, nossas representações foram importantes, mas acho que novos alunos seria muito benéfico. Frequentemente defendemos os alunos nessas reuniões, queremos tutelar gente mais nova porque eu estou prestes a me formar e o Cauê já é velho no curso também.

\item Curupira: Estou a disposição, mas não tenho noção de quanto eu posso ajudar sem me sobrecarregar.

\item Oswaldo: Se você quer ter um gostinho, mas sem entrar de cabeça é bom entrar no colegiado de departamento, onde apesar de haver só uma reunião por ano temos as câmaras, que são subconjuntos dentro do colegiado do departamento que discutem com mais frequência e assuntos mais de nicho.

\item Seis: Não é possível ser apenas ouvinte?

\item Oswaldo: Não tem nada que diga o contrário, mas os professores costumam ser resistentes a abrir as reuniões porque costumam ter opiniões controversas com o resto dos estudantes.

\item Matriques: Não tem vaga de suplente?

\item Oswaldo: Tem.

\item Seis: Queria ter uma ideia da carga horária.

\item Oswaldo: A carga de suplente deve ser no máximo 4h no semestre inteiro, quase 1h por mês. E se o titular estiver presente não precisa nem votar.

\item Seis: Então pode me incluir.


\item Oswaldo: Os representantes precisam ser votados em reunião aberta, vou pedir para que se alguém tem desacordo com o Curupira e o Seis que se manifeste. Esse é um espaço para críticas e ninguém aqui vai levar para o pessoal.

...

\item Matriques: São quantas cadeiras no departamento?

\item Oswaldo: Cinco cadeiras, ou seja, dez representantes, cinco titulares e cinco suplentes. Deveríamos dividir meio a meio com o curso de Sistemas, mas acordamos com eles de 3 cadeiras para nós e 2 para eles, então temos 6 vagas ao todo.

...

\item Oswaldo: Okay, vou interpretar esse silêncio como consenso, farei o mesmo a respeito da minha renomeação e do Cauê.

...

\item Oswaldo: O mesmo vale para a releição dos representantes do departamento, estou considerando que vocês estão concordando com absolutamente tudo que estou falando.

\item Matriques: Realmente, estamos concordando.
\end{itemize}

\pauta{Espaço para eventos}
\begin{itemize}
\item Oswaldo: Obrigado. Último ponto de pauta é a respeito de futuros eventos de integração. Gostaríamos de sugestões, pode ser mais campeonato de jogos, discutir sobre alguma coisa, roda de conversa...

\item Seis: E se fizermos outro JOSE com o segundo mais votado?

\item Oswaldo: Acho melhor fazer uma nova votação excluindo o último jogo votado. Mas podemos fazer algum outro tipo de evento, porque tem gente que não joga nada. Antigamente poderíamos fazer um bar, um linguicinha, infelizmente isso não é mais possível,  mas deixo aberto para outras indicações de eventos que não joguinhos eletrônicos.

\item Paloma: Acho triste que no último evento tivemos apenas um calouro participando. Também podemos fazer eventos visando os veteranos, mas espero que os calouros se sintam mais a vontade com o tempo, talvez com os padrinhos isso melhore.

\item Oswaldo: Inclusive sugiro que os padrinhos questionem os calouros de eventos que poderiam gerar mais interação por parte deles para que o Calico entenda melhor a situação.

\item Roque: Sobre o que eu sei é que a maioria dos calouros já foi contada. Do meu grupo consegui ter mais contato com apenas um calouro, mas de maneira geral o grupo ainda está engatinhando.

\item Oswaldo: Só quero lembrar que coloquem o nome na lista de presença, até pra legitimar as partes de voto.

\item Samuel: Estou envolvido com apenas um calouro, percebi que muitos calouros ficam perdidos em alguns sistemas, como por exemplo ajudar com Linux. Não sei se um minicurso é considerando um evento válido, mas seria bom.

\item Curupira: O apadrinhamento teve a largada prejudicada porque não recolhemos os números de telefone dos calouros, então tivemos que fazer a maioria dos contatos pelo Moodle que é bem mais lento, depende muito dos calouros entrarem por causa de alguma aula e estamos no meio do Carnaval. Contudo, ainda é muito recente, ainda não fechamos nem uma semana desde o apadrinhamento enxergo muito potencial nessa iniciativa. Da minha parte acredito que tive até sorte porque tive calouros muito dispostos, como quando eu fiquei instalando Linux com eles por horas e mesmo os que não conseguiram demonstraram interesse em continuar por conta. Acredito que calouros sempre se perdem com a enorme quantidade de coisas novas que eles tem que se preocupar, mais ainda durante o período remoto, então acredito que os padrinhos podem ajudar bastante nesse sentido.

\item Oswaldo: Acho que podemos encerrar, e peço para que algum padrinho repasse para os outros no grupo que indaguem os calouros a respeito disso. Acho importante o integrações não só de joguinhos, como o Curupira falou, dá para tirar dúvidas de Linux, programação e relacionadas ao curso em geral. Já demos minicursos de como instalar Linux por exemplo, ou de linguagens, principalmente introdutórias como Python. Isso conta como integração e evento acadêmico. Quando falamos do que o Calico pode fazer engloba de tudo, política, programação, joguinhos... Peço que algum padrinho repasse isso no grupo.

\item Curupira: Eu já estou tentando guiar os padrinhos nesse caminho e vejo que o Roque também. Vamos continuar com mais intensidade.

\item Oswaldo: Ótimo, mais alguém tem algo a discutir?

\item Seis: Quão viável é fazermos um minicurso de C++ no fim do semestre? Acho que seria útil para ED.

\item Oswaldo: Acho bom esses cursos, inclusive acredito que os padrinhos deveriam consultar os calouros a respeito desses assuntos. Nós conhecemos pessoas que sabem programar muito bem em C++ e o Calico está a disposição para promover essas integrações.

\item Paloma: Só queria pontuar como é legal ver gente usando o servidor para estudar, como o Chew nesse momento.

\item Oswaldo: Legal, bem legal isso. Se ninguém tem mais nenhuma consideração, podemos dar a reunião por encerrada. Sei que as pessoas que estão chegando agora podem pensar que é chato, mas é bem importante os assuntos debatidos aqui, pois o Calico só consegue agir em cima do que é trazido aqui, é muito importante a participação de todo mundo, inclusive das pessoas que comparecem só de vez em quando. Então é isso, agradeço a participação de todos.

\end{itemize}

\presentes{Teo Haeser Gallarza, Pedro Henrique Aquino Silva, Nicole Schmidt, Paloma Cione, Marcos Tomaszewski, Julien Hervot de Mattos Vaz, Bernardo Borges Sandoval, Helena Aires, Nicolas Dolzan de Araujo, Cauê Baasch de Souza, Arthur Mesquita Pickcius, Bernardo Arruda Silveira, Matheus Dhanyel Cândido Roque, Samuel Cardoso, Wesly Carmesini Ataide, Mauricio Konrath, André William Régis}



\end{document}

