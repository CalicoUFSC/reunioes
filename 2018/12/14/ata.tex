\documentclass{ata-calico}

\begin{document}

\maketitle

\pauta{Eleição do Delegado para o CONEB}
Cauê fala que a importância do evento já foi discutida e que precisamos decidir quem representará o CALICO. Luis diz que a pauta do CONEB será em relação a conjuntura e todos concordam. Lefol, Raineiri, Mikael e Luis se propoem a ir. Luis diz que duas pessoas já seria mais do que o suficiente. Caio diz que seria bom ir mais gente. Lefol mostra que deve ser um titular e dois suplentes. 

O dinheiro do CALICO entra em pauta, Caio diz que poderíamos dividir conforme a necessidade de cada um. Luis mostra o documento que diz que caravanas sairão de todos os estados e que a UNE divulgará uma lista no site em breve. Caio diz que isso é muito vago e não é garantido. Luis diz que deveremos separar uma quantidade de dinheiro para o grupo que vai, dividindo entre o número de pessoas, podendo também cada um ajudar com alguma quantia. Lefol diz que possivelmente consegue alojamento por fora em Salvador e que podemos ir com os ônibus providos pelo DCE, se houver. Se não, podemos contactar DCE's de outras universidades.

Luis diz que seria essencial Lefol e Ranieiri irem, considerando que os mesmos tomariam decisões em nome do CALICO.
Caio propõe Luis ou Patrick como titular e os outros como suplentes. Define-se Lefol como ouvinte, Ranieri e Patrick como suplentes e Luis como titular e todos concordam. Cauê diz que precisamos discutir a questão financeira. Lefol diz que podemos pedir ajuda para a JE para pagar as inscrições. Caio diz que o problema são as passagens, que são mais caras. Luis e Lefol se encarregam de procurar formas de transporte para o evento. Ranieri dia que uma viagem de carro também seria viável e mais barato.

\pauta{Posicionamento do CALICO quanto às pautas do CONEB}
Luis e Caio dizem que pode ser muito difícil alinhar o posicionamento. Fica decidido que as decisões em relação ao evento serão discutidas no local, e qualquer outra pauta colocada antes do evento poderá ser discutida entre todos.


\presentes{Cauê Baasch, João Gabriel Trombeta, Luis Oswaldo Ganoza, Mikael Saraiva, Caio Pereira Oliveira, Ranieri, Paloma Cione, Paulo Default}

\end{document}
