\documentclass{ata-calico}

\begin{document}

\maketitle

\pauta{Contatos para curso de desenvolvedor na UFSC}
Lohan no skype. Facebook apoia em eventos (workshops e etc), Lohan foi rensponsável por fazer contato com universidades em Florianópolis. Pediu sobre a possibilidade de divulgar e fazer eventos na UFSC, a grande maioria dos eventos são gratuitos e a participação permite a participação em novos eventos e networking. Gostaria de se aproximar do CA para se aproximar do curso e oferecer cursos. Dia 8 promoverão um curso de React, por exemplo.

Caio apoia minicursos gratuitos e diz que conseguimos espaços na ufsc para eles. Lohan comentou fazer um minicurso na UFSC a cada dois meses. Foi comentado sobre fazer já dia 8 porém é dia de vestibular, Lohan então sugeriu semestre que vem convidar líderes do projeto de Florianópolis para apresentá-lo para os alunos e fazer um planejamento conjunto. Luis pediu até quando precisamos responder, Lohan respondeu que até janeiro ou fevereiro, Luis diz que precisamos conversar e dar a resposta. Caio concorda e que responderemos nesse período (inclusive comentando nas próximas semanas). Lohan pediu divulgação do evento dia 8, Caio concordou e diz que compartilhará na página do CALICO.

Os eventos parecem ser similares com o que o Mozilla faz, não patrocinado pelo Facebook mas recomendado.

Luis não gosta muito da ideia pois parece coisa de desenvolvedor e o quanto isso se aplica ao perfil do curso. Caio discorda. Cauê diz que não vê problema em cursos gratuitos, mas quanto mais coisas de sistemas fizermos em computação menos conseguiremos diferenciar os cursos. Caio diz que também é computação já que é o futuro da maioria dos alunos. Cauê argumenta que ainda não é o foco do curso. Caio diz que como CA não podemos ditar o que é ou não é computação, se existem pessoas interessadas devemos promover.Lefol comenta que entende o ponto contrário e concorda mas os cursos são sobre ferramentas que também são úteis para computação.

\pauta{CONEB}

Conselho Nacional de Entidades de Base, uma instancia deliberativa na UNE, no qual todo CA pode ter um representante. Trata-se de um evento bianual onde se discute qual o caminho a UNE seguirá, e as inscrições encerrarão em breve. Lefol comenta que seria legal ter representantes do CALICO. Luis pediu para Defol mandar links de referência, Lefol diz que fará. Cauê diz que isso é política nacional, Lefol disse que temos tempo para discutir essas pautas e eleger um representante. Lefol comenta como é um evento importante e Luis concorda. Na pior das hipóteses o representante apenas acompanha e se abstém de votações, diz Luis. Caio comenta que devemos discutir se é viável, o que é concordado.

Lefol encaminhará mais informações para que seja discutida a viabilidade.

\pauta{Identidade visual do CA e imagem com discentes e docentes}

Cauê comenta que devemos ser mais transparentes, avisando antes e enviando ata mais cedo. Trombeta e Luis falam como a ata precisa ser revisada, mas até o final do dia da reunião é viável.

Cauê fala como as pessoas não vêem muito bem o CALICO. Luis comenta como existem duas opções para reverter isso, focando no espaço físico e eventos, e que para eventos está tarde como o semestre está acabando. Caio concorda.

Luis pede para esclarecer o que foi discutido na reunião anterior. Caio comenta como no planejamento estratégico podemos pensar em eventos com outras entidades para depois discutir com elas. Luis diz como seria interessante ter um planejamento estratégico com representantes de outras entidades. Cauê acha que deve ser feito antes do inicio do semestre, todos concordam e Caio comenta que deve ser depois do nosso PE. Conversar com outras entidades para que elas tenham calendários planejados para que tenhamos essas reuniões conjuntas para organizar.

Luis é designado para definir datas e organizar os planejamentos de PE. 

Brenno comenta como é preciso ainda parafusar melhor a porta.Também é ainda preciso cobrar da pessoa que quebrou a porta. Patrick possui o manual da porta caso ele seja necessário.

É concordado que datas, como quando fazer a logo para pôr na porta, devem ser decididas no PE.

Foi também decidido que em reuniões futuras essa pauta deve ser dividida em duas, e será discutida melhor no PE.

\pauta{Colegiados}

Caio foi desligado do colegiado, existe a necessidade de escolher quem ocupará novas cadeiras. Precisamos de uma ata de posse da Comissão Eleitoral e de pessoas para buscar as assinaturas necessárias no documento. A nomeação das cadeiras devem ser feitas o quanto antes, como ainda não tivemos tempo de montar portaria Caio comenta como provavelmente perderemos próxima reunião. Luis e Caio concordam como devemos conseguir logo os documentos.

Já temos nomeações para cadeiras, nomeações estas apresentadas no grupo. Ninguém teve objeção das nomeações. As nomeações de colegiado de centro serão enviadas ainda hoje para Gabriela. As nomeações que temos até agora são: Colegiado de centro - Caio e Trombeta. Colegiado de curso - Patrick e Cauê. Colegiado de departamento - Luis, Paloma e mais alguém, o número de cadeiras ainda não nos foi informado. 

Caio diz que para o colegiado do curso, após conseguir os documentos, precisaremos apenas dos suplentes.

Caio diz que pode conseguir a ata de posse ainda hoje e Cauê se ofereceu para ajudar. Ainda é necessário conseguir os suplentes para o colegiado de curso e o resto dos nomes do departamento, este tendo como impasse a falta de certeza de quantas cadeiras temos, Luis ficou reponsável por contatar o professor responsável pelo cálculo de cadeiras. 

\presentes{Cauê Baasch, João Gabriel Trombeta, Luis Oswaldo Ganoza, João Paulo Tiz, Mikael Saraiva, Caio Pereira Oliveira, Brenno Araújo, Nicole Schmidt}

\end{document}
