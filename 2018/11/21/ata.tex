\documentclass{ata-calico}

\begin{document}

\maketitle

\pauta{BENTO}
Discutindo a viabilidade de um evento BENTO, chegasse a conclusão que o maior impasse do evento é a data, uma vez que estamos no final do semestre e a Amnésia ocorre dia 01/12, antes disso ocorrerão as provas e isso diminuiria em grande quantidade o público do evento, a melhor possíbilidade seria na sexta-feira 30/12.

Felipe sugere fazer o evento em algum bar como um evento que aconteceu no bar do
Maneca no início do ano, todos concordam que é uma boa ideia apesar da prova de 
Calculo I no sábado a tarde.

Todos concordam em fazer dia 31/12 em um bar a decidir ao longo da semana.

Patrick fica responsável por lidar com o evento.

\pauta{Calendário de eventos}
Cauê cita que precisamos nos organizar e fazer um calendário de eventos para o início
do próximo semestre, organizar bem as datas e eventos.

Lefol cita que seria uma boa ideia fazer uma visita aos laboratórios do departamento.
João Gabriel fala que pode não ser tão interessante fazer uma visita aos laboratórios
e que isso já é feito na aula de Introdução a Computação. Caio concorda.
Lucas responde dizendo que o clima não é o mesmo e que poderia ser uma conversa mais
informal e mais animada para os calouros.
Cauê diz que é muito provável que os calouros não se interessariam nos laboratórios.

Todos concordam, então foi sugerido um apadrinhamento de calouros para uma possível
orientação dentro da UFSC.

É dito que além dos eventos já feitos todo semestres precisamos organizar um
encontro no dia da matrícula e Patrick também cita que na terça da primeira semana 
de aula podemos fazer algum evento das 15 as 18 horas.

Será discutido colocar no calendário: Recepção no dia da matrícula, Linguicinha, 
evento na terça-feira, JOSE.COM, CHICCO, BENTO, entre outros.

\pauta{Identidade Visual}
Cauê cita que o Calico tem uma certa imagem negativa tanto com alguns docentes e 
discentes e diz que precisamos fazer alguma coisa para mudar essa imagem, tanto 
mudando o espaço físico para um espaço mais receptivo e também trazer algumas coisas
mais perto dos alunos. É citado que o feitio de um camiseta do curso pode ajudar.

João Gabriel cita que o Calico deveria se abrir mais e perguntar o que tu estudante
deseja que o Calico faça, Paloma e Caio citam que muitas vezes o estudante não sabe
sobre as coisas que o Calico faz e que talvez isso não funcionasse, Simonetto sugere
abrir uma enquete em alguma rede social e talvez com as respostas poderemos fazer um 
brainstorm. Todos concordam.

Sobre o espaço físico, Patrick cita a doação de um sofá, compra de puffs, mais uma mesa para estudo. Sampaio cita a compra de banquinhos de plástico. Todos concordam.

Limpeza (03/12) 15:00: Paloma, Luis Oswaldo, Lefol, Caio, João Trombeta, Mikael e Cauê.

Fazer camisetas do curso.

\presentes{Cauê Baasch, Patrick Machado, Paloma Cione, João Gabriel Trombeta, Luis Oswaldo Ganoza, João Paulo Tiz, Lucas Sousa, Mikael Saraiva, Gabriel Sampaio, William Kraemer, Gabriel Simonetto, Ranieri Althoff, Caio Pereira Oliveira, Wesly Carmesini, Felipe Santos}

\end{document}
