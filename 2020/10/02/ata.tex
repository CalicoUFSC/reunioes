\documentclass{ata-calico}
\usepackage{indentfirst}
\pagenumbering{arabic}
\begin{document}

\maketitle

\pauta{Informes e Repasses}
\begin{itemize}
\item Paloma: Sexta passada teve noite de jogos, em média teve 30 pessoas. Como agora elas são quinzenais, sexta que vem (dia 10) vai ter de novo às 19h.
\item Cauê: O secretário do curso, André, voltou de férias agora e a situação deve estar se normalizando de novo. Além disso o Fletes não é mais coordenador, passou as responsabilidades pro Alexandre. Deve rolar ainda esse ano eleição pra coordenação do curso.
\item Cauê: No moodle do colegiado do curso está tendo discussões de curricularização da extensão e atividades complementares. Precisamos discutir sobre isso.
\end{itemize}

\pauta{Eleições do CALICO}
\begin{itemize}
\item Luis: O CALICO tem eleições anuais, que devem ser feitas pelo menos 3 semanas antes de terminar o segundo semestre letivo do ano. Precisamos pensar nesse ponto, discutir o que seria melhor fazer: esperar uma comissão se inscrever e tocar o processo como sempre ou chamar assembleia pra prorrogar a gestão, por conta da pandemia.
\item Cauê: Eleições sempre são bem importantes, o período de campanha em específico é um momento para se falar sobre o que um centro acadêmico deve ser e fazer. Na situação atual, esse processo seria bastante complicado, a qualidade das discussões seria bastante perdida. Existe um consenso no movimento estudantil de fazer eleições assim às pressas. Prorrogar a gestão parece um bom caminho, mas essa não é a instância para se decidir isso. Chamar uma assembleia parece a solução, ter um esforço maior pra conseguir divulgar isso para todos e conseguir fazer o processo das eleições de uma forma melhor depois.
\item Luis: (AAAA)
\item Hans: Devemos estar chamando uma assembleia para falar sobre isso. É bem complicado estar fazendo todo o processo da eleição online. Isso diminui muito o contato com os estudantes, acaba tendo menos chance de aproximar os estudantes do centro acadêmico.
\end{itemize}

\textbf{Encaminhamento:} Chamar assembleia para decidir se prorroga ou não as eleições. Data será tirada pela gestão.

\pauta{Boicote às eleições do CTC}
\begin{itemize}
\item Luis: Nas últimas reuniões já se foi falado sobre as eleições, em relação à discussão que se teve sobre o peso dos votos. Não conseguimos passar o voto paritário e foi proposto o boicote às eleições para se expressar de forma contrária à falta de voz dos estudantes nas eleições para direção de centro.
\item Luis: A ideia é que o CALICO leve ao CETEC a proposta do boicote, com toda uma campanha mostrando realmente a dessatisfação dos estudantes quanto a isso. Não é apenas falar para os alunos não votarem, mas realmente uma campanha explicando toda a situação e por que isso foi proposto.
\item Cauê: A derrota na questão da paridade dos votos é algo inaceitável; tentamos ter alguma participação no processo, que atualmente não temos. A ideia é realmente denunciar, expor o absurdo disso. Não temos reais perdas ao boicotar as eleições, pois nossos votos já são basicamente inexpressivos. Não basta só o boicote, precisa de uma campanha, de toda a propaganda. É importante que seja tirado por todo o CETEC. O ideal, se não estivéssemos na pandemia, seria fazer assembleia em todos os cursos.
\item Paloma: O CETEC já mostrou que consegue mobilizar bastante gente. As condições não são as mesmas, mas por ser justamente algo do CTC tem a chance de conseguir mobilizar muita gente. Existe sim condições de fazer esse boicote.
\end{itemize}

\textbf{Encaminhamento:} Levar a proposta do boicote pro CETEC.

\pauta{Avaliação do ERE}
\begin{itemize}
\item Luis: Dia 29/09 seria o dia que começaria o período de avaliação do ensino remoto emergencial. A reitoria não se mexeu para puxar essa avaliação, então o DCE tomou isso nas mãos para mobilizar a galera. É importante o CALICO fazer essa avaliação dentro do curso não só para fazer esse repasse para o DCE, mas também ter a chance de tentar resolver os problemas que apareçam.
\item Paloma: Dia 28/09 teve prova de formais, com a prof. Jerusa. A prova não foi navegável, ou seja, só podia ver a próxima questão quando entregasse a anterior e não tinha chance de voltar. Inicialmente tinha um período de 2h para se fazer a prova, que tinha 11 questões, com 5 de verdadeiro ou falso e mais 6 discursivas. Quando deu as 2h, a prof. abriu mais 1h porque viu que ninguém ainda tinha conseguido. A prova foi muito extensa, com um conteúdo muito cansativo que não cabia realmente nas 2h iniciais.
\item Cauê: Agora o DCE está entrando em contato com todos os centros e diretórios acadêmicos para conseguir juntar tudo. Não tem como o DCE sem sentar com todos cas realmente juntar todas as demandas dos estudantes. Além disso, também está em disputa a questão da menção P e de contar ou não o IAA, que está tentando fazer com que seja opcional. Um problema que eu percebo é a questão da frequência, que tem vários problemas. Há uma luta pelo DCE para que se faça com que a frequência não seja contada para reprovação e até mesmo para provas de recuperação.
\item Julien: Tenho várias reclamações. Tenho POO2, com os profs. Jônata e Mateus Grellert, e os professores não estão sabendo dosar a quantidade de exercícios pedidos. Em SD, com o prof Guntzel, as aulas estão muito extensas e ninguém está conseguindo aprender direito. GA não está sabendo dosar o tamanho das videoaulas, que estão muito extensas (algumas chegando à 5h).
\item André: De maneira geral, os professores não têm uma didática boa, mas isso é complicado com a falta de preparo que se teve. Em concorrente, as aulas síncronas não contam presença, o que conta são questionários que são semanas/quinzenais; o problema é que os questionários são muito mais profundos do que as aulas dadas. Em SD, com Cancian e Guntzel, estou com problemas para contatar os profs, preciso resolver coisas e eles não dão retorno.
\item Luis: Foi trazido algumas problemáticas: tem um aluno fazendo cálculo 2 , com o prof Daniel, que, antes da prova, teve uma lista obrigatória valendo nota com mais de 100 exercícios; a prova da matéria chegou a demorar 6h para conseguir ser feita.
\item Hans: Concorrente realmente está sendo meio pesado. Em ModSim teve alguns problemas em questão do tempo das atividades, que é muito pequeno para o que precisa ser feito. Em ED, as aulas do prof. Alexandre estão bem ruins.
\item Arthur: Em modelagem e simulação teve uma avaliação que era para ser realizada durante a aula. Mas a aula era síncrona e o professor deu aula durante todo o horário. Então ele deu uma tolerância até 22h, mas precisava de um simulador e só consegui terminar o download meia-noite. Mandei mensagem pro Cancian e ele disse que a nota não ia afetar tanto (lembrando que a média é exponencial), mas que podia rever isso caso falte nota final do semestre.
\item Cauê: Em questão do cansaço, como está? Dá pra perceber que muitos dos docentes estão sentindo a dificuldade pra eles, mas falta empatia para com os alunos. Muitos professores estão bombardeando os alunos com exercícios e a carga horária das pessoas acaba subindo muito. Em questão da qualidade também, é importante avaliar. Em geral, as disciplinas que eu estou cursando presencialmente já são ruins, mas com o ensino remoto a qualidade caiu ainda mais.
\item Julien: Um amigo meu acabou trancando algumas matérias justamente por não estar conseguindo focar em tudo por causa da carga horária. Alguns professores acabam agindo meio desapontados quando os alunos não conseguem fazer todas as tarefas, o que pode adicionar um peso sobre os estudantes.
\item Hans: a qualidade da bibliografia também caiu muito em algumas matérias. Um exemplo é em cálculo 2, onde a bibliografia mudou para livros muito obscuros com exercícios muito difíceis e gabarito errado. Houve outro relato de prova de cálculo 2, onde o aluno acabou só conseguindo completar a prova depois de 12h.
\item André: é interessante pontuar também o oposto. Pelo menos pra mim, algumas aulas se tornaram mais fáceis de suportar. Uma aula só eu conseguiria dizer que está realmente boa: a parte prática de SD com o Mateus, que o professor criou uma boa dinâmica.
\item Paloma: esse semestre eu tinha pego 5 matérias. Acabei tendo que sair de uma, pois não consegui seguir com a carga horária; os exercícios também não batiam muito com o conteúdo passado nas aulas. As aulas de BD1 com Ronaldo e formais, por outro lado, estão sendo bem participativas, o que é bom. As aulas de ESI, da Vilain, estão com muito pouca qualidade, a professora apenas deixa a própria videoaula rodando.
\end{itemize}

\presentes{Cauê Baasch, Helena Aires, Luis Oswaldo, Paloma Cione, Teo Gallarza, Arthur Pickcius, Hans Buss, Pedro Aquino, Julien Vaz, André Régis}

\end{document}
