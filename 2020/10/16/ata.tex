\documentclass{ata-calico}
\usepackage{indentfirst}
\pagenumbering{arabic}
\begin{document}

\maketitle

\pauta{Discussão sobre os problemas relacionados as avaliações e métodos de ensino dos professores de Sistemas Operacionais 2 e Modelagem e Simulação}

Recebemos muitas reclamações nas ultimas semanas em relação ao comportamento dos professores tanto quanto os conteúdos que os alunos precisavam estudar. 
\begin{itemize}
\item Balde: Em SO2, Diz que está impossível de entender o conteúdo, as aulas não são produtivas e o professor se comporta mal, com falta de respeito. OS trabalhos são passados sem muita informação, ele não responde perguntas, é grosso. 

\item Luis: Com relação ao Guto, temos que enfrentar também a forma como a disciplina está organizada. Há quizzes semanais que são relativamente fáceis, sem muitos problemas, o que dá a sensação que você está indo bem e consegue acompanhar a disciplina, mas quando chega o seminário se vê que o problema é muito maior e requer muito mais tempo, sem muita clareza do que se tem pra fazer. A carga horária da disciplina é muito maior do que o previsto e os alunos acabam se perdendo, gastanto muito tempo se dedicando. Seria bom que as partes dos trabalhos fossem entregues parcialmente, para os alunos conseguissem entender melhor os processos. O Guto pode ser uma pessoa muito prepotente, e os alunos se sentem mais confortáveis conversando com o estagiário de docência. Ele não leva em conta que o calendário acadêmico não começou em março, e diz que os alunos tiveram desde esse mês para começar os trabalhos. Então, há problemas didáticos e pedagógicos na disciplina.

\item Pico: Sobre Mod e Sim, os alunos tiveram problemas em relação ao material, já que a apostila ainda estava em produção. Ao mudar de material, os exemplos eram genéricos e os  exemplos não eram suficientes para cobrir os exercícios. Os alunos gastaram muito tempo tentando resolver os exercícios e não conseguiam responder, e ao entrar em contato com o professor, ele não tirava duvidas de forma proveitosa. As avaliações semanais também demandam muito tempo, e o professor diz que os exercícios são pra ser realizados durante a aula, mas ele dá aula normal no horário, e os alunos só conseguem responder depois. Depois de alguma reclamação, ele propôs uma mudança na avaliação dando mais peso pro artigo. Ao ser questionado mais uma vez, respondeu de forma grosseira culpando os alunos por demorar muito.

\item Trombeta: Em relação aos exercícios feitos em horário de aula (Mod e Sim), quando o professor mudou a bibliografia da matéria, os exercícios ficaram muito desproporcionais. Diz que passou mais de 15 horas para responder, chegando na hora da entrega sem muita coisa pra entregar. Os Alunos fizeram uma thread no moodle dizendo como estava muito cansativo e que demandava muito tempo. A resposta do professor. foi grossa, dizendo que não ia dar mais tempo, mas ia redimensionar os exercícios, dando na verdade menos tempo. Depois, ele admitiu que de fato percebeu que os exercícios demandavam muito mais tempo do que o esperado, e se propôs a dimensionar melhor. Ele vai tentar diminuir o peso dessas pequenas avaliações, para poderem se dedicar mais ao artigo. Parte do problema foi também como ele se posicionou, sendo grosso e depois admitindo que estava errado.

\item Pico: Outro problema foi em relação aos atendimentos, já que por causa dos artigos eles são quase obrigatórios. Diz que tinha selecionado um modelo de artigo e pediu pro professor verificar se o tamanho estava aceitável, e se passaram 2 semanas sem resposta. No atendimento ainda comentou sobre o modelo, mas o professor disse que estava sem tempo. (Mod e Sim) 

\item Cauê: No ano passo, também recebemos reclamações sobre esses professores. Quando fomos conversar, os professores se mostram muito dissimulados, com dificuldade de assumir erros, com um complexo de superioridade. O Calico está disposto a conversar com os professores na presença da coordenação. Porém, é difícil nutrir esperanças de que eles se tornarão professores com mais empatia, tolerância e paciência. Se há alguma medida que podemos exigir, como coisas concretas fora do plano de ensino, coisas fora da legislação, é mais fácil de confrontar. Para além disso, sabemos que eles vão arrumar uma forma de fazer coisas do que jeito que querem. Não está claro em nenhuma resolução o que se deve fazer em casos como esse. Então, se quisermos um impacto maior, devemos demonstrar de alguma forma maior nossa insatisfação, como abaixo assinados, pedidos de afastamento. O problema aqui é como colocar isso de forma que isso não crie uma guerra, com muito atrito.

\item Pico: A thread que foi feita no moodle foi delatada, de forma a tentar esconder o problema.

\item Balde: É muito provável que haja atrito com os professores caso peçamos os afastamentos, e seria melhor vermos as nossas opções, já que ao se afastar um professor ele ainda pode lecionar alguma outra disciplina do mesmo jeito.

\item Cauê: Legalmente, não há procedimento sobre alunos tentando afastar alunos. Porém, os professores não são donos das disciplinas, estas são organizadas pelo departamento, que distribui os professores. Não há nada que defina algo sobre nossas ações, mas podemos tentar algo de forma a constranger e queimar publicamente. Devemos tentar conversar com os professores, trazendo depois pra reunião quais foram as respostas, e assim pensar no que faremos.\newline

\textbf{Encaminhamentos:} Marcar conversas com a coordenação e talvez envolver o departamento de alguma forma.

\end{itemize}

\pauta{Repasse do CETEC - Boicote às eleições do CTC}
\begin{itemize}
\item Luis introduz a pauta falando sobre a sessão do conselho do voto paritário. O CALICO levou a proposta ao CETEC para organizar um boicote em cima das eleições, fazendo uma campanha de consciência entre os estudantes da importância do voto. Outros CA's levaram a proposta completamente contrária de fazer uma campanha pró-votação, promovendo debates, fazendo vídeos explicando sobre as chapas e sobre o processo. O CALICO pediu então que os vídeos deixassem claro que o processo não é feito de forma democrática. Então, é essa a postura que o CETEC irá tomar.

\item Cauê: Fala sobre o episódio em que os alunos da arquitetura fizeram um boicote sobre as eleições. Não precisamos entender o boicote como voto vencido, apesar de não ter unidade com o CETEC, o CALICO pode tomar essa posição de ser contra, um boicote só da computação, apesar de não parecer muito proveitoso. 
\end{itemize}

\pauta{Avaliação do ERE}
Abrimos nossa reunião para recolher os relatos.

\begin{itemize}
\item Cauê: Esta sendo disputada a questão da menção P, a não inclusão do FI e também a contabilização do IAA. A menção P seria a menção I durante a pandemia, podendo ser atribuída sem problemas.

\item Sem mais relatos essa semana.
\end{itemize}

\presentes{Paloma Cione, João Paulo T.I.Zanette, Telmo da Silva, Matheus Roque, Arthur Pickcius, Luis Oswaldo, Gabriel Trombeta, Helena Aires, Cauê Baasch, Bruno Huebes, Cristian Alchini, Gabriel Baldessar, Lucas Suppes, André Regis}

\end{document}
