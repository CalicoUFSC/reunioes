\documentclass{ata-calico}
\usepackage{indentfirst}
\pagenumbering{arabic}
\begin{document}

\maketitle

\pauta{Informes e repasses}
\begin{itemize}
    \item Cauê: Semana passada aconteceu reunião com a reitoria, com formato de audiência. Estavam presentes figuras da ADM Central. Ponto de discussão era o plano de auxílio para substituir o RU com objetivo de garantir a alimentação dos estudantes. No início o valor era de 200 com promessa de revisão conforme a certeza do orçamento. Foi exigido, pelos representantes dos estudantes, que fosse aumentado para 600 reais conforme o custo de vida em Fpolis (cálculo baseado no custo médio das cestas básis, feito pelo DIESE)*. Não foi aprovado por culpa da falta de orçamento. Cortes previstos para o ano que vem deixam a situação ainda mais complicada. Foi exigido que a ADM Central se comprometesse a articular com demais setores (MEC entre outros) para garantir recurso para isso.
    
    \item Paloma: Comissão eleitoral de CCO. Aberto o período de inscrição, porém nenhuma chapa inscrita. Previsão é que nenhuma chapa se inscreva. Alexandre é uma boa possibilidade para tentar pressionar a se inscrever. 
\end{itemize}

\pauta{Discussão assembleia}
\begin{itemize}

\item perdi a fala da paloma, foi mal

\item Luis: Acha que a ideia das imagens é realmente isso, não dá pra passar tudo o que a gente precisa nas imagens, colocando apenas os pontos principais/chaves, e esperar que as pessoas leiam o resumo que está embaixo. O ponto mais interessante da assembleia foi criar algum lugar de forma a promover a ideia de unidade, fazer a galera trocar ideia entre si, seria bom tirarmos nessa reunião uma forma de integrar as turmas entre si, e com o resto do curso. Seria especialmente interessante criar esses espaços também pra cada turma, já que podemos ter um contato maior, pois é fácil de conhecer alguém da sua turma. O Calico poderia criar um discord com salas para cada turma, por exemplo.

\item Tiz: O maior problema e que estragou a integração dos calouros foi culpa de comportamento inadequado de alguns veteranos (através do grupo de WhatsApp). Sugeriu criar um código de conduta que garantiria o comportamento em grupos do Calico, evitando comportamento imaturo ou inadequado dos integrantes.

\item Paloma: Esqueceu o que ia falar. A ideia do Discord é legal, mas teme que não vá criar engajamento. As turmas já possuem discords que frequentam. Seria interessante fazer esses espaços, mas o discord não parece a melhor opção. Com relação aos veteranos com comportamento inadequado, seria interessante criar regras de utilização dos grupos e não um código de conduta.

\item Curupira: Deu branco. Paloma já demonstrou preocupação com a falta de engajamento. Acredita que os espaços se constroem ativamente e não da noite para o dia. Seria interessante popular o discord do Calico com pessoas que participem ativamente, de forma a gerar engajamento dos demais. Acredita que deva ter um espaço específico para discutir questões de conduta, mas teme estar exagerando a importância da discussão

\item Paloma: Paradoxo do engajamento: precisa ter gente para chamar mais pessoas. Acha que o foco no discord não é bom, devendo pensar em outras alternativas. Lembrou do conselho de classe sugerido pelo Hans na assembleia.

\item Luis: Acha que isso poderíamos criar o discord, tirando responsáveis, e ver no que dá, chamando a galera, principalmente calouros, justamente pra tentar chamar a turma inteira. O Cauê tinha  pensado em algo sobre os conselhos de turma, de forma a fazer de forma melhor (sem muitas repetições, ou sem muitos conselhos), Se efetivamente construirmos conselhos, devemos tirar pessoas para construir essa ideia.

\item Tiz: Sobre a composição do conselhos/grupos de turmas, Escolher alguém responsável para dar repasse das reuniões e noticias dos Calico para outros grupos, focando nos calouros. Formato similar ao de conselho de classes, onde teriam responsáveis por turmas. Acredita que ajudaria a gerar interesse e engajamento pelo Calico.

\item Hans: Ta perdido por ta em 2 reuniões. Discord não deve ser o foco. Reitera ideia do conselho de classes dada por ele, que é algo que já acontece em outros cursos como medicina. Acha que o problema não é a falta de repasses mas sim não fazer os estudantes ao nosso redor (que frequentam as reuniões, participam do grupo aberto, etc) se interessarem pela luta estudantil.

\item Cauê: Parcialmente alheio as discussões, pois faltou à assembleia. A ideia de um conselho de representantes de turma é interessante, por mais que pouco comum. Garante capilarização das discussões, não dá pra imaginar que a gente consiga 400 pessoas numa reunião, dividir por turma ajuda com isso. A natureza do nosso curso torna complicada agrupar por turmas. Começou a brincar de puxar dados do CAGR e do fórum para testar formas de conformar  grupos. Uma decisão desse tipo precisa de mais participação dos nossos colegas. Gerar acúmulos ao longo das semanas para tomar a decisão. O momento de agora não é o ideal para isso em vista das dificuldades apresentadas pela pandemia

\item Hans: Vamos precisar de mais discussões por ser algo bem complexo. Como integrar mais as pessoas que tão fazendo parte das reuniões mas não da gestão. Pensar em como inserir elas nas atividades do Calico. Fica feliz por estar vendo gente nova.

\item Luis: Concorda com o Cauê, o momento que estamos passando é muito complicado para tomarmos decisões, mas até o termino do período precisamos gerar engajamentos, ou seja, se essa não é a solução agora, precisamos criar outra. Daqui a pouco tempo precisaremos de novas pessoas para ficarem na gestão do Calico, seria interessante buscar uma forma de engajamento já agora.

\item Cauê: Nesse espírito que o Hans colocou, é importante que estejamos dando abertura para as pessoas que querem participar da gestão do Calico. Claro que o grupo é uma forma de participar, mesmo que uma forma mais distante. As pessoas que se demonstram mais interessadas podem ser puxadas para o grupo da gestão. Devemos pensar em formas de cooptar pessoas para a gestão, com algum cuidado.

\item Paloma: Sugestão é parecida com a do Cauê. Colocar as pessoas que tão na reunião de agora e na assembleia e que tem interesse, no grupo da gestão. Colocar as tarefas no grupo do Calico é pouco proveitoso. O grupo é grande demais e as tarefas ficam jogadas ao vento. Grupos menores é mais fácil de organizar. Criar um novo grupo ou adicionar no grupo atual.

\item Hans: Tem acordo de colocar pessoal novo no grupo da gestão. Devemos usar mais o grupo aberto do Calico porque é lá que o pessoal novo acaba entrando e tendo contato com o Calico. Quando entrou os membros podiam pedir pauta pelo grupo aberto e acha que isso pode incentivar mais a participação

\item Nai(ara?): Também acha que o conselho de classe não vai dar certo. Para criar um centro de união não é a melhor estratégia. Talvez criar grupos de discussão (no discord, acho). A noite de jogos até gerou participação mas tem pessoas que podem não querer participar de eventos com esse formato. Acha que seria interessante criar grupos sobre temas de ligados à computação (IA, Hardware). Divulgar mais esse grupo de que já existe (CCO-UFSC não faço ideia do que é isso).
\end{itemize}

\pauta{Pauta CEB: Vestiba}
\begin{itemize}
    \item CEB foi convocado com 2 pautas. Não vou explicar elas aqui.
    \item Luis: Com relação ao vestibular, acha que precisamos tirar uma posição. Não há condições de fazer um vestibular presencial num momento como esse, mas também precisamos colocar em pauta que muitas pessoas serão prejudicadas, já que isso foi divulgado depois do término das inscrições do ENEM. Acha que nós precisamos deixar bem claro que a pandemia não começou há pouco tempo, e sim desde o começo do ano, ou seja,  a reitoria apresentou uma falta organização e capacidade de pensar em coisas como essa. O vestibular não pode ser feito presencialmente, mas a reitoria deve pensar em jeitos de remediar a situação, e não apenas fazer o ingresso pelo ENEM/SISU
    
    \tem Hans: Acha que a gente tem que estar pensando que primeira que o vestiba é excludente. Para ele não ser excludente ele não pode existir. Acha que o movimento estudantil deveria estar pensando em alternativas ao vestibular. DCE fez reunião com cursinhos populares. Proposta do ingresso ser feito pelas notas de vestibulares anteriores (, notas de médio/técnico/ENCCEJA/EJA. Não tenho certeza se essas foram as propostas feitas pelo Hans, mas foram as discutidas em reunião). Existe uma disparidade muito grande entre o hist. escolar de escolas diferentes. O CUn é na terça feira, não temos tempo de fazer nada.
    
    \item Cauê: Acordo com o que o Hans falou. Vestiba é excludente, ele privilegia alguns grupos. Ações afirmativas ajudam mas não resolvem. A gente não pode pensar em não soluções agora. Notas desde 2009 ajuda a não excluir, mas não é o suficiente. Pessoas que estão tentando pela primeira vez não tem chances por culpa do método escolhido. Concorda com o Enzo e acredita que o esforço da nossa universidade deveria ser de articular com o MEC para reabrir as inscrições. O que a gente ta vendo é a preparação sendo feita para o presencial e às pressas cancelou e preparou essa proposta, que superficialmente é boa, mas que deixa diversas lacunas, como as já listadas até agora (excludente, não dá oportunidade pra quem nunca fez ENEM, etc). Sabemos que o MEC está bem complicado, mas a UFSC deveria procurar soluções melhores, que compreendam as nossas propostas.
    
    \item Hans: É contra a proposta de solicitarmos a reabertura do ENEM. Se levamos em conta que o vestiba é excludente. Não devemos propor isso sem discutir as realidade do ENEM. Devemos discutir com os demais CAs da UFSC. Propõe a criação de um GT na CEB de amanhã. Devemos discutir que possibilidade de entrada por sorteio. O vestiba não é algo natural e a forma dele é definida pela gente. Na Argentina as coisas não são dessa forma, e sim por sorteio (pelo que eu entendi)
    
    \item Luis: (Ninguém anatou a fala. O que segue é um tentativa de repassar as ideias abordadas durante ela). Concorda plenamente que o vestibular é excludente, mas precisamos de uma solução para garantir o ingresso dos estudantes. A mudança na forma do como o vestibular é feita não acontece de uma hora pra outra. Precisamos não só pensar em alternativas, como convencer os estudantes e a população que esse novo método de ingresso é melhor que o vestibular. Não temos tempo nem condições de fazer isso no momento e precisamos de uma solução, por mais que imediatista.
    
    \tem Hans: Acha complicado seguir em frente com a proposta do ENEM. Houve uma companha feita pelo UNE e outras entidades, como o DCE. O "Adia ENEM", entendendo que o momento singular garante uma exclusão maior do que o convencional. 
    
    \item Paloma: Concorda de levar para o CEB a reabertura do ENEM. Acredita que devemos pensar na democratização da universidade, é um tema interessante. Acredita que o calico já tenha feito discussões sobre isso. Mas como o Hans disse,o o tempo é curto e precisamos de uma proposta pra agora.
    
    \item Cauê: Concorda com o Hans que tem que ter um GT no CEB, incluindo CAs e cursinhos interessados que deve elaborar uma contraproposta. A elaboração da minuta completa e detalhada é dever da COPERVE, não nosso. Acredito que devemos procurar a reabertura do ENEM. Não acredita que vá da certo, mas acredita que é o que devemos fazer. Não entende o problema colocado pelo Hans da problemática adicional do ENEM (não entende o problema colocado de o período atual ser mais excludente, pelo que eu entendi). Além disso devemos buscar critérios alternativos, citados pelo Hans. Não devemos descartar uma solução alternativa, se acharmos que não é vantajoso usar notas antigas. Talvez fazer um vestibular reduzido, a depender da COPERVE. Não podemos aceitar um vestibular 100\% presencial, como alguns setores da classe média estão reivindicando. Devemos, em CUn, votar contra a proposta da COPERVE. O ingresso vai ser após o final de 2020.2 então a COPERVE tem tempo para pensar em soluções alternativas.
    
    \item Hans: A dificuldade de comunicação com o MEC é um problema. Calico é o único centro onde apareceu a proposta de abertura do ENEM, Acha que é uma proposta que vai passar e não vai resultar em nada.
    
    \item \textbf{Encaminhamentos:} As propostas de encaminhamento apresentadas foram as que seguem:\\
    1 - criar GT no CEB com CAs e cursinhos populares, para elaborar ofício com sugestões à COPERVE

    2 - votar no CUn contra o ofício da COPERVE (de usar ENEMs 2009~2021)

    3 - apresentar ofício com sugestões
        a - buscar MEC para reabrir inscrições pro enem, mesmo que apenas modalidade digital
        b - como alternativa, em paralelo, elaborar critérios de seleção análogos à UDESC (ENEMs, vestibulares, notas de médio/técnico/ENCCEJA/EJA)
        c - como último caso, vestibular presencial reduzido para uma fração das vagas (Último critério de seleção, ou seja, somente para quem não se enquadra em nenhum dos outros critérios as já propostas pela COPERVE mais as propostas pelos estudantes (3.b)). \\ \\
    As propostas 3.c e 3.a foram votadas. O resultado final dos votos foi 7 à favor e 6 contra  a proposta 3.c, que foi aprovada. A proposta 3.a teve 10 votos à favor e um contra, logo também foi aprovada. As demais propostas foram aprovadas por consenso. As propostas aprovadas serão apresentadas e defendidas pelos representantes do Calico em CEB.
        
\end{itemize}

\pauta{Pauta CEB: Calenda}
\begin{itemize}
    \item Cauê: Discussão na CGRAD das propostas de comprimir o próximo período de 18 pra 16. Estudantes defenderam a proposta de 18 entendendo que esse período à distância é ainda mais desgastante. A justificativa para ter sido aprovada por 16 foi para normalizar o quanto antes o calendário. A proposta não levou em conta as condições concretas dos estudante E dos professores. Não podemos ter ilusão de que aumentar em 2 semanas vai resolver todos os problemas e não deve ser nosso foco. Devemos defender a extensão mas temos coisas mais importantes para defender.
    
    \item Hans: Acredita que reduzir de 18 para 16 não é o maior dos problemas, por mais que revele o caráter mecanicista da universidade. Acredita que a pauta tem o objetivo de colocar 3 semestres em um ano só e isso vai prejudicar imensamente os estudantes e que devemos levar para o CEB a defesa de não termos 3 semestres em um único ano.
    
    \item Cauê: Não acha ser fora do ponto isso do Hans por ser o ponto da reunião, o calendário. Acredita que a gente precisa que a universidade normalize o calendário de alguma forma e que temos condições de fazer isso, ninguém pretende fazer isso tirando férias, excluindo recessos e afins. A proposta é que o início de um semestre se dê no final do ano, que haja um recesso para o natal e o reinício no ano seguinte. Não normalizar o calendário resultará em um semestre a menos de ingresso, impacta nos recursos, na visão da população. É fundamental preservar a saúde mental, qualidade de ensino e etc, mas precisamos tentar garantir que as coisas se normalizem.
    
    \item Hans: Cauê coloca que a universidade precisa ta formando gente. Acha triste ver isso Precisamos levar a proposta de não termos 3 semestres remotos. Acha que ninguém está realmente aproveitando esse semestre. Se for para as aulas serem respostas, que isso seja presencial.
    
    \item Samuel: Concorda que a faculdade que não (serve apenas) para formar gente, mas querendo ou não é um dos focos dela. É complicado perder um semestre. Não sabemos nem como faremos os próximos vestibulares. Não sabemos nem como os 2020.2 entrarão no próximo. Sobre aprender, acredita que tenha aprendido bastante, depende bastante da matéria. Não acredita que seja o melhor método, mas acredita que não seja o pior. Não acha que tenhamos desperdiçado um semestre. Reitera o ponto do foco da universidade não ser formar.
    
    \item Tiz: Sobre o  tema dos 3 semestres: por mais que ache que o período de recesso seja grande, acredita que colocar um semestre a mais vai ser prejudicial. Acredita que um recesso "legal" é fundamental para fixação dos conteúdos. Acredita que as 18 semanas já é pouco tempo e tem diversos problemas didáticos, diminuir isso prejudica ainda mais. Falou das pessoas que não tem dinheiro e precisam se formar no período mais curto possível. Talvez outras solução possível devessem ser pensadas, como o por exemplo o que o Cauê comentou (início de terceiro semestre no final do ano, recesso, continuação no início do ano seguinte).
    
    \item Hans: Quando pensamos em política, pensamos para todos os cursos e não só para CCO. Compartilha email recebido do curso de Ciências Sociais. Muitos estudantes de primeira fase não estão conseguindo acompanhar os estudos. A gente consegue recuperar de outras formas. Acredita que 3 semestres em 1 ano vai [destruir] nossa saúde mental. Dá exemplo de outras formas de recuperar, como aulas no sábado.
    
    \item Samuel: Como seria efetivamente recuperado esse semestre, então? Ter o sábado utilizado, como exemplificado pelo Hans é uma péssima ideia. Estamos num momento excepcional. O problema não é o número de semestres, mas sim a forma como os professores estão lidando com isso. Dá exemplo absurdo de calculo (acredito que calc 1) que o professor extrapola o tempo de aula e muito.
    
    \item Hans: Samuel coloca a questão levantada pelo Samuel com relação ao sábado. Acredita que todo mundo está passando tempo demais na frente do computador. Reitera a proposta
    
    \item Samuel: mesmo com reposição os professores passariam tarefa extraclasse. Reitera que o problema é a forma como os professores tão lidando com isso. Adicionar o sábado é menos um (perdi o restante da fala. Continua sendo contra a proposta de reposição nos sábados)
    
    \item Cauê: Não entende a proposta do Hans e acha que o Hans não entende o que está sendo discutido. Ano que vem terá mais de 2 semestres. Um deles cortado, ao final do ano. Todas as propostas colocadas até agora são de 2 semestres e uma fração. A proposta de dois períodos é a de nunca regularizar. Tal proposta implica em não regularizar a quantidade de estudantes, ficar em desacordo com o orçamento. Não conseguiu compreender a justificativa dada pelo Hans. Reitera que o desgaste vai acontecer até a retomada das atividades presencias. Pede esclarecimento.
    
    \item Hans: Reposição é literalmente a gente estar repondo o semestre que perdeu. A única coisa que quer deixar no final é: Nós 'conquistamos' algumas garantia para o ensino remoto e que essas garantias estão sendo rediscutidas. (Se houve algo além disso, foi perdido)
    
    \item \textbf{Encaminhamentos:} As propostas de encaminhamento foram as que seguem: \\
    1 - Exigir que o calendário dos próximos semestres contenha 18 semanas e não 16
    2 - Exigir que, enquanto estamos passando pelo período de ERE (Ensino Remoto Emergencial), não aconteça mais de 2 semestres por ano. \\\\
    
    A proposta 1 foi aprovada por consenso. Já a proposta 2 foi votada, apresentando 4 votos à favor e 7 contrários, logo não foi aprovada. A proposta aprovada seŕa apresentada e defendida pelo Calico em CEB.
\end{itemize}

\presentes{Paloma Cione, Matheus Dhanyel Cândido Roque, Arthur Mesquita Pickcius (Coordenador da reunião), Naiara Sachetti, Teo Haeser Gallarza, Cauê Baasch de Souza, Luis Oswaldo dos Santos Ganoza (Relator da reunião), Helena Aires, João Paulo Taylor Ienczak Zanette, André William Régis, Enzo Coelho Albornoz, Mikael Luan da Silva Saraiva, Cristian Alexandre Alchini, Samuel Cardoso}\\

* Os conteúdos em parêntesis são comentários do relator.

\end{document}
