\documentclass{ata-calico}
\usepackage{indentfirst}
\pagenumbering{arabic}
\begin{document}

\maketitle

\pauta{Informes e Repasses}
\begin{itemize}
\item Foi tirada no CUn uma comissão para receber sugestões de alteração da minuta de volta às aulas.
\item DCE ta organizando uma campanha contra o avanço do ensino à distância e pela inclusão de todos. Foi escrita uma carta de reivindicações, CALICO já assinou.
\end{itemize}

\pauta{Estatuto do CETEC}
\begin{itemize}
\item Estatuto do CETEC foi arrumado desde a última reunião, alterando o que foi pedido. Ainda vão ter alterações de formatação. Começou a ser feita uma distinção das assossiações que fazem parte do CETEC.
\item Sugestão de alteração: O CETEC não pode ter como finalidade ser um conselho. Seria melhor se fosse colocado representar e defender os estudantes.
\item Foi colocado em voto no CETEC que fosse colocado representar e defender os estudantes ao invés de as entidades, mas não passou. O ponto que colocaram é de as entidades representarem os estudantes. O que pode ser feito é pontuar que o CALICO se posiciona a favor da representação dos estudantes.
\item É preciso tirar a parte no Art.20º inciso 2º, foi consenso no CETEC que seria tirado.
\item Talvez seja necessário fazer alguns destaques no documento para que se consiga mediar para algo mais decente. 
\end{itemize}

\textbf{Encaminhamento:} Comissão para rever novamente o estatuto, sexta-feira 17h.

\pauta{Eventos do CALICO}
\begin{itemize}
\item O CALICO conseguiu manter por um período rodas de conversa, foram 3 seguidas. Poderíamos ter seguido, porém não foi continuado. Agora o CALICO ta parado, é preciso pensar em algum evento para engajar o curso de novo.
\item Uma roda de conversa parece uma boa ideia, algo mais leve. Também uma boa ideia é algo mais de confraternização, algo tipo uma noite de jogos.
\item Uma ideia interessante é levar a ideia para o CETEC de algum evento sobre desindustrialização nacional. Temos 9 cursos de engenharia no CTC, essa questão toca diretamente na vida dos recém-formados. Talvez ter algum convidado para o evento. Se eles não toparem, talvez tocar isso pelo CALICO.
\item Ideias de tópicos: Para que a UFSC tem contribuído na pandemia; os males da tecnologia na pandemia; os deveres de um profissional da tecnologia.
\end{itemize}

\textbf{Encaminhamento:} Noite de Jogos. Responsáveis: Helena, Paloma.

\textbf{Encaminhamento:} Elaborar e levar a ideia do evento para o CETEC. Responsáveis: Luis, Cauê.

\textbf{Encaminhamento:} Evento sobre o que o INE está fazendo na pandemia. Responsável: Luis.

\presentes {Arthur Pickcius, Cauê Baasch, Helena Aires, Hans Buss, Luis Oswaldo, Mikael Saraiva, Paloma Cione}

\end{document}
