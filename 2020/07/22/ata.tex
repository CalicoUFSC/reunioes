\documentclass{ata-calico}
\usepackage{indentfirst}
\pagenumbering{arabic}
\begin{document}

\maketitle

\pauta{Informes e Repasses}
\begin{itemize}
\item Está acontecendo uma reunião da Frente pela Educação de Qualidade (FEQ).
\end{itemize}

\pauta{Pauta do CEB: Balanço da Discussão sobre o Ensino Remoto na UFSC}
\begin{itemize}
\item Cauê: Teve um comitê, que no fim gerou um relatório que tinha uma proposta de minuta de resolução que foi encaminhada para o CUn. No CUn, foi tirada uma comissão com 6 representantes que recebeu propostas da comunidade (dando quase 200 emails). Teve nessa sexta uma reunião sobre essa minuta, com voto secreto, e várias medidas importantes para os estudantes não passaram.
\item Hans: Na sexta-feira basicamente não se teve vitórias, o ensino remoto foi aprovado. Só na segunda e terça-feira foram conquistadas algumas vitórias. Os estudantes ficarão dependendo da boa-fé dos professores. Precisamos pensar em como garantir alguns pontos e como vamos nos posicionar nos colegiados e no CEB.
\item Paloma: Não estamos conseguindo trazer as pessoas pras reuniões nem conseguir direito tocar direito os eventos planejados. Precisamos conseguir ouvir a voz dos estudantes.
\item Cauê: Calico teve grande participação nos debates, assim como ajudou o DCE. A retomada apesar de ser do jeito que foi, não foi feita de forma imediata. Em outros lugares, os estudantes estão desamparados, sem garantias. O DCE teve derrotas muito ruins, como a não obrigatoriedade de gravas as aulas, a contabilização da frequência, coisas que o DCE continuará disputando, entrando com o pedido de reconsideração. Ainda temos condições de somar mais garantias para que a retomada seja feita da forma menos dolorosa possível.
\item Hans: O ensino remoto acelera a precarização do ensino, e diz que devemos estar pensando que somos um CA que constitui o movimento de uma universidade popular, dizendo que vários colegas trancariam o curso caso o ensino remoto fosse aprovado. Diz que poderíamos nos apoiar na proposta de aulas complementares, ela deixa pessoas pra trás, mas não causa o impacto do ensino remoto.
\item Cauê: São grandes disputas que tão tendo no movimento estudantil agora. Temos que disputar pra conseguir mitigar ao máximo as dificuldades que se encontram, mas impedir que se dê continuidade nas atividades não resolveria o problema de inclusão que existe. Cada semestre perdido é prejudicial e pode trazer problemas para os alunos. Tem que ter a crítica à retomada remota, mas temos que entender que nesse momento de pandemia a alternativa é não ter formação nenhuma. Precisamos fazer essa concessão agora, sabendo que é uma disputa que retomaremos depois.
\item Paloma: Sobre a precarização, sobre deixar estudantes pra trás, tudo isso é horrível, nosso cursos não são planejados pra ser feito com ensino remoto, toda a pedagogia da coisa. O ponto é saber de tudo isso e fazer com que tudo seja feito da melhor forma possível nesse período de pandemia e se manter sempre CONTRA o EAD pra que isso não seja implementado depois (o EAD na UFSC, não em geral).
\item André: Mesmo que disponibilizem o espaço físico da UFSC, não se tem estrutura suficiente pra isso na universidade. Outro ponto é a BU, que era muito importante para os alunos, é importante que tenha um bom acervo digital.
\item Cauê: Desde o começo das discussões a reitoria tentou colocar a abertura da infraestrutura para quem precise. É uma discussão bem delicada, o DCE se colocou contra isso, seria muito complicado colocar os estudantes mais vulneráveis para se deslocar até o campus. O que está se preparando é um auxílio de 70 reais para internet e a disponibilização de computadores para os estudantes que precisarem, dos computadores dos laboratórios.
\item Cristiano: Eu não vejo como uma coisa ruim o EAD em si, mas sim a precarização que ele pode trazer. Precisamos nos preocupar em implementar isso de forma que não traga prejuízos futuros e precarização. Precisamos saber que não é a solução, mas é algo temporário. As conquistas que a gente já teve são incrivelmente proveitosas para nós.
\item Hans: Em relação à termos, o EAD é realmente uma aula planejada para ser feita à distância, diferente do ensino remoto (er). O ensino remoto agora já é dado. Aponta de novo a proposta de levar as atividades complementares como alternativa.
\item Paloma: Nosso curso não possui atividades complementares, teríamos que adicionar horas no nosso currículo.
\item Hans: Seria uma questão de articular com nossos colegiados, propondo optativas extracurriculares.
\item Cauê: Queríamos proibir a contabilização da frequência, que as atividades não pudessem ser síncronas e que as aulas fossem gravadas. Conseguimos o trancamento de matrículas sem prejuízos, etc. Devemos continuar tentando com os conselheiros de reverter os argumentos toscos de "direito de imagem", alegando ilegalidades. Não podemos botar todos os órgãos numa cesta só e devemos continuar nossa luta também no nosso colegiado.
\item Hans: Temos poucas chances de reverter as decisões do CUn, devemos levar ao CEB um CUn bem realizado, somos poucos estudantes no meio do CUn para q possamos levar as nossas propostas. Não como alternativa ao ensino remoto, pois este já esta dado, mas sim como uma forma de amenizar.
\item Luis: O que foi aprovado no CUn muito provavelmente não será revertido, e devemos lutar agora nos nossos colegiados. Não podemos achar que todos os colegiados vão aprovar as mesmas medidas nem que as bases tem as mesmas demandas. Quem tem representação nos órgãos colegiados devem entrar diretamente em contato com suas bases, entendendo suas necessidades. O problema é que o INE dá aulas para muitos cursos (uma porrada) dentro da UFSC. De alguma forma, deverá ser feito um mapeamento dentro dos cursos para saber quais disputas devem ser feitas dentro dos colegiados. Na computação, sabemos q não conseguimos fazer as mesmas coisas que um colegiado do CFH. Devemos mapear os cursos e saber o que eles precisam e suas necessidades para levarmos ao colegiado.
\end{itemize}

\pauta{Pauta do CEB: Próximos Passos para as Mobilizações}
\begin{itemize}
\item Hans: Acha possível que amanhã alguns cursos como a biologia, a filosofia, aparecerem com um boicote ou greve contra o ensino remoto. Não sabemos como vai ser e está tudo bem nebuloso.
\item Cauê: Sem aquelas garantias, parece inviável começar como está dado agora. Não devemos nos opor a essa proposta, mas devemos travar até o fim a batalha contra o CUn; se formos usar esse instrumento de luta, devemos ter cuidado em construir a pauta e qual é a sua reivindicação. Não podemos ter dúvida sobre o que é a greve, e sobre quando termina-la. Alguns setores se afobam um pouco em propor essas ações sem pensar em suas consequências.
\item Caz: nenhuma greve não presencial deu certo, não devemos apoiar isso agora.
\item Cauê: Se alguém propor um boicote ou greve agora nesse CEB deveríamos ser contra.
\item Hans: Não haveria problema em se abster agora, já que é bem dificil de passar essa ideia.
\item Defesa do ERE como prioridade somente nas fases finais. Não foi consenso, enviado para votação. 4 votos a 1 de não incluir esse encaminhamento.
\end{itemize}

\textbf{Encaminhamentos:}
\begin{itemize}
\item Ir até o final na disputa pelas garantias no CUn
\item Defender a realização de um CUn paritário
\item Já discutir nos colegiados as garantias na retomada
\item Defender que as ACs sejam consideradas e discutidas nos colegiados como uma alternativa possível ao ensino remoto
\item Não se opôr à ideia de um boicote/greve como último recurso
\end{itemize}

\presentes {Cauê Baasch, Helena Aires, Hans Buss, Paloma Cione, Darlan Ferreira, André Régis, Cristiano Azevedo, Luis Oswaldo}

\end{document}
