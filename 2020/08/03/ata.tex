\documentclass{ata-calico}
\usepackage{indentfirst}
\pagenumbering{arabic}
\begin{document}

\maketitle

\pauta{Informes e Repasses}
\begin{itemize}
\item Sobre o auxílio emergencial da PRAE, o desse mês tava demorando a cair; foi falado com a PRAE, teve alguns problemas esse mês, mas a princípio é pra cair hoje. O auxílio vai mudar as inscrições, ser uma inscrição só até o final do ano e vai ficar aberto até lá.
\item Na semana que tava sendo discutida a normativa no CUn, o DCE fez uma publicação; alguns alunos do CETEC se sentiram na responsabilidade de mostrar apoio aos conselheiros e decidiram mandar um email. Esse email acabou não sendo mandado, principalmente por alguns pedidos de alteração do CALICO. Agora foi revisto esse email que vai ser lido depois uma nota. Além disso, foi escrita uma nota pelo CETEC se posicionando contra esses mesmos conselheiros.
\end{itemize}

\pauta{Discussão sobre as aulas "experimentais" no CTC}
\begin{itemize}
\item Luis: Teve uma resolução baixada no CTC falando que a partir do dia 30/07 os professores teriam direito de realizar aulas "experimentais" com os alunos. Um problema nessa resolução é que não tem nenhuma garantia para os alunos, nada impediria os professores de contar esse conteúdo como dado.
\item Luis: Alguns alunos do CETEC acharam que seria interessante marcar uma reunião com o diretor do CTC para conversar sobre essa resolução. Foram expostos alguns dos problemas com essa resolução, que alguns dos professores poderiam abusar disso e os alunos ficariam com medo de expor isso. A resposta do professor foi de dizer que não tem problema, pois se mantém o sigilo dos alunos. Foi entendido que o professor acreditava que qualquer problema que surgisse, os centros acadêmicos iriam saber e falar para o diretor, que falaria com o professor para saber o que estava acontecendo.
\item Paloma: Foi esperado desse conselho isso, que parece que eles veem a relação professor-aluno de forma diferente do que realmente é.
\item Cauê: O grande problema disso é que é 'opcional', parece que acham que resolve todos os problemas. Na prática, sabemos que a realidade é mais complicada. Conhecendo a má-vontade de alguns dos professores, sabemos que é bem capaz de que muitas das aulas sejam contadas como dadas nesse período opcional. É complicado, pois os professores não acham que estão errados, é muito mais uma questão de disputa política do que de tentar convencer eles.
\end{itemize}

\textbf{Encaminhamento:} Redação de uma nota, pontuando os problemas da resolução publicada pelo Diretor do CTC e reafirmar a disponibilidade do CALICO para ajudar os alunos caso passem por algum problema relacionado à isso. Responsável: Paloma

\pauta{Discussão sobre a retomada das atividades}
\begin{itemize}
\item Cauê: O calendário ainda não foi definido, vai passar pela câmara de graduação nessa quarta-feira (05/08); se for aprovado, dia 31 seria o início das aulas mesmo, planos de ensino e matrículas aconteceriam antes disso. Muito em breve deve ter uma reunião do colegiado de curso, existem coisas a se estar pautando. As derrotas que se teve no CUn, também questões como plataformas.
\item Luis: Precisamos pensar em como será a metodologia pra definir isso tudo. Ficar fazendo reuniões longas talvez não seja o melhor jeito, talvez separar um grupo ou todo mundo conversando no grupo do CALICO mesmo e depois chamar uma reunião pra sumarizar o que foi discutido. É importante que mostremos que estamos pensando nisso, justamente para não chegar no fim do mês e ter algum problema. Também temos que pensar em alternativas, pois existem propostas que tem chances de não conseguir passar.
\item Cauê: Sobre a questão das presenças, é essencial que as pessoas não possam reprovar por FI. Ta sendo escrito uma portaria pra ser encaminhada para o CUn tentando proibir isso, provavelmente vai passar. O DCE também está elaborando uma cartilha sobre como prosseguir nisso, é importante ficar de olho. Algo muito importante pra se discutir no colegiado é como vai ficar a questão das atividades avaliativas, é preciso garantir que todos consigam participar e, se não conseguir, que não seja prejudicado.
\item Luis: Várias dessas questões são questões do colegiado de departamento, é importante conseguirmos reuniões dos dois.
\item Cauê: É meio que nossa responsabilidade também tentar conseguir garantias não só para os cursos do INE, mas para outros cursos que tenham matérias com o INE.
\item Luis: É importante que os representantes dos colegiados consigam entender bem a realidade do curso, ainda mais considerando que não conhecemos muito a realidade de outros cursos que também teremos que defender. É preciso ter o máximo de conhecimento que a gente consiga ter.
\item Paloma: É importante lembrar que temos um aluno surdo no curso, precisamos lembrar de levar isso pros colegiados para que consigamos garantir a inclusão. Também é importante levar para garantir a inclusão de estudantes em outros cursos que não sejam o nosso.
\item Luis: Depois precisamos articular outras reuniões do CALICO para dar o repasse aos alunos e ter esse retorno, continuar pensando em conjunto sobre o tema.
\end{itemize}

\textbf{Encaminhamento:} Articular reuniões dos colegiados de departamento e curso para propor os pontos que não foram aprovados no CUn, sendo eles: a gravação de aulas, não aferição de frequência, limite da carga horária das atividades síncronas e a não realização de atividades avaliativas síncronas.

\pauta{Feedback da noite de jogos}
\begin{itemize}
\item Luis: Teve a noite de jogos, com o campeonato. Tanto o primeiro quanto o segundo lugar não quiseram o prêmio e não temos terceiro colocado.
\item Paloma: A divulgação foi um pouco de um problema, seria melhor mais animação da gestão nisso.
\item Cauê: Talvez quinzenal seja uma boa. Seria bom fazer o campeonato separado da noite de jogos, para não esvaziar os jogos.
\item Teo: Em questão do campeonato, foi bom, mas durou muito tempo
\end{itemize}

\presentes{Cauê Baasch, Hans Buss, Helena Aires, Luis Oswaldo, Arthur Pickcius, João Trombeta, Enzo Albornoz, Teo Gallarza, Lucas Wodtke, Matheus Roque, João T.I.Zanette, Bruno Huebes}

\end{document}
