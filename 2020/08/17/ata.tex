\documentclass{ata-calico}
\usepackage{indentfirst}
\pagenumbering{arabic}
\begin{document}

\maketitle

\pauta{Informes e Repasses}
\begin{itemize}
\item Calico foi informado que o CETEC foi assaltado novamente. Existe a possibilidade o Calico ter sido um dos CAs arrombados. Polícia já identificou e predeu um dos envolvidos. Os itens roubados já foram recuperados. Precisa fazer perícia no local, Mikael ta responsável por acompanhar.
\item Auxílio internet da PRAE. Saiu o edital pra inscrição pro auxílio com diversas restrições, incluindo cadastro na PRAE e contratuante provedor de internet.
\item Aberto edital pro auxílio emergencial de 200 reais, agora até o final do período de pandemia, sem necessidade de reinscrição.
\item Cessão do CUn amanhã. Será debatido pedido de reavaliação feito pelo DCE e a aprovação do novo calendário.
\item Adiamento da reunião do colegiado de curso. Calico mandou um email apontando as problemáticas da reunião ter apenas uma hora e a pauta ser a aprovação dos planos de ensino.
\end{itemize}

\pauta{Avaliação dos Planos de Ensino}
\begin{itemize}
\item Cauê: Temos reunião do colegiado de curso na próxima terça feira, com a pauta apenas com a aprovação do plano de ensino. O Colegiado é a última instância que aprova os planos de ensino. Precisamos nos atentar àqueles planos com avaliações síncronas, prestar atenção na forma de se fazer a frequência, chances de recuperação, etc. Não temos reunião de colegiado desde dezembro, o que foi um desleixo da coordenação, que não se reuniu nenhuma vez para discutir as questões que estavam aparecendo, principalmente por ser um curso que poderia ajudar com questões tecnológicas.
\item Luis: Não vamos conseguir fazer com que os professores não façam 3 provas síncronas, e começar a fazer discussões dos pontos mais problemáticos. Há muitos planos que não são delimitadas as horas de atividade síncrona, que é bem importante. É importante ressaltar que no final do ano passado enviamos os nomes pro colegiado de departamento, mas uma pessoa precisa ser atualizada.
\item Paloma: Uma ideia é ter uma garantia que, pelo menos para as pessoas que tenham certeza que não conseguiriam fazer provas síncronas, ter alguma exceção para que consigam participar. É interessante também colocar no plano de ensino se as aulas serão ou não gravadas, ou se pode ou não ser gravadas.
\item Luis: Acha melhor que, se a turma toda puder fazer avaliações síncronas, tudo bem, mas se algumas pessoas não conseguem, todas as provas devem ser iguais, e não provas diferentes/avaliações diferentes para as pessoas que não podem fazer as síncronas.
\item Cauê: Aponta que é necessário apontar pros professores que eles estão excluindo pessoas ao fazer provas síncronas quando há pessoas que não conseguem. Sempre deixar o tom claro e quando algo é inaceitável. A disputa tem que ser feita no colegiado, teremos espaço e votação se necessário. Temos que tomar cuidado para que não seja igual a sessão passada, em que discutimos bastante mas mesmo assim os planos foram aprovados de forma implícita. Podemos contatar novamente as turmas, jogar no moodle de novo para que tenhamos munição e conseguirmos ter argumentos fora dos planos de ensino. Amanhã no CUn teremos mais respostas se isso vai ser necessário.
\item Luis: Precisamos apontar quais as maiores problemáticas. Mandar  um email para que seja aumentado o tempo da reunião, e que não seja adiada por tanto tempo. Uma das justificativas que eles deram foi que não tínhamos os planos da MTM, mas podemos aprovar em blocos os planos de ensino que tem menos problemas, e uma segunda reunião que terão discussões mais longas, junto com os planos da MTM.
\item Cauê: Fletes aprovou por email que a matéria de ES1 com a Vilain não tivesse prova de recuperação, pois terá mais horas práticas. Apesar das chances de recuperação colocadas no CUn, seria bom colocar mais uma coisa possível para o aluno, já que está de acordo com a resolução. Também devemos ir além da legalidade.
\item Paloma: O professor Santiago, da disciplina Grafos, pretende fazer tudo assíncrono; assim como a professora Jerusa, de Formais. Em questão de plataforma, existem 3 que parecem mais comuns entre os professores.
\item João Trombeta: Cancian vai fazer presença, e dará aula pelo Microsoft Teams, 10 exercícios síncronos que valem 40\% da nota e 60\% de um trabalho assíncrono.
\item Paloma: Temos que olhar melhor os planos de ensino dos calouros, pra conseguir disputar o melhor possível.
\item Mikael: Podemos falar com o CA da MTM.
\end{itemize}

\textbf{Encaminhamento:} Redigir um email pedindo mais tempo de reunião e a separação dos planos mais leves/com mais discussão.

\presentes{Cauê Baasch, Helena Aires, Luis Oswaldo, Paloma Cione, Hans Buss, Mikael Saraiva, Arthur Pickcius, João T.I.Zanette, Cristian Alchini, Vitor Egami, João Trombeta, Nicole Schmidt, Bruno Huebes}

\end{document}
