\documentclass{ata-calico}
\usepackage{indentfirst}
\pagenumbering{arabic}
\begin{document}

\maketitle

\pauta{Informes e Repasses}
\begin{itemize}
\item Vai ter reunião da Amnésia quarta das 18h-19h na sede.
\item Ta rolando um monte de atividades do 8M, domingo vai ter panfletagem de manhã na Hercílio Luz, segunda vai ter uma marcha com concentração às 16h30 no DCE e 17h30 no TICEN para sair em ato às 18h.
\item Quarta 11/03 vai ter o festival do DCE na praça da cidadania ao meio-dia com várias atividades culturais e barraquinhas.
\item Editais do CDS para atividades.
\item Teve reunião do departamento: Alex, Cristina e Maicon foram efetivados como A2.
\item Teve reunião da comissão unificada, que reune apufsc, andes, sintufsc, apg, DCE e outras representações discentes que quiserem participar. Falaram do calendário e começaram a discutir as mobilizações do dia 18.
\item Tem assembléia dos estudantes dia 17 às 12h.
\item Ta tendo surto de sarampo e febre amarela em Florianópolis, lembrar de se vacinar.
\end{itemize}

\pauta{Grafite (sede e vão)}
\begin{itemize}
\item Para fazer isso no vão, seria bom falar com o CA da sanitária. Existem outros prédios já com grafites, então tem precedência para a ideia. Precisaria falar com a prefeitura universitária, com o centro e com o departamento.
\item Para a sede só precisa ver se todo mundo aceita, é mais fácil de operar.
\item Se o CALESA topar, precisamos ver viabilidade. É melhor começar a ver isso por baixo, começar primeiro vendo com o departamento.
\item No CALICO seria bom fazer na divisória de MDF, mas precisa primeiro descobrir se tem como fazer ali, se tem tinta que pegue na parede.
\item Seria bom dar mais atenção no vão do INE, que é mais público.
\item Na parede do CALICO tem também a opção de adesivar caso não role o grafite.
\item Foi discutido a um tempo atrás a possibilidade de adesivar a porta.
\item Grafite seria melhor pela questão de ser arte popular, pela significância que tem.
\item Talvez já tirar responsável para correr atrás da burocracia de grafitar, mesmo que só comece depois do aval do CALESA.
\item Talvez dê para grafitar a porta e adesivar dentro da sede.
\end{itemize}

\textbf{Encaminhamento:} Falar com o CALESA sobre o grafite no vão. Responsável: Paloma.

\textbf{Encaminhamento:} Falar com departamento, centro e prefeitura caso CALESA aceite. Responsável: Luis.

\textbf{Encaminhamento:} Ver sobre o grafite na sede. Responsável: Cauê.

\textbf{Encaminhamento:} Ver sobre a adesivagem da sede. Responsável: Fábio.

\pauta{Mobilizações de Março}
\begin{itemize}
\item 17 de março tem assembléia estudantil convocada para discutir a participação no ato do dia 18. Já tem planejado pelo DCE panfletagem, seguido de arrastão terminando no RU e concentração na frente da reitoria no começo da tarde para ir para o centro.
\item Parece desnecessário chamar uma assembléia do curso agora, mas precisamos pensar em como vamos estar trazendo essa pauta para o curso e como vamos divulgar. Passagens em sala e diálogo com outros cursos, talvez.
\item Fazer publicações e word art com divulgação do tema voltado para o CTC/CCO talvez engaje melhor os estudantes.
\item É importante manter a participação de CCO que teve ano passado.
\item Mesmo com divulgação online, é importante ainda a passagem em sala pelo contato mais pessoal com os estudantes, também considerando os professores e a discussão que surge nas salas.
\item Talvez misturar a satirização com a seriedade, fazendo algo que chame a atenção seguida de um texto mais sério e explicativo.
\item Podemos assumir pelos princípios do CALICO e dessa gestão que é um dever do CALICO estar compondo esse ato em defesa da educação.
\item Apenas informar da assembléia e do ato não é o suficiente, precisamos também informar as pessoas com passagens em sala e com posts nas mídias sociais para conscientizar da necessidade do ato.
\item Talvez caiba fazer um evento (mesa redonda, aula pública, roda de conversa) para falar sobre o tema, principalmente sobre temas mais sensívels como as privatiações de serviços publicos, seria um pouco desonesto levar as pessoas para a rua somente por algumas das pautas, tem que ter bem definido o que estamos defendendo e combatendo.
\item Talvez esse evento possa ser feito com outros centros acadêmicos que estejam interessados, é interessante ter essa mobilização de base já desde o começo.
\end{itemize}

\textbf{Encaminhamento:} Word art + texto explicativo sobre as mobilizações. Responsáveis: Paloma (word art), Cauê, Luis, Helena, Caz.

\textbf{Encaminhamento:} Passagens em sala de aula dia 16 manhã e tarde e 17 manhã. Responsáveis: Cauê e Helena.

\textbf{Encaminhamento:} Falar sobre as passagens em sala e criação de um evento informativo na reunião do CETEC para juntar com mais cursos. Responsáveis: Cauê, Luis, Mikael.

\textbf{Encaminhamento:} Tirar uma reunião para discutir a participação de CCO no 18M.

\pauta{Revitalização da Varanda do CETEC}
\begin{itemize}
\item A ideia é fazer assentos com pallets e almofadas para que as pessoas ocupem o espaço.
\item Seria bom algum jeito de não perder os sofás e colocar uma régua para o pessoal conseguir carregar celular e coisas do tipo.
\item Seria bom levar para o CETEC que em questão financeira não é o CALICO e sim o CETEC que banca a revitalização.
\item A galera que ta fumando aceita bem de boa o diálogo e é bem fácil isso.
\item Talvez criar outro espaço para o pessoal conseguir fumar e sair da varanda.
\item O piso está quebrado, também seria bom arrumar isso.
\item Na revitalização seria bom fingir pelo menos que tem uma área de fumantes longe das janelas dos CAs.
\item Em questão do lixos seria bom ter lixeiras fixas, para que não roubem. Ou colocar lixos bem baratos para não ter muito problema caso roubem.
\item Seria bom já mostrar para o CETEC que estávamos certos para que tenhamos mais credibilidade para a revitalização em si.
\item Propor uma faxina com todos os CAs seria bom.
\item O ar-condicionado da automação ta pingando no chão e seria bom ligar num cano.
\end{itemize}

\textbf{Encaminhamento:} Levar todas as ideias para o CETEC. Responsáveis: Cauê, Luis e Mikael.

\pauta{Representações Discentes}
\begin{itemize}
\item Vai ser renovado os representantes do conselho de unidade.
\item Perdemos um representante no conselho do departamento, não tem mais condições de frequentar.
\item Perdemos dois representantes no conselho do curso.
\end{itemize}

\textbf{Encaminhamento:} Cauê e Luis para o conselho de unidade, Mikael para o do departamento, Fábio e Pedro para o do curso.

\presentes {Cauê Baasch, Helena Aires, Heloisa dos Anjos, Luis Oswaldo, Arthur Pickcius, Fábio Coelho, Paloma Cione, Pedro Silva, Matheus Roque, Mikael Saraiva, Bernardo Farias, Cristiano Azevedo}

\end{document}
