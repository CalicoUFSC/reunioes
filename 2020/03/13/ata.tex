\documentclass{ata-calico}
\usepackage{indentfirst}
\pagenumbering{arabic}
\begin{document}

\maketitle

\pauta{Informes e Repasses}
\begin{itemize}
\item Sobre a questão do grafite, é possível fazer na divisória. Em questão de custo, daria 200 reais para o artista e uma estimativa de 100 de tinta. Precisaria só medir a parede.
\item Recebemos reclamações dos calouros quanto ao professor Carvalho, de cálculo 1, tanto da conduta em aula quanto do material disponibilizado. O material tem caráter de defesa da ditadura de 64.
\item Recebemos reclamações do ar-condicionado da sala CTC102, que não funciona.
\item Quanto à pandemia de coronavírus, a UFSC se posicionou com medidas recomendadas, o que inclui orientações contra atividades que envolvam aglomerações. Com isso o ato do dia 18 e a assembléia do dia 17 estão sendo avaliados. O trote integrado vai ser discutido pela CO com algum orgão para ver se continua ou não.
\item Foi levado no colegiado do curso um relatório sobre ingressos tardios, vagas não ocupadas. Não foi muito propositivo, teve mais um carater de reflexão.
\item Amanhã (14/03) vai ter ato pelos 2 anos da morte de Marielle.
\item Os trabalhadores do serviço público de Florianópolis estão em greve por tempo indeterminado a partir do dia 16/03, aprovado em assembleia do dia 11/03.
\item Teve reunião da câmara de administração do colegiado de departamento, foram discutidas 3 pautas. Foi criado um núcleo dentro de outro laboratório, com uma mudança no nome. Foi aprovado o plano orçamentário do depto para 2020, também foi discutido a criação de um fundo para bancar passagens para eventos, mas ficou a cargo de pesquisa de viabilidade da criação do fundo.
\item Foi discutido na reunião do CETEC os cargos da diretoria do CETEC, mas a conversa foi adiada. CALICO talvez pegue a secretaria do CETEC. As reuniões ordinárias vão ser todas as segundas às 18h.
\item No CEB continuou a discussão do regimento eleitoral. Como CALICO, foi votado a favor dos representantes nos colegiados/conselhos não precisarem estar com nome na chapa. Foi votado contra todas as representações serem da chapa vencedora e pela manutenção da representação ser proporcional ao número de votos de cada chapa. A proposta do CEB também poder centralizar os votos dos representantes do conselho teve voto do CALICO a favor da proposta.
\item O Chicco acontece hoje. 
\end{itemize}

\pauta{Nomeação da Comissão Organizadora da SECCOM}
\begin{itemize}
\item A CO da última edição vem preparando bastante material para guiar as próximas edições. Também sugeriram nomes para essa edição. Parte da CO está se formando agora e não conseguem mais participar.
\end{itemize}

\pauta{Programa de Formação}
\begin{itemize}
\item A ideia é tanto aumentar o nível de formação do curso quanto aproximar os estudantes do CALICO.
\item É importante seguir o que a UFSC está recomendando quanto a aglomeração de pessoas e adiar todas as atividades.
\item É importante sim se cuidar com a doença, mas em eventos menores de 20-30 pessoas não afeta tanto, principalmente considerando que aulas ainda continuam. A menos que a UFSC suspenda todas as atividades, não precisamos adiar os programas de formação.
\item Uma boa opção seria um evento sobre entidades/CAs, que provavelmente vem bem menos pessoas, ai não teria problemas de aglomeração.
\item Em questão de cronograma, seria bom ter CFE na semana do dia 23, CFS um pouco depois e roda de conversa por último.
\end{itemize}

\textbf{Encaminhamento:} Procurar um mediador para o Como Funciona a Sociedade (CFS). Responsável: Hans.

\textbf{Encaminhamento:} Evento sobre centros acadêmicos, formato de roda de conversa. Responsáveis: Cauê e Luis.

\textbf{Encaminhamento:} Entrar em contato com o caliss para o Como Funciona Entidades (CFE). Responsável: Hans.

\pauta{Reclamações sobre o professor Marcelo Carvalho}
\begin{itemize}
\item Podemos levar direto para a ouvidoria, fazer uma nota de repúdio e/ou ir no departamento de matemática falar com eles sobre isso.
\item O que podemos discutir sobre o caráter do conteúdo e sobre o fato de o conteúdo não ter a ver com a matéria ministrada. Já existiram campanhas e ações contra esse professor que não resolveram o problema, não temos condições de manter essa luta.
\item Se fizermos uma nota, ela precisa ser uma nota bem completa, explicando tudo. Seria algo demorado e cansativo de elaborar, mas valeria a pena. É interessante ver de ir para a ouvidoria, fazer a luta institucional; se conseguirmos toda uma turma de cálculo fazer reclamações na ouvidoria, teria bastante impacto. Em questão de ir no departamento, possivelmente não surja muito efeito.
\item Talvez falar com outros CAs que têm cursos com a matéria para ter mais apoio, colher assinaturas para a nota.
\end{itemize}

\textbf{Encaminhamento:} Nota de repúdio. Responsáveis: Paloma, Cauê, Luis, Helena.

\textbf{Encaminhamento:} Averiguar burocracia da UFSC sobre páginas pessoais. Responsável: Luis.

\textbf{Encaminhamento:} Reclamação individual na ouvidoria. Responsável: Helena.

\pauta{Noite de Jogos}
\begin{itemize}
\item Considerando a assebléia no dia seguinte e as atividades da paralização, precisa avaliar se a noite de jogos continua nessa data ou não.
\item Considerando a recomendação da UFSC, seria melhor não manter, ainda mais considerando a quantidade de pessoas e o comportamento delas na noite de jogos.
\item Podemos adiar o evento, focar mais em outras coisas por enquanto.
\item Precisamos ter cuidado pra não só ficar empurrando o evento com a barriga, temos que ficar reavaliando a conjuntura para ver o momento propício do evento.
\end{itemize}

\pauta{Avaliação das Atividades da Primeira Semana}
\begin{itemize}
\item O bar não teve problemas.
\item O linguicinha teve problemas de operacionalização, mas foi bastante gente. A varanda não foi o lugar ideal para as conversas, porque é muito estreito.
\item Sobre o bar, é preciso se ater ao local, porque o evento está crescendo e pode logo dar problema.
\item Para o linguicinha, teve também problemas para tocas as coisas.
\item Sobre o tour, foram apenas três pessoas da gestão e seria interessante ter mais presença do CALICO.
\item A reunião de sexta fracassou com o propósito dela, de chamar calouros para participar.
\item Sobre o local do bar, tem agora a oportunidade de fazer em outro lugar, que nos foi ofertado um local para eventos bem flexível de discutir.
\item Deveríamos falar com os veteranos já no fim do semestre anterior para deixar a primeira semana para o CALICO.
\item O linguicinha teve problemas organizativos sérios, em questão de planejamento e divisão de tarefas. Devíamos ter começado a preparar as coisas um pouco mais cedo. Erramos também em questão de quantidade, precisaríamos comprar mais coisas.
\item Sobre os veteranos diretos, a gente já tinha a agenda desde o fim do semestre passado, ai precisa ver onde foi o erro em ter eventos juntos.
\item A recepção na aula de introdução foi mal organizada e teve alguns problemas. Precisamos forçar que tenhamos cada vez mais espaço, ser o primeiro contato deles com a UFSC.
\item Sair do bar vai ser meio complicado, a ideia é ser um evento simples onde a gente só vá para o bar.
\end{itemize}

\presentes {Helena Aires, Matheus Roque, Luis Oswaldo, Cauê Baasch, Hans Buss, Bernardo Farias, Paloma Cione, Mikael Saraiva}

\end{document}
