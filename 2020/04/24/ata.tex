\documentclass{ata-calico}
\usepackage{indentfirst}
\pagenumbering{arabic}
\begin{document}

\maketitle

\pauta{Informes e Repasses}
\begin{itemize}
\item Foi aberto semana passada o segundo edital do auxílio da PRAE, tiveram 40 inscritos a menos.
\item Teve uma reunião do DCE, com um encaminhamento sendo a chamada de um CEB para dia 27/04 (segunda-feira) às 17h.
\item O CALICO está com uma nova identidade visual, lançada essa semana.
\end{itemize}

\pauta{Eleições do DCE}
\begin{itemize}
\item No final do semestre passado, foi prorrogada as eleições do DCE, principalmente por não se acreditar que tinha condições de fazer as eleições por conta de vários motivos. Porém, um pouco antes de se retomar as eleições esse semestre, começou o problema com o COVID-19. Agora estão tendo discussões para ver como que resolve o problema das eleições.
\item A discussão vai continuar do CEB. Uma possível proposta vai ser de encerrar a gestão e passar o controle ao CEB. Essa proposta é inviável e provavelmente não vai ter apoio. Outra possível proposta é tocar as eleições de forma online, o que vai prejudicar o debate político que historicamente acontece nas eleições da UFSC. Nesse cenário, é uma questão de prorrogar novamente, mesmo que a gestão já esteja desgastada e não gostaria disso em qualquer outra situação.
\item Outra questão levantada no CEB vai ser a renovação das representações discentes.
\item É importante não defendermos as eleições online, pois não chegaria a alcançar nem mil estudantes.
\item O maior problema não é só o alcance, mas a qualidade do debate. Querendo ou não, sem as eleições presenciais, as pessoas não entram em contato com todas as propostas apresentadas pelas chapas. Todo o contato orgânico é perdido, com as pessoas não vão ter acesso a todas as informações com qualidade suficiente pro que se espera de uma eleição da UFSC.
\item Se surgir a ideia do CEB centralizar a gestão, é preciso ter bastante cuidado. Talvez acabe se mostrando uma ideia menos democrática, dependendo dos elementos trazidos para a discussão. Isso vai acabar sendo algo que precisa ser analisado na hora do CEB. Seria bom que o CALICO consiga ter o maior número de representantes, nem que seja só para ouvir, para que se possa ter uma análise na hora.
\item Seria importante o CALICO se mostrar mais ativo no DCE, para dar nosso apoio à gestão nessa situação, caso acabe sendo prorrogada a gestão.
\end{itemize}

\textbf{Encaminhamento:} Posicionamento do CALICO no CEB contra as eleições online e a favor do prorrogação da gestão.

\pauta{Calendário Acadêmico e Ensino à Distância}
\begin{itemize}
\item Precisamos pensar em quando e como retomar as atividades da universidade.
\item Vazaram documentos da reitoria com propostas de como retomar as atividades. Existem nessas propostas pontos muito estranhos, como por exemplo o distanciamento mínimo de 1,5m entre as pessoas, algo que é muito difícil de controlar em salas de aula e no RU.
\item Essas propostas levam várias coisas em consideração. Se estamos falando de utilizar o período de férias e de qualificar o ensino a distância, estaríamos falando de uma reorganização maior da universidade, pois não se tem infraestrutura para se manter uma proposta dessas. Também estaríamos falando de uma redução do número de estudantes no campus. É preciso pensar também nos estudantes que moram longe. Precisaria de um bom tempo de estudo por parte da reitoria para pensar em soluções viáveis.
\item Não é impossível, mas mesmo com aulas online teriam vários pontos a ser pensados em questão de permanência.
\item É perigoso começar a falar em retomas as aulas quando ainda não chegamos no pico da pandemia em Florianópolis. É preciso pensar principalmente na segurança e na saúde de todos. Se formos discutir medidas como ensino remoto, é preciso pensar em vários pontos, como garantir que todos os alunos tenham acesso à internet, a qualificação dos professores, a adaptação do material, entre vários outros pontos.
\item No grupo de trabalho formado pela reitoria, precisamos pressionar a reitoria para ter representação discente. Precisamos estar participando desses processos, pois isso nos afeta diretamente.
\item Ensino a distância não é apenas disponibilizar o material existente online e fazer provas online, é preciso toda uma adaptação do material e do conteúdo. Isso no nosso curso é algo complicado, pois a qualidade de ensino já é baixa. Seria interessante tirar um grupo para analisar essa possibilidade para que o CALICO já tenha um posicionamento formado quando isso aparecer nos colegiados.
\item No nosso curso, é necessária a infraestrutura da UFSC para um ensino de qualidade. A realidade de muitos estudantes se mostra contrária à EaD, ainda mais quando nos comprometemos a um curso presencial e não estamos preparados a uma realidade dessas.
\item Uma possibilidade seria abrir parte da infraestrutura da UFSC para aqueles que realmente precisem utilizar, para uma variedade de questões como laboratórios e saídas de estudo. É extremamente preocupado pensarmos nisso tudo. É preciso nesse primeiro momento parar e estudar mesmo a EaD, termos argumentos sobre o assunto e manter em mente aqueles estudantes que talvez não tenham condições de um ensino nesses moldes.
\end{itemize}

\textbf{Encaminhamento:} Grupo de trabalho para estudar EaD. Responsáveis: Paloma, Cauê, Luis.

\presentes {Cauê Baasch, Helena Aires, Luis Oswaldo, Arthur Pickcius, Teo Gallarza, Paloma Cione, Hans Buss, João Trombeta}

\end{document}
