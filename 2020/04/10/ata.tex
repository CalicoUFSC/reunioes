\documentclass{ata-calico}
\usepackage{indentfirst}
\pagenumbering{arabic}
\begin{document}

\maketitle

\pauta{Informes e Repasses}
\begin{itemize}
\item Teve uma audiência do DCE com a reitoria, está no youtube. Existe um programa de auxilio para os isentos, já que o RU fechou.
\item A UFSC aumentou o período de quarentena até o dia 30 de abril, mas provavelmente vai continuar por mais tempo esse período. A UFSC está colaborando de várias formas com a situação.
\end{itemize}

\pauta{Debate de Conjuntura}
\begin{itemize}
\item Seria interessante fazer uma nota do Calico, se ater realmente às coisas que estão acontecendo no cenário atual. Existe um jogo político por trás de todo o discurso que tem sido falado pelo governo e é importante o Calico se pronunciar quanto a isso. Não é só idiotice, é mal-caratismo mesmo.
\item No início da pandemia, surgiu muito um discurso contra a China; existe uma recomendação de que não se chame as doenças pelo lugar de onde se originou, para não constranger os países. Existe toda essa conjuntura obscurantista, uma ofensiva contra a ciência. Precisamos também ter o cuidado de não cair nos achismos também.
\item Algo bem problemático é a postura de alguns governos, desprezando o que vários médicos e cientistas apontam e focando simplesmente na economia, falando que tem que trabalhar. Foi feito uma inversão que coloca o povo contra si mesmo, fazendo discursos que fazem o povo acreditar que é necessário voltar a trabalhar.
\item Precisamos pensar no papel do estado numa hora dessas, pois é preciso reconhecer que muitas pessoas dependem sim do trabalho. O Estado, nesse momento, precisaria se colocar a frente para ajudar a população não só na questão financeira mas também no acompanhamento da doença. No governo brasileiro, existe um grande conflito onde muito pouca coisa acaba sendo feita. Precisamos falar pontualmente do nosso direito de ficar em locais seguros, não podemos ficar dependendo da economia e de coisas tão vagas. Temos um Estado que poderia muito bem estar tomando conta do povo e não indo contra isso.
\end{itemize}

\textbf{Encaminhamento:} Fazer uma nota falando a posição do Calico sobre toda a situação atual. Responsáveis: Helena, Luis, Paloma, Hans, Cauê.

\pauta{Assistência Estudantil}
\begin{itemize}
\item DCE já estava lidando de forma alternativa com a pauta. Surgiu também a proposta pelo Cafil da Frente Estudantil de Segurança Alimentar (FESA).
\item Tem como ponto central que não tem mais RU. DCE falou com a reitoria para que um auxílio seja dado aos estudantes isentos, saindo um programa para auxílio de 300 reais por mês.
\item A FESA se mostra com uma proposta não muito efetiva, juntando ajuda dos CAs para auxílio aos estudantes. DCE compõe a frente e logo serão distribuídos alimentos. O Calico poderia ajudar tanto com os alimentos que já tem da doação do integrado ou até mesmo com dinheiro em caixa.
\item A FESA tem feito um acompanhamento bem detalhado e estão organizando distribuições de comida na UFSC.
\item É preciso acompanhar dentro do curso quem precisa de ajuda, saber dizer aos estudantes as opções existentes de auxílio e responder em questões pontuais.
\item É importante participar para ficar por dentro do movimento, ajudar como puder. Tem um pouco de comida na sede do Calico que pode ser doada, só precisa antes ser contabilizada para o TI, só que não é muita coisa.
\end{itemize}

\textbf{Encaminhamento:} Ter um representante na FESA, organizando a participação do Calico na frente. Responsáveis: Luis, Paloma.

\pauta{Propostas de Formação}
\begin{itemize}
\item Existe uma cartilha sobre produção de conhecimento , com um texto bem tranquilo e curtinho, seria legal ler (páginas 7 à 14) e depois discutir sobre o texto como uma forma de formar os estudantes. (https://gtnup.files.wordpress.com/2013/07/cartilha.pdf)
\item Seria bom aproveitar o momento da quarentena para criar espaços online de formação, criar coisas para entreter e aproximar os estudantes.
\item A cartilha sobre produção de conhecimento cita no começo uma carta, que também é um texto interessante. (https://gtnup.wordpress.com/2012/06/08/carta-de-porto-alegre/)
\end{itemize}

\textbf{Encaminhamento:} Formação com a cartilha. Data: 17/04 às 18h.

\textbf{Encaminhamento:} Formação com a carta. Data: 24/04 às 18h.

\presentes {Helena Aires, Mikael Saraiva, Arthur Pickcius, Cauê Baasch, Matheus Roque, Hans Buss, Luis Oswaldo, Paloma Cione, Teo Gallarza}

\end{document}
