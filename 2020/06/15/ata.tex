\documentclass{ata-calico}
\usepackage{indentfirst}
\pagenumbering{arabic}
\begin{document}

\maketitle

\pauta{Informes e Repasses}
\begin{itemize}
\item Teve reunião do subcomitê acadêmico com o comitê central e foram apresentados os resultados do levantamento feito na última semana. A apresentação de slides foi bastante tendenciosa, utilizando os dados para alavancar algo que já estavam tentando fazer.
\item Saiu comunicado da PRAE avisando que vai sair o quarto edital do auxilio aos estudantes.
\item Em relação às notas do vestibular, o colegiado avalia que as notas de corte vão se manter as mesmas, mas os pesos podem ser aumentados de 3 para 3.5 tanto em matemática quanto em redação.
\item O CALICO agora compõe a secretaria do CETEC.
\end{itemize}

\pauta{Retomada das Atividades Pedagógicas}
\begin{itemize}
\item O DCE chamou um CEB (conselho de entidades de base) para se discutir mais a fundo a retomada das atividades. Isso foi chamado levando em consideração a velocidade em que as coisas estão sendo tocadas pela reitoria, quem vem sem muita discussão e com planos já prontos. Percebe-se que os estudantes precisarão preparar uma alternativa, uma própria resolução, para conseguir disputar isso com a reitoria.
\item Parece que se tem dois grupos bem distintos na universidade hoje: os que são favoráveis ao ead e os que se mostram contra. Esses dois grupos se subdividem. Porém, percebendo que o ead tem grandes chances de ser passado, é preciso se esforçar para que isso seja aprovado da melhor forma possível, tentando garantir que seja da maior qualidade possível e que seja o mais inclusivo possível.
\item Percebe-se um viez na pesquisa feita pelo moodle, pois é uma pesquisa na internet para saber quem tem internet.
\item Em diversas universidades, já se tem a implementação do ead e se percebe vários problemas de acesso dos estudantes. Também se vê problemas para os professores conseguirem adaptar o ensino, considerando que não estão parados nesse período e continuam com suas pesquisas.
\item Não se tem muita coisa concreta ainda sobre como seria feito o ensino remoto. Essa adaptação seria feita pelos cursos, com uma avaliação de matéria em matéria. A proposta existente é que tudo seja feito à distância.
\item Foi feita votação com 2 propostas: 6 votos para proposta 1, 1 voto para 2, 2 abstenções.
\begin{enumerate}
    \item A favor de uma minuta que defenda a qualidade do ensino excepcional em caráter emergencial à distância e acessibilidade aos estudantes, ou seja, haverá progressão dos currículos
    \item Elaborar uma minuta contrária a proposta da Reitoria contendo: Contrariedade à substituição do ensino presencial pelo ensino remoto; Autorização a possibilidade de realizar atividades complementares, por adesão, em caráter emergencial e excepcional por meio eletrônico, ou seja, não haverá progressão dos currículos.
\end{enumerate}
\end{itemize}

\pauta{Estatuto do CETEC}
\begin{itemize}
\item Começou a ser discutido a revisão do estatuto no CETEC. É preciso garantir que ele seja reescrito da melhor maneira possível para que depois não se tenha nenhum problema.
\end{itemize}

\textbf{Encaminhamento:} Grupo de trabalho pra revisar o estatuto. Responsáveis: Luis, Paloma e Helena.

\presentes {Cauê Baasch, Helena Aires, Hans Buss, Luis Oswaldo, Mikael Saraiva, João Trombeta, Teo Gallarza, André Régis, Gustavo Fukushima, Paloma Cione, Julien Vaz}

\end{document}
