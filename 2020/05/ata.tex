\documentclass{ata-calico}
\usepackage{indentfirst}
\pagenumbering{arabic}
\begin{document}

\maketitle

\pauta{Informes e Repasses}
\begin{itemize}
\item Foi aberto o terceiro edital do auxílio emergencial da PRAE, de 18 à 20 de maio.
\item Reunião do conselho do CTC: Teve uma situação delicada, de um professor do EPS pedindo afastamento de 3 anos por questões pessoais. Se aponta que o problema é que gera uma dificuldade no departamento de encontrar um novo professor. No final, a votação foi a favor do afastamento. Foi também passado o resultado do formulário feito pelos centros acadêmicos sobre atividades online nos cursos.
\end{itemize}

\pauta{Propostas de alterações pontuais no estatuto do CETEC}
\begin{itemize}
\item Existem propostas sobre o quorum atual das reuniões do CETEC. Uma possível proposta é a de excluir as entidades não representativas da conta do quorum mínimo, já que elas não podem participar da votação. Logo, não faria sentido obrigá-los a participar das reuniões.

Outra possível alteração seria a de diferenciar no estatuto as entidades associadas.
\item Talvez se apareça a proposta de dar poder de voto às entidades não representativas. Se aponta que isso pode ser um problema, pois cria uma dupla representação para os estudantes dentro do CETEC. Além disso, as entidades que participam estão ali apenas pelo espaço físico, não por representarem estudantes.
\end{itemize}

\textbf{Encaminhamento:} Os representantes deverão levar o posicionamento do CALICO à reunião (tirar a ATCT e o TI do quorum)

\pauta{Retorno presencial e ensino remoto}
\begin{itemize}
\item Surgiu no CTC a discussão sobre retorno das atividades. A administração central da UFSC lançou propostas de como retomar as atividades, tanto de forma presencial quanto remota, com também propostas mistas.
\item Uma opção vista em algumas empresas é a adoção de uma mistura de presencial com remoto, com pessoas em grupos de risco ou de alguma forma impossibilitadas de irem presencialmente trabalham de forma remota, enquanto aqueles que podem e estão dispostos têm o presencial, com medidas de segurança impostas.
\item Existe o problema do transporte público, que impossibilitaria muitos de voltarem às atividades presenciais na UFSC.
\item Um problema apontado é que não se pode deixar cair a qualidade de ensino e não se pode negligenciar ninguém nesse processo, não se pode deixar ninguém pra trás. Não é só uma questão de ter condições materiais, equipamentos, precisa-se considerar também que é preciso ter um ambiente adequado e tempo para a volta aos estudos. A rotina de muitos foi mudada, com mais pessoas em casa o tempo todo. Em questão da volta presencial, é apontado o alto risco de ter uma volta às atividades agora.
\item É apontado que, mesmo com os problemas existentes, precisa-se manter a mente aberta para algum tipo de volta no final do ano, considerando todos os estudantes.
\item Fala-se que é preciso considerar não só a qualidade futura, mas a qualidade que teve-se até o momento. A qualidade da formação profissional dos alunos não pode cair, e isso não envolve apenas produtividade, mas também uma formação cívica.
\item Aponta-se que talvez não seja possível manter a mesma qualidade de ensino de forma remota. É preciso muito esforço na elaboração dos materiais, e não se pode negar tudo que vier só porque não se alcança um patamar. Mas tem-se que ter em mente esse patamar e tentar ao máximo alcançá-lo.
\item Uma opinião é que é preciso focar na inclusão digital, que precisa-se levar o acesso à todos os estudantes.
\item Falam da necessidade de ter em mente o impacto das ações tomadas.
\item Em questão da inclusão digital, fala-se sobre a necessidade de uma boa plataforma para de ter aulas de forma remota, principalmente para aulas onde não se pode ter apenas slides e o professor apresentando algo.
\item Toda a universidade precisa responder à reitoria, que por sua vez responde ao Conselho Universitário. Teoricamente, as atividades atualmente estão suspensas, por isso não é possível ter a retomada das atividades atualmente.
\item É posicionada a defesa da qualidade de ensino e da universalização do acesso. Pensar o ensino à distância apenas para o próximo semestre, sem a tentativa de retorno presencial.
\end{itemize}

\textbf{Encaminhamento:} Levar ao CEB os seguintes pontos: (anotar os pontos)

\pauta{Adiamento do ENEM}
\begin{itemize}
\item O ministério da educação decidiu manter o ENEM esse ano. É preciso ter essa discussão sobre o adiamento ou não do ENEM.
\item Traz-se o ponto de que a discussão tem pontos parecidos com a discussão da retomada das atividades acadêmicas.
\item É pontuado que a dificuldade de estudo para os alunos de ensino médio pode se mostrar maior do que a dos universitários. Para além disso, não são todos os estudantes secundaristas que têm acesso à internet em casa. Isso não dificulta apenas o estudo, mas também impossibilita a inscrição no ENEM. É preciso pensar nesse grupo mais fragilizado, entender que esses estudantes fragilizados que perderem essa prova, no jeito que a sociedade se organiza, vão ter ainda mais dificuldade de se locomover na sociedade.
\item Fala-se que precisa também ser levada em consideração a situação não só dos secundaristas, mas também das universidades. No retorno das atividades universitárias, terão que ser atendidos não só os estudantes desse semestre, mas também aqueles que estariam entrando, o que prejudicaria a infraestrutura existente, mesmo que sendo apenas em atividades remotas. Isso dificultaria o acesso na universidade também.
\item Se fala que precisa-se ter em mente que desde a entrada do ministro atual da educação, não tem nenhuma forma de diálogo entre o MEC e as universidades. O MEC, portanto, não tem noção da realidade das universidades com relação às consequências da manutenção, adiamento ou cancelamento do ENEM. Realidade essa que o MEC já deveria ter conhecimento.
\end{itemize}

\textbf{Encaminhamento: A favor do adiamento do ENEM, levando em consideração os alunos mais prejudicados. As decisões e consequências futuras deverão ser lidadas pelo MEC.}

\pauta{Pesos e notas de corte do vestibular}
São apresentados os pesos (multiplicadores de notas) e notas de corte (nota mínima para ser aprovado) para entrada no curso de ciência da computação. Tais notas e pesos são definidas pelo colegiado de curso, levando em conta as disciplinas consideradas essenciais.

O ponto foi levado para a próxima reunião, que será antes da próxima reunião do colegiado de curso.

\begin{itemize}
\item 
\end{itemize}

\textbf{Encaminhamento:}

\presentes {Cauê Baasch, Helena Aires, Luis Oswaldo, Arthur Pickcius, Paloma Cione, Hans Buss, João Trombeta, Enzo Albornoz, Julien Vaz, Gustavo Fukushima, Murillo}

\end{document}
