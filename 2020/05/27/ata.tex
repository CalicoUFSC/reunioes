\documentclass{ata-calico}
\usepackage{indentfirst}
\pagenumbering{arabic}
\begin{document}

\maketitle

\pauta{Informes e Repasses}
\begin{itemize}
\item Na semana passada teve uma reunião do Conselho Universitário, uma das coisas mais importantes que saiu foi a aprovação de uma moção a favor do adiamento do ENEM. A sessão foi gravada no YouTube, mas foi bem longa.
\item Nas últimas semanas foram formados comitês e subcomitês, alguns com representação acadêmica pelo DCE. Temos representação do CALICO no comitê acadêmico.
\item Existe a pauta na UDESC de retomar as atividades na forma de EaD.
\end{itemize}

\pauta{Análise de Conjuntura}
\begin{itemize}
\item A universidade deveria dar o exemplo para a sociedade de como as coisas devem ser, então não se pode deixar que as coisas passem deixando gente para trás. Na situação atual, se vê pressão para a retomada das atividades imediatamente, sem muita consideração para aqueles que não tem a infraestrutura ou capacidade de qualquer forma para voltar agora.
\item Foram feitos levantamentos pelos cursos do CTC para ver a capacidade atual dos estudantes de retomar as atividades. Eles mostram que nem todos os discentes têm essa capacidade, então é preciso levar em consideração as diversas dificuldades dos estudantes ao elaborar planos.
\item Os professores, no conselho do CTC, demonstram pressa em retomar às atividades. A UFSC ficou um tempo parada, começando tardiamente a discussão do retorno às aulas. Mesmo com esse atraso, é preciso de calma para elaborar um bom plano de ação para garantir qualidade à todos.
\item Muitos professores parecem perdidos quanto à como fazer EaD, apresentando em muitas reuniões algum problema técnico também. O movimento docente, em maioria, parece preocupado com a qualidade e com as próprias condições; apenas certos setores se mostram apressados em implementar a educação à distância. Não há muito risco na UFSC de se ver algo absurdo sendo passado.
\item Considerando a situação no CTC e no CETEC, pode-se propor que a conversa sobre as condições e perfis de cada curso seja feita nos colegiados de curso. Existe a preocupação de que as coisas sejam passadas no CTC sem muita consideração com os estudantes.
\item Seria interessante se conseguíssemos conversar com os estudantes da pós-graduação para que consigamos trazer o ponto deles também às nossas discussões. A APG está fazendo esse debate, mas a maior parte dos pós-graduandos do CTC não participa.
\end{itemize}

\pauta{CETEC}
\begin{itemize}
\item Teve a reunião do conselho de unidade do CTC, deixando alguns estudantes preocupados. Assim, surgiu a ideia no CETEC de que fosse feita uma nota sobre o assunto. Além disso, também está acontecendo a conversa sobre a atualização do estatuto do CETEC, pois atualmente a redação se mostra confusa em vários pontos. Outra discussão proposta é a organização do grupo do CETEC, pois várias pessoas que já não participam mais ainda estão nele.
\item Se percebe uma falta de discussão sobre o papel do CETEC realmente no CTC, já que ele não é um centro acadêmico. Essa discussão é importante para que se possa elaborar uma nova redação do estatuto.
\item Precisamos nos fazer presentes na redação do novo estatuto, para que tentemos garantir que nada absurdo passe.
\item Uma possibilidade é pedir para adiarem a elaboração de um novo estatuto, pois coisas em conjunto ficam mais complicadas quando feitas puramente online. Isso pode ser um problema, pois não se sabe quanto tempo mais iremos ficar no isolamento social.
\item Em questão dos grupos, uma possível solução é ter um grupo apenas dos representantes e um grupo aberto para que se tenha essa discussão geral.
\item É importante ter um grupo que operacionalize as reuniões e outro grupo, aberto, para que se possa ter as discussões.
\item Nota aos docentes do CTC, pelo CETEC, recomendando que as discussões sobre o retorno das atividades pedagógicas sejam levadas para os cursos, em reuniões amplas com estudantes e professores, e que os acúmulos sejam encaminhados para o subcomitê acadêmico
\end{itemize}

\textbf{Encaminhamento:} Sugerir ao CETEC que redija uma nota aos docentes do CTC, recomendando que as discussões sobre o retorno das atividades pedagógicas sejam levadas para os cursos, em reuniões amplas com estudantes e professores, e que os acúmulos sejam encaminhados para o subcomitê acadêmico.

\textbf{Encaminhamento:} Participar ativamente da elaboração do novo estatuto do CETEC, da melhor forma possível.

\textbf{Encaminhamento:} Defender o fim do grupo dos presidentes do CETEC no WhatsApp e a criação de um grupo específico para os representantes dos CAs, com caráter operativo, mantendo o atual como um grupo aberto para discussões e repasses de informações relevantes aos estudantes do CTC.

\pauta{Pesos e notas de corte do vestibular}
\begin{itemize}
\item As notas de corte e os pesos no vestibular para Computação priorizam os itens que são avaliados como mais importantes para o curso. Também é levado em consideração o preenchimento de todas as vagas.
\item Os professores se mostram bastante preocupados em relação a isso, têm discussões boas nesse ponto. Levando isso em consideração, poderíamos apenas defender que se mantenha assim, pelo menos até que se apresentem novos pontos.
\item Nessa discussão, temos que nos basear em quem tem a maior capacidade de completar o curso, não algo como quem merece mais.
\item Em relação à transferências internas, atualmente para se transferir para Computação é preciso ter completado pelo menos três matérias da primeira fase. Isso foi colocado em pauta pois uma estudante entrou com recurso na câmara de graduação, pois teve o pedido de transferência negado. Isso entrou pois existe uma regra pautando que todas as vagas precisam ser preenchidas.
\end{itemize}

\presentes {Cauê Baasch, Helena Aires, Arthur Pickcius, Hans Buss, João Trombeta, Lucas Machado, Paloma Cione, Gustavo Fukushima, Mauricio Konrath, Luis Oswaldo}

\end{document}
