\documentclass{ata-calico}
\usepackage{indentfirst}
\pagenumbering{arabic}
\begin{document}

\maketitle

\pauta{Informes e Repasses}
\begin{itemize}
\item Vai começar o concurso de camisetas. Ele vai até 06/03 (sexta-feira) e as votações vão ser no fim de semana (07 e 08/03). O vencedor vai ser anunciado na segunda-feira e ganha uma camiseta.
\item Teve um boato de que o labufsc ia fechar, mas aparentemente não tem nenhum indicativo real disso.
\item CALICO está passando por uma reforma, a parede já foi pintada. Compramos uma estante e uma escada, ainda estamos em busca de um sofá.
\item Matrículas presenciais são dias 27 e 28/02 e 02/03.
\end{itemize}

\pauta{Descartes}
\begin{itemize}
\item Precisamos dar um fim adequado a cada uma das coisas a ser descartada. Sofá pode ser doado para o dce, estante pode ir para uma mercearia.
\item A cadeira estragada pode ir para um ferro-velho, podemos falar com a concap ou deixar junto da caçamba de lixo que vai ser tirado do CETEC.
\item Precisamos tirar o forro do sofá com as assinaturas de limpezas passadas, para manter de recordação.
\item Existem cápsulas de café vencidas, vão testar elas para ver se ainda dá de tomar.
\item O papel do TI com as assinaturas pode ser jogado fora, anunciado no grupo do CALICO, recortado para que parte seja mantida ou doado para o CASIN. Decidido por anunciar por 24h no grupo do CALICO; se ninguém quiser, oferecer para o CASIN; caso contrário, jogar no lixo.
\item Bichos de pelúcia e brinquedos que estavam no CALICO serão doados, com um dos bichos de pelúcia ficando na sede de decoração.
\item Almofada de tronco vai ser jogada fora.
\item As tintas que existem podem ser doadas para o dce ou mantidas na sede por algum tempo caso a gente precise delas. Decidido por manter por um ano por garantia.
\item Os livros vão ser doados para a BU, tirando 6 que são mais 'icônicos' no curso e ficarão na sede.
\item O computador tem que ser arrumado.
\item Os copos serão doados para o CASIN.
\item Fichas de poker vão ser mantidas para futuros eventos, materiais escolares ficam à disposição na sede, quadro magnético vai para doação, tintas pequenas vão ser mantidas (estragadas serão jogadas fora), urna será doada, controles de ps2 vão ser testados e depois doados, própolis será doado.
\item Cartazes que não são de eventos da computação serão jogados fora. Os que são de eventos de cco vão ter cópias tiradas, os originais vão ser guardados e as cópias vão ser expostas.
\item Apostilas vão ser doadas para a BU. Caso não aceitem, vão ser jogadas no lixo.
\end{itemize}

\textbf{Encaminhamento:} Cauê será responsável por avisar ao CETEC que vamos deixar a cadeira para ser tirada junto com os entulhos.

\textbf{Encaminhamento:} Mikael se responsabiliza por dar um fim para a estante: tentar por uma semana e meia falar com mercearias e depois arrumar outro jeito.

\textbf{Encaminhamento:} Cauê vai falar com DCE para doar o sofá, também se responsabiliza por tirar o forro.

\textbf{Encaminhamento:} Mikael vai testar os controles de ps2.

\presentes {Cauê Baasch, Helena Aires, Arthur Pickcius, Hans Buss, Paloma Cione, Mikael Saraiva, Fábio Coelho}

\end{document}
