\documentclass{ata-calico}
\usepackage{indentfirst}
\pagenumbering{arabic}
\begin{document}

\maketitle

\pauta{Reforma e Limpeza da Sede}
\begin{itemize}
\item A procura de móveis foi feita, sofás estão na faixa de 200-300 reais na olx. Armários foram mais difíceis de achar, um por 140 e outro por 180, que parece valer mais a pena pois já cobre frete e montagem.
\item Precisa conferir se tudo cabe certinho antes de comprar.
\item Também foi procurado móveis que serviriam para armazenamento além de sentar. A maioria se encontra na faixa de 150~250 reais.
\item Existe a possibilidade de colocar algumas prateleiras na parede ao invés de comprar um armário, juntando com o sofá-baú (ou seja lá o nome, procurar).
\item Prateleiras não daria certo, não tem espaço suficiente pra isso, com o ar já ali e as janelas. Daria para colocar umas 2 prateleiras só.
\item Precisamos de mais informações e todo mundo presente numa reunião na sede para conseguir decidir isso. Proposta uma nova reunião para decidir tudo certinho.
\item Precisamos de uma reunião aberta para conseguirmos nos desfazer dos móveis que serão trocados, apenas uma questão burocrática.
\item Pode ser que seja muito tarde fazer limpeza+pintura na semana do dia 17.
\item O tempo parece de boa, já que ainda teria outra semana antes do começo das aulas.
\item Talvez seria bom fazer uma limpeza superficial, a pintura e depois uma limpeza mais minuciosa para garantir que a gente limpe a sujeira da pintura depois.
\item Precisamos comprar as coisas na semana do dia 10 para não ter chance de atrasar com o feriado de carnaval.
\item Precisa comprar a tinta, ver mais para frente a cor, e ferramentas para pintar.
\end{itemize}

\textbf{Encaminhamento:} Reunião aberta do CALICO para se desfazer dos móveis que serão substituídos. Data: 17/02 às 16h.

\textbf{Encaminhamento:} Limpeza e pintura da sede. Data: 13/02 às 9h.

\textbf{Encaminhamento:} Comprar a tinta e outros materiais necessários para a pintura. Responsável: Paloma.

\pauta{Varanda}
\begin{itemize}
\item Precisa limpar a varanda e dar uma revitalizada. Talvez comprar uns pallets e umas almofadas, pra ser algo resistente à chuva. Precisa limpar e consertar os vidros, mas ai é com o dinheiro do cetec.
\item Ver como vai funcionar o diálogo depois com o pessoal pra ocuparem o espaço.
\item Talvez chamar os outros cas pra ajudar na limpeza, talvez convocar os cursos em si.
\item Dar uma olhada sobre colocar réguas ou aumentar de algum modo a quantidade de tomadas. Provavelmente iria ajudar a chamar mais gente. Ver com o caeel.
\item Comprar as coisas e depois levar pro cetec um rateio pra quem quiser colaborar.
\item Levar o sofá do CALICO que for descartado não é uma boa ideia, pois é muita sujeira e pode dar problema com o pessoal destruindo.
\end{itemize}

\textbf{Encaminhamento:} Falar com o CAEEL sobre as tomadas. Responsável: Luis.

\textbf{Encaminhamento:} Limpeza da varanda. Data: 13/02.

\textbf{Encaminhamento:} Falar com o resto dos centros acadêmicos para ajudar na limpeza. Responsável: Luis.

\textbf{Encaminhamento:} Ver móveis baratos e resistentes para a varanda. Responsável: Cauê.

\pauta{Indicação do Colegiado}
\begin{itemize}
    \item Luis foi indicado para a câmara de administração do colegiado do departamento.
\end{itemize}

\pauta{Identidade Visual}
\begin{itemize}
\item Precisamos decidir se continuamos com o Sheldon, começando de novo o processo, ou se fazemos com outra pessoa.
\item Alguns não gostaram do trabalho anterior do Sheldon, mas podemos falar com ele pedindo pra tentar alguma coisa diferente. Ele faz o trabalho bem direitinho e não tem nenhum problema fazer com ele.
\end{itemize}

\textbf{Encaminhamento:} Marcar reunião de orçamento com o Sheldon. Responsável: Cauê.

\pauta{Início da Organização dos Eventos}
\begin{itemize}
\item Bar pós prova de pré-cálculo, dia 28/02: precisa divulgar o evento, só.
\item Trilha, dia 02 ou 03/03: precisa escrever um guia, escolher qual trilha vai ser e divulgar, além de ver certinho qual data é boa, precisa de um dia de sol.
\item Linguicinha, dia 04/03: comprar duas grelhas elétricas, ficava por menos de 350 reais as duas. Precisa olhar preço do pão, linguiça, legumes, bebidas e o resto do que for precisar (principalmente pão, que é o mais caro).
\item Tem registros da linguicinha passada falando de quantidades, mas a galera comeu bastante. Seria bom ter um liquidificador pra fazer o pão de alho de novo.
\item Ter mais opções vegetarianas seria uma boa, talvez usar esse semestre de experiência pra ver se a galera aceita bem ou não.
\item Chicco, dia 13/03: precisa ver lugar certinho, onde vai ser e preços e tudo mais. Evento para umas 50 pessoas, indo até 22h normalmente.
\item Olhar preços de salão de festa no Granville e outros condominios que conhecemos para comparar e ver qual é o mais acessível.
\item Noite de Jogos: tem que decidir um dia. Na primeira semana não tem mais espaço, surgiram sugestões de fazer na segunda e terceira semanas de aula.
\end{itemize}

\textbf{Encaminhamento:} Procurar duas grelhas para comprar. Responsável: Luis.

\begin{center}
\begin{tabular}{|c|c|c|} \hline
    \multicolumn{3}{|c|}{Fevereiro} \\ \hline
    13 & 17 & 28 \\ \hline
    Limpeza da sede+varanda & Reunião & Bar pós pré-cálculo \\ \hline
    & & Matheus \\ \hline
\end{tabular}

\vspace{.5cm}
\begin{tabular}{|c|c|c|c|c|c|} \hline
     \multicolumn{6}{|c|}{Março} \\ \hline
     02-03 & 04 & 05 & 06 & 13 & 16 \\ \hline
     Trilha & Linguicinha & Tour & Reunião & Chicco & Noite de Jogos \\ \hline
     Cauê & Matheus, Paloma & & & Matheus, Luis, Lucas, Paloma & Matheus \\ \hline
\end{tabular}
\end{center}

\presentes {Fábio Coelho, Cauê Baasch, Matheus Roque, Mikael Saraiva, Luis Ganoza, Lucas Sousa, Helena Aires, Paloma Cione}

\end{document}
