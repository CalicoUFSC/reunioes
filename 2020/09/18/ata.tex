\documentclass{ata-calico}
\usepackage{indentfirst}
\pagenumbering{arabic}
\begin{document}

\maketitle

\pauta{Informes e Repasses}
\begin{itemize}
\item Luis: quarta-feira (16/09) teve reunião do conselho do CTC. Foi aprovada um monte de mudança de categoria de professor, foi discutido o próximo representante do CTC no CUn e também foi discutido as eleições de direção de centro. A única diferença para os últimos anos é que dessa vez vai ser online. Foi decidido que a proporcionalidade dos votos vai continuar a mesma (70:20:10 para professores, técnicos e alunos, respectivamente), calculada pelo número total da categoria, não de votantes.
\item Luis: Começaram as discussões no colegiado de curso sobre a curricularização da extensão, pelo menos no fórum. Já houve essas discussões antes, mas voltaram agora.
\end{itemize}

\pauta{Sugestão de temas para SECCOM}
\begin{itemize}
\item Cauê: é importante puxar temas de conjuntura, estar sempre se atualizando em questões que muitos podem não perceber que influenciam no nosso futuro como profissionais. Uma ideia é o sistema público de pesquisa e seu desmonte, a soberania nacional na tecnologia; dados abertos e a transparência governamental, e sua importância agora na pandemia; propriedade intelectual, como no caso do surgimento de uma vacina.
\item Hans: Pensando no perfil do curso de interesse no mercado de trabalho acima do interesse mais acadêmico, poderia se pensar em algo que una de alguma forma os dois.
\item Luis: Talvez possamos aproveitar o espaço para mais uma vez trazer uma discussão sobre a curricularização da extensão. Mesmo que não puxe tanto o interesse de muitos estudantes, ainda é um tema importante.
\item Luis: Uma opção é levar a discussão para o grupo aberto do CALICO, para conseguirmos durante a semana que vem levantar temas e semana que vem mesmo já dar uma resposta para a organização da SECCOM.
\end{itemize}

\pauta{Avaliação da atuação do CETEC e do Conselho do CTC}
\begin{itemize}
\item Luis: Na penúltima eleição do conselho, foi tirada uma comissão eleitoral (com um professor, uma técnica e uma aluna). Nessas discussões, surgiu a questão do peso dos votos. A aluna trouxe o ponto de volta para o CETEC, onde se elaborou uma proposta de votos paritários (de peso de 1/3 para cada categoria), sendo pelo número de votantes, não pelo número total da categoria. Foi feito um levantamento de como funciona a votação nos outros centros.No CETEC, todos os centros acadêmicos aceitaram essa proposta, também aprovando uma campanha sobre o assunto. Na reunião do conselho, a proposta não passou (apenas 3 professores votaram na proposta dos estudantes). Agora é importante avaliar a atuação do CETEC, os ganhos nisso e os erros cometidos.
\item Cauê: Já no levantamento feito, percebeu-se que os pesos atuais (70:20:10) são na verdade diluídos em todos os estudantes. Ou seja, apenas se consegue 10\% pelos estudantes se todos votarem. O objetivo da campanha feita era fazer uma pressão prévia à reunião. No conselho, há uma composição tão ruim que não importa a defesa que tenha na reunião em si, apenas a pressão que pode ser levantada previamente com o corpo estudantil como um todo do CTC. Precisamos pensar nos próximos passos partindo disso. Talvez uma via de ação é o boicote, já que não haveria perda real na votação e mostraria ao mesmo tempo a dessatisfação dos estudantes nesse processo.
\item Luis: Essa mobilização no CETEC, mesmo que não tenha sido grande, foi um passo importante na conscientização dos centros estudantis. Mostrou que os estudantes têm demandas que não são conquistadas pelo 'amiguismoo' com os professores. Essa conscientização foi um ganho político grande, mas é um problema porque se concentra na diretoria dos centros acadêmicos. Essa proposta de boicote seria efetiva apenas com mais de uma chapa. Um problema que teve foi que muitos poucos representantes estudantis participou ativamente das discussões, tanto no CETEC quanto na reunião. Outro ganho foi que a posição que o CALICO mantém desde sempre no CETEC, que não se consegue coisas apenas pelo 'amiguismo' com os professores, foi confirmada, mesmo que do jeito difícil.
\item Cauê: É bom jogar esses alunos do CETEC contra o conselho do CTC, expor essas divisas e mostras a importância de confrontar os professores para ser ouvido na universidade. Uma ideia seria talvez fazer uma urna paralela, para se ter a comparação dos votos. O boicote seria bom para ter registrado que nenhum aluno quis participar nesses termos. Outro ponto é o que os professores falam nessas reuniões, se portam de jeitos que não fariam se as reuniões fossem transmitidas abertamente. Houve um ganho nesse processo, foi um grande aprendizado. Se fez o que dava, marcamos posição e lutamos, mesmo que não tenhamos uma vitória final.
\item Paloma: Foi bem importante a nossa participação no conselho. Precisamos fazer com que essa pauta tenha um peso mais histórico, insistir no ponto até que consigamos o voto paritário.
\item Luis: Um ponto falho foi que CETEC tirou de sondar os professores e técnicos sobre a posição deles. Fiquei de ver isso no INE, mas acabei não fazendo por puro relapso. Deveria ter feito.
\item Luis: Levantamentos contra que partiram dos professores foram coisas como alunos passarem menos tempo na universidade. Foi levantado o ponto da ilegalidade do processo; o que todos os outros centros fazem (e outras universidades na escolha do reitor) é uma consulta informal com paridade, mas depois quem faz a lista tríplice é feita pelo conselho seguindo o resultado dos votos. Esse ponto foi levantado depois do período de inscrições, quando os alunos não poderiam mais fazer a defesa. Foi pedida uma fala dos estudantes, que acabou sendo incrivelmente curta. Foi um processo bem atropelado.
\end{itemize}

\pauta{Convocação do CEE da UCE}
\begin{itemize}
\item Luis: O Conselho Estadual de Entidades está pra União Catarinense de Estudantes assim como o CEB está pro DCE.
\item Marco: Na UCE existem 11 diretores executivos; na gestão atual têm 2 diretores da oposição, eu e o Antônio. A UCE é uma entidade histórica no nosso estado, por muito tempo foi a mesma entidade do DCE da UFSC e esteve em vários momentos importantes. Nos últimos 20,30 anos em vários momentos acaba se restringindo a apenas passar informações. No estado estamos com duas instituições sob intervenção e várias outras com processos bem atropelados. A diretoria da UCE vem a muito tempo ignorando essa tarefa de reunir todos, se mostra não estar preocupada com isso. Uma forma possível de conseguirmos organizar os estudantes é o CEE. Estamos tentando articular um chamado para a UCE cobrando que seja chamado um CEE, mostrando que as entidades de base têm interesse e estão preocupadas com o andamento da UCE. A nossa proposta é que seja chamado um para debater a conjuntura e a ação nos próximos semestres.
\item Cauê: Essa foi uma questão também levada na reunião do DCE de ontem final da tarde. No fim foi votado que se redija uma nota, uma carta à diretoria da UCE para que seja convocado esse conselho. As entidades gerais são muito importantes na articulação da luta dos estudantes, para que essa luta seja potencializada, tenha mais força. Elas têm esse papel de estar dirigindo mesmo a atuação, puxando as discussões. É importante que tenhamos esse espaço para que a UCE se mostre mais na vida dos estudantes, que coloque os estudantes para travar a própria luta.
\item Luis: É importante tirarmos algum encaminhamento para ajudar nisso, talvez um post no Instagram.
\item Marco: Tem uma reunião da diretoria plena da UCE amanhã, seria importante ter uma manifestação sobre isso até amanhã. Algo que talvez ajudasse é marcar a UCE em posts sobre o assunto, já faz um peso legal.
\item Cauê: Além de apoiar a nota do DCE, algo que se pode fazer é levar o tópico ao CETEC para conseguir o apoio dos outros centros acadêmicos. Também talvez redigir uma notinha pequena, talvez um parágrafo ou dois, para estar postando no Instagram.
\end{itemize}

\textbf{Encaminhamento:} Apoiar a publicação do DCE, seja assinando ou divulgando;

\textbf{Encaminhamento:} Levar essa publicação do DCE e o tópico para o CETEC para conseguir mais apoio;

\textbf{Encaminhamento:} Fazer algum conteúdo em nome do CALICO para movimentar os estudantes. Responsáveis: Helena e Paloma.

\presentes{Cauê Baasch, Helena Aires, Luis Oswaldo, Paloma Cione, Pedro Aquino, Teo Gallarza, Matheus Roque, Arthur Pickcius, João Paulo T.I.Zanette, Hans Buss, Marco Antônio}

\end{document}
