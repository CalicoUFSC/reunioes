\documentclass{ata-calico}
\usepackage{indentfirst}
\pagenumbering{arabic}
\begin{document}

\maketitle

\pauta{Informes e Repasses}
\begin{itemize}
\item Cauê: O ajuste excepcional de matrícula vai até dia 10 de setembro, com avaliação dos pedidos feita todos os dias. Até o dia 10, também, vai ficar aberto para destrancamento de matrícula. O trancamento do curso e de matrículas específicas vai ficar aberto até o fim do semestre.
\item Cauê: Na última sessão do CETEC foi tirada a Gabriela como representante para a comissão eleitoral do centro. Até a última eleição, o peso dos votos para direção de centro no CTC era 70\% professores, 20\% para técnicos e 10\% para estudantes. Está sendo elaborada uma proposta para uma eleição paritária, onde o conjunto dos votos de cada categoria teria peso de 1/3.
\end{itemize}

\pauta{Representações discentes no colegiado de curso}
\begin{itemize}
\item Luis: No começo do ano, foi feita uma reunião do CALICO para definir as novas representações discentes, mas as indicações acabaram não sendo feitas. A indicação anterior já venceu, então teoricamente nossa representação agora é extra-oficial, mesmo que continuemos votando. É importante que a gente faça as nomeações, para não ter nenhum problema.
\item Cauê: Estou meio sem condição de continuar nessa representação, então gostaria de sair. Além disso, é um espaço bem formativo, para entender mais como funciona melhor a organização da universidade e começar a pensar política.
\item Paloma: Tenho problemas de continuar agora de forma remota, pois meu espaço não é muito bom quando preciso me manifestar. Até poderia continuar, mas seria melhor alguém que possa acompanhar com mais qualidade.
\item Matheus: Não estava realmente pensando em pegar essa vaga, mas consigo pegar se necessário. É uma boa oportunidade de começar nas coisas.
\end{itemize}

\textbf{Encaminhamento:} Ficam como nomeações Paloma Cione, Luis Oswaldo, Matheus Roque e Pedro Aquino, saindo Patrick Machado e Cauê Baasch.

\pauta{Planejamento de atuação do CALICO}
\begin{itemize}
\item Luis: Nesses últimos meses de pandemia, a atuação do CALICO não foi o que poderia ter sido (com motivos válidos, mas ainda assim). Agora que as aulas estão voltando, é preciso pensar em como aproximar mais o centro acadêmico dos estudantes. Fazer mais eventos, utilizar mais as redes sociais. Pensar mais em espaços novos, já que muitos dos nossos eventos não poderão acontecer (como o Linguicinha), e como substituir o que já tinha. Tanto questão de eventos de integração como eventos políticos.
\item Matheus: A única forma de integração que a gente tem agora seria por mensagens e chamadas, talvez um pouco das redes sociais. Talvez alguma forma de tutoragem, com alunos mais antigos ajudando os calouros com toda a situação e os estudos, de forma a integrar os veteranos com os calouros.
\item Cauê: Em primeiro lugar, é importante que a gente fique bem atento nas questões das aulas, em qualquer problema que possa acontecer, para ajudar os alunos e não deixar as coisas só rolarem. Quanto à integração, conseguimos fazer grande parte da nossa calourada antes da pandemia, mas ainda é bom ter um acompanhamento com os calouros e fazer algum esforço à mais para esse contato e essa ajuda que podemos fazer. A noite de jogos é algo que funciona bem, em questão de integração. Quanto à discussões, é interessante ter coisas sobre conjuntura em reuniões até para tirarmos posições. Talvez seja interessante voltar a ter reuniões ordinárias, achar alguma forma de substituir as passagens em sala. Também estamos chegando no final da gestão, é preciso começar a falar do processo eleitoral, se será adiado ou se vai ser o processo todo online mesmo.
\item Luis: Integração contínua ao longo do semestre sempre foi um problema para o CALICO, então na situação atual é um problema maior ainda. Em relação às reuniões ordinárias, é importante lembrar que não é preciso ter pauta fixa para elas, é um espaço para discutir o que surgiu ao longo da semana.
\item Matheus: É importante ter reuniões regulares até mesmo para que a atuação do centro acadêmico não seja apenas em surtos de energia ou problemas. Em questão do rolê de apadrinhamento/tutoria, talvez ser algo entre eles mesmo, ou de forma mais anônima.
\item Arthur: Seria bom passar para os calouros o costume de criar grupos para as disciplinas. Talvez criar canais no discord para isso.
\item Luis: Algo a se pensar é o quanto o CALICO pode servir como vetor de informação para os alunos do curso. Outros centros acadêmicos as vezes tem esse papel mais divulgador, usando o instagram ou outra mídia social. Talvez seja algo interessante de o CALICO começar a fazer, talvez gerar material a partir das reuniões. Seria bom centralizar as mídias do CALICO. Talvez pensar em uma reorganização do servidor do discord, mesmo, para ficar mais convidativo para as pessoas, talvez fazer ao mais equivalente à sede do CALICO na hora do almoço.
\item Matheus: Uma coisa é que o server no CALICO parece algo muito mais sério comparado à servers que existem da computação (tipo o Computaria). Muitas vezes as pessoas podem não se sentir muito confortáveis, por parecer algo muito formal. O que talvez seja uma saída pra isso é talvez sincronizar as coisas com outro servidor, tendo um espaço mais formal e um mais informal, não sendo necessariamente algo paralelo ao curso.
\item Matheus: Algo de se pensar para as reuniões é algo mais tranquilo para atrair mais a atenção dos calouros, não apenas coisas tão burocráticas, para que consigam participar das reuniões.
\item Luis: Sobre as reuniões, se fossem semanais a gente conseguiria discutir melhor qualquer coisa que surja na semana, mas é preciso ver a questão de todo mundo conseguir participar mesmo.
\item Cauê: É preciso considerar que as pessoas agora talvez não tenham tantas condições para acompanhar as reuniões se forem semanais. Se for quinzenal, é sempre possível chamar outras reuniões caso surja algo mais urgente.
\item Matheus: Se for semanal, isso pode servir para ter um acompanhamento melhor de tudo.
\item Cauê: Uma ideia é intercalar eventos do CALICO com as reuniões, fazendo com que sejam quinzenais. Isso ainda daria um contato semanal com os alunos.
\end{itemize}

\textbf{Encaminhamento:} Ter reuniões regulares quinzenais, intercaladas com momentos de integração dos estudantes (como noite de jogos).

\textbf{Encaminhamento:} Utilizar mais o Instagram pra tentar divulgar informações e discussões pertinentes.

\textbf{Encaminhamento:} Organizar um esquema de tutoria entre os veteranos e os calouros.

\presentes{Cauê Baasch, Helena Aires, Luis Oswaldo, Paloma Cione, Matheus Roque, João Paulo T.I.Zanette, Hans Buss, Nicole Schmidt, Arthur Pickcius, Cristian Alchini}

\end{document}
