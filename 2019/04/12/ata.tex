\documentclass{ata-calico}
\usepackage{indentfirst}
\begin{document}

\maketitle
\pauta{Informes}
Cauê informa que quarta-feira que vem o caravela promoverá o Calouros a Bordo, terão duas palestras, uma sobre "o que o curso não mostra" e outra sobre computação quântica, às 18:30 no EPS.

Luis diz que com Cauê começou a redigir um ofício para fazer a requisição de interpretes, já pesquisaram a legislação procurando por leis que exijam garantia de interpretes para dar suporte ao documento. Caso alguém queira ajuda, conversar com eles.

Cauê diz que CALA aceitou a proposta da roda de conversa. Na quarta-feira uma pequena comissão se reuniu com eles e agora estão decidindo coisas como uma boa data e assim que mais informações surgirem serão repassadas.

Cauê diz que a reunião com CALIPRO não foi muito frutífera, Luis complementa que o foco deles é suporte voltado à integração, o que não é nosso problema atual. Eles se colocaram a disposição caso precisemos de algo que se encaixe na proposta deles.

Lucas diz que o INTERPET chamou ex-petianos para entender melhor a situação, já que o PET computação não está participando das reuniões. Lucas diz que foram numa reunião do INTERPET na segunda e redigirão um texto relatando a situação e diz que foi proposto que isso fosse levado para reunião do CA. Luis questiona se isso é um pedido de inclusão de pauta, Cauê diz que é algo importante para representar os alunos. O assunto é adicionado como pauta não prioritária.

Patrick diz que não obteve resposta do TDC.

\pauta{Evento Dev Circle}
Caio diz que Lohan, estudante da graduação que participa de um grupo de estudos de desenvolvimento de software, propôs um evento na UFSC. O evento seria 4 palestras, IA, carreira, gestão e não se lembra agora do quarto. Eles precisam de um auditório, que será alugado para o começo do mês que vem. Caio gostaria de ajuda para conseguir o auditório, preferencialmente o EPS, Cauê diz que já fez reservas o EPS e não é necessário professor, é simples. Lucas questiona o quão válido é apenas reservarmos o espaço para terceiros fazerem o evento, Caio diz que seremos co-parceiros e eles farão as palestras, como na SECCOM. 

Caio irá realizar a reserva do EPS.

\pauta{Coordenadoria do curso}
Cauê diz que estamos a duas semanas sem coordenador. Questionada, a secretaria diz que serão escolhidos coordenador e vice, porém não podem dar uma previsão de quando os cargos estarão preenchidos.

Luis diz que teremos um coordenador que não passou por eleição, o coordenador será escolhido apenas por estar a mais tempo no departamento. É importante que passemos por um processo de eleição, podemos elencar possíveis professores para conversar sobre a possibilidade de ser coordenador do curso. Cauê diz que processo eleitoral acontecerá novamente no semestre que vem, uma vez que o coordenador escolhido pelo centro ficará apenas até o final do semestre. Podemos procurar professores com horas a realocar, bater de porta em porta para conversar.

Mikael, Luis e Cauê irão elencar possíveis professores.

\pauta{Reforma do currículo}
Uma parte da gestão esteve em reunião com o Cancian. Luis diz que ele quer reestruturar toda maneira de ensino, existem pontos positivos e negativos. Estavamos sem permissão para ver o projeto pedagógico mas agora temos acesso, ele será enviado nos canais do CALICO após a reunião. Precisamos conversar sobre o projeto e decidir o que apoiamos e o que queremos mudar, Cauê diz que é muita informação envolvida e é bastante complexo, é preciso chamar o curso para discussão para levar até ele. Lucas questiona como podemos ajudar, Cauê diz que a reforma é baseada em competências, é feito perfis de egresso e para cada perfil são elencadas competências necessárias, e serão fornecidas disciplinas para ensiná-las aos alunos, nossa ajuda seria a parte mais mecânica de preencher coisas no sistema. Paloma diz a proposta é que todas as disciplinas tenham carga extra-classe. Patrick explica que no começo a carga horária de aula é maior, e conforme os semestres vão passando o trabalho extraclasse é maior.

Patrick sugere que Cancian seja chamado para explicar, Luis diz que a ideia era que todos lessem e debatessem, e a presença do Cancian não é necessária. Mikael aponta que nem todos vão ler e Patrick diz que é complexo, Luis diz que a ideia da reunião é justamente discutir para que todos entendam. Cauê propõem que aluguemos um auditório para explicar um pouco da reforma e como ter acesso às informações, e em outro momento ela é discutida. Lucas diz que a discussão não deve ser feita na reunião do CA por demandar muito tempo, o que é concordado.

Luis, Cauê, Paloma, Lucas, Trombeta e Patrick irão escolher uma data para apresentar o projeto para os alunos e organizar futuras discussões.

\pauta{Espaço de estudo no INE}
Cauê diz que temos um problema de falta de lugares de estudo coletivo, a BU não é espaço para isso por ter poucos computadores e exigir silêncio, LABUFSC exige silêncio e é individual. Cauê comenta o fizeram no CCE, onde era a editora UFSC fizeram um espaço de estudos coletivo com materiais necessário. É preciso que pressionemos para existência de espaços assim. Caio comenta que a sala do Bosco é grande e ele irá se aposentar, ele acha que a sala será liberada. Luis comenta que pegar uma sala toda talvez não seja viável, assim como professores estão se aposentando novos professores estão sendo contratados. Caio comenta como o sexto andar do INE é um plano muito futuro e precisamos de algo imediato. Cauê diz que independente disso precisamos expor a demanda. Caio diz que existe uma sala gerenciada pelo CAME e pelo CAMAT com propósito similar. Luis reforça que não temos espaço físico no INE, com apenas uma solicitação não conseguiremos resultados, precisaremos brigar pelo espaço, o que não é algo que podemos fazer no momento atual. Cauê meciona como o Caravela fica onde antes era uma sala vazia, o laboratório utilizado pela Vânia também estava vazio. Caio diz que podemos conversar com o CETEC e a longo prazo ter um ambiente para o CTC. 

Sobre professores com salas vagas é comentado como poucos professores aceitariam ficar responsáveis por isso. Luis comenta como por regulamento do INE salas não podem ter acesso aberto. Caio diz que se tivermos a sala colocar computadores é o menor dos problemas, podemos comprar com dinheiro do CALICO, e nosso maior problema é espaço físico, e o acesso controlado deverá ser feito de toda maneira.

Precisamos fazer a solicitação da sala antes de tomar ação.

Luis fará solicitação do espaço.

\presentes{Paloma Cione, Luis Oswaldo, Mikael Saraiva, Cauê Baasch De Souza, Leonardo Kazuyoshi, Lucas Cavalcante de Sousa, Arthur Mesquita, Caio Oliveira, Bruno Pamplona, Marcos Tomaszewski, Patrick Machado}

\end{document}
