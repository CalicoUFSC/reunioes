\documentclass{ata-calico}
\usepackage{indentfirst}
\begin{document}

\maketitle

\pauta{Informes gerais}
Hans informa a data do Conune (19 e 23 de Junho), apontando que é importante discutir sobre em futuras pautas.

\pauta {Caravela HC}
Cauê diz que precisamos trazer a questão do Caravela pro Calico e mostra o documento oficial (em anexo) feito para reconhecermos o apoio ao Caravela. Todos concordam e o documento é assinado.

\pauta{Colegiado do Departamento}

Caio comenta como a representação nunca foi feita no departamento, e que poderíamos usar esse espaço para contestar muitas coisas que acontecem. Como nunca foi oficialmente dito quantas cadeiras nós tínhamos, essa pauta nunca foi abordada corretamente. Atualmente, temos 6 cadeiras no colegiado do departamento, totalizando 12 pessoas (titulares e suplentes), não podendo estes serem calouros nem representantes de outros órgãos de colegiados. Caio diz que houve uma lista de pessoas designadas um tempo atrás, mas grande parte já se formou ou não está mais aqui. Mikael pergunta se já poderíamos escolher nomes agora e Cauê sugere abrir uma convocação, mas Caio prefere chamar isso em reunião, para podermos indicar pessoas engajadas com o Calico.\newline

\textbf{Encaminhamentos:} Será feita uma convocação para reunirmos nomes a serem indicados.

\pauta {Coordenador do Curso}
Na última semana, ocorreu uma reunião de colegiado para definir o novo coordenador do curso. Como não houve nenhuma chapa candidata, estamos atualmente sem coordenador nem secretário. Cauê diz que precisamos pressionar a diretoria do CTC. Sem o coordenador do curso, tudo o que está a nível de colegiado está congelado, como o cancelamento de matrículas. Gava diz que a nova turma de formandos precisa que a secretaria informe para a UFSC a lista de possíveis alunos a se formarem. Caio diz que, de acordo com a direção de centro, há a previsão de um novo secretário ser indicado a partir de abril.

Luis comenta que os professores não tem mais interesse na coordenadoria pois o governo atual cortou a bonificação extra e que poderíamos fazer uma lista de professores que possivelmente seriam bom coordenadores para tentar convence-los ao cargo.

Caio comenta que o presidente do colegiado é o coordenador do curso, não podendo haver reuniões do colegiado sem o tal.\newline

\textbf{Encaminhamento:} Mandar e-mails avisando a situação atual e informando que o Calico irá entrar em contato com a diretoria do Centro.

\pauta{Acessibilidade das aulas}
Luis diz que a UFSC não possui intérpretes suficientes para os alunos, sendo este um problema da gestão da universidade. Comenta que foi no SAAD (Secretaria de ações afirmativas e diversidade), e que já foram feitos vários requerimentos, sendo aparentemente um problema recorrente, mas que a UFSC não toma nenhuma ação. Apesar de ser um problema da UFSC, podemos começar a nos movimentar sobre isso, comentando que poderíamos até mesmo parar as aulas até conseguirmos intérpretes suficientes para o calouro Guilherme. Caio diz que poderíamos levar essa pauta ao CETEC, pois algo similar poderia ter acontecido. Luis diz que poderíamos levar a pauta ao DCE. Eduardo diz que se fizéssemos uma greve, pressionaríamos diretamente a pessoa responsável. Caio diz que pensou em levar a pauta ao CETEC justamente para podermos pressionar o diretor de centro. Paloma diz que podemos contatar o DCE como CETEC, para termos mais volume.
Caio sugere montar um projeto e contatar a direção de centro diretamente, pois estes tem poder de pressionar a UFSC.

Letícia (interprete) diz que já houveram alunos surdos de outros cursos, mas que é um dos primeiros de Florianópolis que não faz libras.

Guilherme relata que enviou e-mails e contatou a UFSC, e ninguém fez absolutamente nada, recomendando o movimento estudantil a parar as aulas para uma mobilização mais forte. Luis diz que podemos convocar uma reunião e vermos quão a ideia de paralisação é viável. Gava comenta que uma paralisação nesse momento pode ser muito difícil, podendo antes recorrermos ao CETEC, DCE, diretor de centro, etc, para depois  recorrer a uma força mais bruta. Luis diz que a gestão do CTC está ciente do que está acontecendo, mas que poderíamos organizar uma reunião caso as vias normais não funcionem.


Hans diz que a principal força dos estudantes tem que partir dos calouros, visto que eles são colegas do Guilherme. E que essa solidariedade deve se espalhar sobre os veteranos.\newline

\textbf{Encaminhamentos:} Luis irá levar a pauta em nível de CETEC e DCE e, junto com Hans, Leonardo e Guilherme irá organizar a reunião com os calouros. 


\presentes{Paloma Cione, Luis Oswaldo, Caio Pereira, Marcos Tomaszewski, Mikael Saraiva, Cauê Baasch De Souza, Hans Buss, Leonardo Kazuyoshi, Gabriel Meireles, André Fernandes, Arthur Mesquita Pickius, Northon Vinicius, Bernardo Schmidt, Carlos Eduardo Libardo, Eduardo Ramme, Gustavo Gava, Lucas Calvacante de Sousa, Naiara Sachetti, João Paulo Tiz, Bruno Pamplona, Guilherme Rodrigues Andariola, Letícia Regiane Tobal (Intérprete), Paulo A. C. Júnior (Intérprete)}

\end{document}
