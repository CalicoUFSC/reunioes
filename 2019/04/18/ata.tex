\documentclass{ata-calico}
\usepackage{indentfirst}
\begin{document}

\maketitle

\pauta{Repasses}
Repasses da reunião do conselho de centro:
Será aberto um concurso para preencher 4 vagas de professores do INE.
Foi feito um panfleto para apresentar o CTC.

\pauta{Apresentação da nova identidade visual}
Foi apresentada a nova logo do Calico, para podermos aprova-la. Patrick diz que a logo não remete de prontidão ao Calico, devendo ser algo mais esteriotipado. Paloma diz que a parte escrita está boa, mas o desenho deve ser algo menor ou mais visual. William dá a ideia de construir um formulário perguntando para todos se eles aprovam ou não, e Patrick diz que poderíamos também aceitar sugestões.\newline

\textbf{Encaminhamentos:} Luis irá criar um formulário aberto pedindo sugestões para mudar a logo.


\pauta{Ofício}
É mostrado o ofício que o Calico fez para levar a reitoria,relatando que a UFSC está descumprindo a lei e pedindo mais intérpretes. Guilherme relata que já fez uma denúncia ao Ministério Público e que temos que aguardar o prazo de resposta.

Luis diz que apesar disso, podemos enviar o ofício para mostrar que o curso se importa  e sabe o que está acontecendo. 

É perguntado também se não deveríamos, através do oficio, colocar mais pressão na reitoria, talvez colocando algo com um tom mais agressivo.
Foram feitas algumas alterações no ofício e este será revisado por algum profissional da área do direito.

\textbf{Encaminhamentos:} Tiz irá falar com sua amiga formada em direito parar revisar o ofício.

\pauta{Soluções alternativas - Intérpretes}

\itemize
\item{Transcrever as aulas:
Se dá a ideia de gravar as aulas e colocar no Youtube para podermos ter as legendas automáticas. Porém, não teríamos certeza se as legendas estariam corretas. Poderíamos também legendar os vídeos, mas isso seria muito longo/gastaria muito tempo. Contratar algum interprete para traduzir seria caro.}

\item{Cauê comenta das bolsas de acessibilidade, ou chamar pessoas graduandas de libras. Porém, essas pessoas não necessariamente tem a competência necessária e é também muito caro.}

É comentado que a solução mais viável seria conseguir mais interpretes, pensando também em outros alunos surdos que podem entrar futuramente na universidade.

Guilherme diz que uma solução temporária pode resultar em mais atraso pela UFSC, pois eles veriam que o problema está sendo "resolvido".

Guilherme diz que um grande problema é que ele não sabe o quanto um professor pode cobrar de algum assunto, não sabendo quando parar de estudar (já que ele estuda por livros/vídeos). É dito que os colegas precisam o ajudar a selecionar o conteúdo, tendo do apoio de calouros ou veterano, estagiários de docência e monitores. 

Encaminhamenos: Luis, Patrick e Tiz fica encarregados de juntar pessoas para ajudar o Guilherme. 


\presentes{Paloma Cione, Luis Oswaldo, Mikael Saraiva, Cauê Baasch De Souza, Leonardo Kazuyoshi, Lucas Cavalcante de Sousa, Patrick Machado}

\end{document}
