\documentclass{ata-calico}
\usepackage{indentfirst}
\pagenumbering{arabic}
\begin{document}

\maketitle

\pauta{Informes/Repasses}
Cauê diz que ocorreu a primeira reunião do colegiado com o novo coordenador, Fletes. Está sendo revisto o regimento do colegiado e a forma de eleição do coordenador, o que é algo que podemos discutir. As possibilidade são eleição direta, onde os estudantes tem 30\% do peso, ou indireta, apenas no colegiado, onde possuímos representantes.

Durante a reunião do colegiado de centro foi debatido a segurança no campus, foram expostas as intenções de cercar o campus de maneira mais efetiva e limitar o número de entradas, e implementação de reconhecimento facial nas câmeras.

Foi criado um grupo para aproximar o movimento estudantil no CTC, visto acontecimentos recentes. Eles terão uma reunião hoje, às 18:00, no CETEC. Eles também querem continuar o projeto "UFSC na praça", para expor para a comunidade projetos e pesquisas, acontecerá uma reunião amanhã na APG ao meio-dia.

A UNE convocou mais um dia de paralisação na próxima quinta-feira, o DCE já confirmou assembleia para terça-feira.

Patrick comenta do projeto do CALICO com o CAME para ajudar uma casa de idosos na palhoça, a intenção é arrecadar 1000 reais para infraestrutura como colchões e também acompanhamento médico. Qualquer valor é bem recebido. Patrick está aceitando doações em dinheiro, e contas para transferência estão no evento no Facebook.

\pauta{Regimento CTC}
A proposta foi enviada para os estudantes, com pontos importantes como número de representantes discentes.
\bigbreak
É encaminhado que será ponto de pauta para próximas reuniões, para que as pessoas possam ler e para que possamos discutir a proposta.

\pauta{Paralisação dia 30}
Foi feita convocação para uma nova paralisação no dia 30, mas outros CAs concordam que é inviável realizar uma nova assembleia. O Mobiliza CTC está planejando atividades para esse dia.

É questionado o motivo de ser inviável uma nova assembleia. É respondido que uma nova assembleia teria um quorum menor, além disso outras entidades também tem em mente que no dia 14 de junho ocorrerá uma greve nacional, para qual precisaremos de uma nova assembleia. É desgastante para os alunos realizar assembleias de maneira consecutiva, o problema não está na paralisação e sim no excesso de assembleias.

Também é comentado a possibilidade de fazermos parte na "UFSC na praça", que é aberto para que a graduação participe.
\bigbreak
É concordado que quem quiser pode se juntar às atividades, as quais ajudaremos a elaborar, mas que não ocorrerá nova assembleia para realizar paralisação. Além disso, será divulgado o UFSC na praça para que estudantes e laboratórios possam ter conhecimento e decidir se gostariam de participar da iniciativa.

\pauta{Processo de eleição para coordenador}
Foi discutido duas possibilidades, eleição de maneira direta, onde todos os alunos votariam, ou apenas no colegiado, onde o voto discente seria o voto dos representantes que lá estão. É discutido qual seria o peso do voto discente se fosse através de representação uma vez que temos menos de 30\% de representatividade no colegiado, peso que seria caso fosse feita de maneira direta.

Cauê comenta como eleição direta pode ser problemático uma vez que os alunos votariam de acordo com motivos pessoais e com uma visão enviesada da relação aluno-professor, e que se feita discussão aberta com os estudantes a eleição indireta não seria problemática. É questionado como seria feito os votos em caso de divergências na discussão. Gava diz que apoia voto direto, que também atrairia mais pessoas para votação, podemos fomentar mais discussão mas o voto direto é mais representativo, é também comentado como se mais alunos votarem mais os candidatos se importarão com propostas que apoiem os estudantes. Luis comenta que pode ser problemático que pessoas que não participarem de discussões e que não tem certeza sobre o papel do coordenador possam votar por "birra", Gava responde que obrigar uma discussão seria apenas mais uma pauta do CALICO e sem o devido impacto de que os estudantes tem responsabilidade no processo de eleição. Will propõe que seja proposto um período eleitoral maior para que o CALICO possa promover discussões. Patrick comenta como estamos sujeitos a não ter representação, Luis responde que se alguém tiver interesse em ter representação e CA morra essas pessoas podem correr atrás.
\bigbreak
É votado a preferência por voto direto e indireto, considerando que as duas possuem o mesmo peso, onde foi decidido voto direto com ampla maioria. Caso os pesos sejam diferentes a pauta será rediscutida. 

\pauta{Utilização da sede do CALICO}
Cauê comenta que pegou na gráfica panfletos sobre a paralisação de quarta passada e deixou no CALICO para distribuição, e quando voltou lá estavam no lixo. Gava comenta como isso é um problema recorrente de gestões passadas, uma vez que o CA deve ficar aberto o maior tempo possível. Patrick comenta como ocorre em outros CAs, que membros da gestão fazem turnos para permanecer na sede. Leo comenta como opção ter um armário ou algo tipo para que a gestão guarde coisas mais importantes e que seria bom a gestão ocupar mais o espaço. É comentado que antigamente, quando era através de chaves, a gestão possuía chaves e uma com a atlética, não temos mais chave mas com a fechadura atual é possível substituir senha pela identificação via carteirinha da UFSC. Também é levantada a possibilidade de colar plaquinhas no espaço físico para incentivar boa conduta.
\bigbreak
A senha da porta será substituída pelo esquema de carteirinhas, serão feitos cartazes e a gestão discutirá a viabilidade da compra do armário.

\presentes{William Kraemer, Gustavo Gava, Leonardo Oliveira, Patrick, Cauê Baasch, João Maia, Hans, Lucas Souza, Mikael Saravia, Luis Oswaldo, João Trombeta, Leonardo Kazuyoshi, André Fernandes, Bernardo Farias, Gabriel Holstein}
\end{document}

