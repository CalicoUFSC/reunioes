\documentclass{ata-calico}
\usepackage{indentfirst}
\pagenumbering{arabic}
\begin{document}

\maketitle

\pauta{Informes/Repasses}
Para semana que vem, o mobiliza CTC está planejando duas mesas redondas sobre precarização do trabalho e reforma da previdência. Durante a semana passarão mais informações.

Pessoas do CTC estão reclamando do Collecta. A ideia é fazer um novo site, puxado pelo CASIN e CALEQA. É preciso de mão de obra, interessados podem contatar o Patrick.

\pauta{Médias que prejudicam alunos}
É levantado que são aprovados planos de ensino que prejudicam os alunos, mencionando a média de ES1, SO1 e Modelagem e Simulação. É questionado o sentido pedagógico de dar um peso grande para uma única nota ruim a ponto de inclusive acarretar reprovação dada uma única nota.

Caio explica que o plano de ensino é aprovado no colegiado do departamento e não de curso, e que geralmente são aprovados todos ao mesmo tempo e não individualmente. Sugere-se que os planos de ensino sejam disponíveis para alunos para que possam ser questionados antes da sua aprovação.

Caio diz que devemos conversar com coordenador, inclusive para adicionar como item de pauta para reunião do departamento. Um aluno da matéria disse que conversou um pouco com Fletes e que ele está ciente e é contrário a essa situação. Luis diz ainda que é preciso ressaltar que é necessário questionar a existência de objetivos pedagógicos por trás dessas médias. É dito que a justificativa dele foi simplesmente obrigar os alunos a ter bom desempenho em todos os trabalhos.

É comentado como não apenas a média foi problemática como também a maneira como a prova foi elaborada e avaliada, grande parte do conteúdo abordado foi de disciplinas anteriores e não houve nenhum retorno da avaliação além de certo ou errado uma vez que foi feita pelo Moodle. Nenhuma nota foi acima de acima da média, sendo a maior delas um único 5,6 .

É dito ainda como Cancian parece ter problemas pessoais com o professor de estatística, uma vez que nunca se refere a ele pelo nome e inclusive sugeriu que os alunos deveriam iniciar um processo contra ele. Além disso, a postura dele em sala amedronta os alunos o e causa inclusive problemas psicológicos.O problema não é a dificuldade da matéria e sim a maneira desproporcional como ele se coloca em sala de aula, chegando a ser desrespeitoso.

É dada a ideia de que todos os alunos deveriam pedir revisão da prova para incomodar o departamento, mas existe um medo de que o professor acabe descontando isso nos estudantes.\newline

\textbf{Encaminhamentos:} Será feito um ofício com abaixo assinado em anexo que será encaminhado primeiramente ao colegiado de curso. Também será questionada a maneira como são aprovados os planos de ensino no colegiado de departamento. É sugerido que os alunos da disciplina discutam e peçam juntos revisão da prova. Também é sugerido que os alunos compilem relatos de comportamentos problemáticos do professor e que isso seja encaminhado para o Calico para que possamos iniciar um processo na ouvidoria. O Calico também conversará com o coordenador do curso sobre a situação.

\pauta{Política de segurança no campus}
A SSI, que substitui a DESEG, tem como objetivo fechar o campus, deixando apenas as entradas principais abertas (rótulas da carvoeira, pantanal e trindade), colocar guaritas nesses pontos e implementar câmeras com reconhecimento facial.

Cauê diz que isso é um ataque à liberdade de tomar proveito dos diferentes papéis que o campus pode ter na vida das pessoas, ações como essa abrem precedentes para a UFSC fechar a noite e nos finais de semana, por exemplo. Luis menciona que como acabar festas diminuiu assaltos (pois não há ningúem na ufsc) e eles glorificam isso, mas que impedir que pessoas usem o local não é uma boa maneira de garantir segurança. Caio também comenta a postura autoritária e agressiva com a qual ele já foi abordado pela DESEG, Patrick comenta que já passou por situações similares.

É concordado que a segurança é fraca, mas a maneira como as medidas novas de segurança serão implementadas é problemática.\newline

\textbf{Encaminhamentos:} Levar a situação para outros CAs, CETEC e DCE.

\presentes {Patrick Machado, Cauê Baasch, Hans, Mikael Saravia, Luis Oswaldo, João Trombeta, Gustavo Tarciso, Arthur Pickius, Matheus de Brito Silveira, Matheus Silva P. Bittencourt, João V. Porto, Alexandre Behling, Mathias Riolen, Jacyara Bosse, Renato Souza, Paola de Oliveira Abel, Juliana Pinheiro, Augusto Zwite, João Paulo Tiz}
\end{document}
