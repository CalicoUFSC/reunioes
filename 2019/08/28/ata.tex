\documentclass{ata-calico}
\usepackage{indentfirst}
\pagenumbering{arabic}
\begin{document}

\maketitle

\pauta{Informes e Repasses}
Hoje às 18:30h:, Aula sobre o future-se - MobilizaCTC/DCE, no EPS.

Hoje às 14h: Apresentação future-se por De Pieri e Peters, EEL004.

Hoje às 17h: Reunião do DCE, na sede.\newline

\textbf{Repasses Colegiado:} \begin{itemize}

\item Fizemos uma força-tarefa para ler todos os planos de ensino do curso, alguns planos tinham problemas e levamos para a pauta. 

\item Professores estão se demonstrando contra o uso de celulares e computadores em sala de aula, falaram muito sobre e todos concordam sobre ser falta de respeito, etc. 

\item Houve 3 casos de prorrogação de curso curricular, todos aprovados, um deles uma garota com problemas psicológicos, que parou o curso um tempo e voltou com com acompanhamento, foi recomendado que ela refizesse o vestibular, apesar de terem aprovado o pedido.  

\item Foi colocado em votação o regime do curso, mas ainda não aprovado pois é preciso discutir a eleição de coordenador do curso.

\item Muitos alunos faziam disciplinas de fora do curso, que não contam como optativas e sim como extracurriculares, mas a ementa da disciplina se relacionava muito com o curso de computação. A discussão pretende definir quais optativas são extracurriculares ou não. Foi aprovado que será feita uma revisão sobre estas disciplinas, não olhando apenas o código do departamento e sim a ementa, para podermos considerá-las como optativas do curso.

\item Das formas de ingresso, se um aluno conseguir validar pelo menos 4 das 5 matérias iniciais, esta vaga será colocada em outra partição, tendo uma vaga a mais disponível para vestibular/ENEM.

\item Será colocado mais um requisito para matéria de Mod\&Sim, calculo numérico

\item Estatística será realocada para o quinto semestre, banco de dados 1 para o sexto, banco de dados 2 continua no sétimo. 

\item Diretiva do MEC mostra que nós não cumprimos uma carga mínima necessária. Será criada uma nova disciplina de atividades complementares, com +- 172h/a, sendo discutido mais a frente quais serão os pontos abordados e o que será ministrado nela. 
\end{itemize}

\textbf{Repasse reunião APUFSC, DCE, SINTUFSC e Reitoria:}

\begin{itemize}
\item Reitoria demonstra que não tem interesse em cancelar o semestre, fazendo o que for possível para terminar o período letivo. há confirmação de que a SEPEX foi cancelada, apesar de controvérsias.

\item Bolsas de monitoria congeladas (se não tivermos monitor de certa disciplina x até agora, não haverá mais). 

\item Supressão da vaga de terceirizados. 

\item Ar condicionado só em lugares extremamente necessários.

\item Outras formas que estão sendo estudadas: apenas estudantes isentos com acesso ao RU, podendo cobrar o preço de custo (R\$10,50) para os outros 11,5mil. 

\item SINTUFSC pediu reunião aberta para reitoria explicar os cortes e esclarecer as questões sobre o RU, terceirizados e afim, que acontecerá amanhã 20/09 às 13:30, no auditório do CCE.

\end{itemize}
\pauta{Curricularização da extensão}
Está sendo discutido no conselho de centro uma diretiva do MEC que diz que cada curso deve ter 10\% dedicados a extensão.

Professores do CTC são contra, pois não querem mudar o currículo do curso. Pretendem ignorar isso e torcer para que as pessoas esqueçam. Alguns dizem que isso não pode ser feito pois já há muitas reprovações, ou que o currículo já é bom o suficiente, etc. Isso está em pauta agora e precisamos discutir isso. 

A ideia de extensão no CTC é distorcida, e deveríamos dicutir no curso o que é realmente a extensão universitária.

Cauê diz que CTC realmente não está pronto para isso, pois estamos longe da sociedade há muito tempo. Professores irão tentar colocar como extensão coisas que não são extensão (atletica, EJs, laboratorios), devemos aproximar nosso centro da sociedade e então conseguir fazer a extensão propriamente dita.

Luis complementa dizendo que é muito dificil ver como as engenharias/ctc poderiam fazer extensão pois estamos muito longe das comunidades ao nosso redor, e é dificl ate mesmo pensar em algum projeto, já que é um conceito totalmente diferente.\newline

\textbf{Encaminhamentos:} Levar esse assunto para uma roda de conversa da SECCOM. Fazer uma nota para o conselho de centro, explicando a importância, fazendo a discussão acontecer.


\pauta{Regimento de curso}
Foi pedido que elaboremos uma proposta para a eleição de coordenador de curso, podendo ser uma eleição direta ou indireta.

Primeira proposta, eleição direta: Cada estudante poder votar individualmente nos candidatos, representando 30\% dos volto, sendo os outros 70\% o voto de professores elegíveis.

Segunda proposta, eleição indireta: Cada representando de colegiado (2) teria um voto, tendo pesos iguais entre professores/alunos, porém, dentro do colegiado, a representação estudantil é de apenas de 15\%.\newline

\textbf{Encaminhamentos:} Ficou decidido levar a primeira proposta ao colegiado, por termos mais peso. Cauê vai investigar se é possível aumentar a representação discente no colegiado.


\presentes {Cauê Baasch, Luis Oswaldo, João Trombeta, Paloma Cione, Hans, Lucas Coquinho, Thales, Julien, João Paulo Tiz, Arthur Pickius, Mikael Saraiva, Matheus}
\end{document}
