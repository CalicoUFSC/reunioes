\documentclass{ata-calico}
\usepackage{indentfirst}
\pagenumbering{arabic}
\begin{document}

\maketitle

\pauta{Informes/Repasses}
\begin{itemize}
\item Alguns membros do DCE Luis Travassos foram intimados a depor na polícia federal sobre o dia 14 de junho.
\item A resolução n.132 foi aprovada no Conselho Universitário, junto do novo calendário acadêmico de 2019. Isso estendeu o período de cancelamento de matrícula em matérias até o dia 15 desse mês.

\end{itemize}
\pauta{Reclamações da Turma do prof.Alex de POOII}
Foram levantados, pelos alunos, os seguintes pontos:
\begin{itemize}
    \item O professor normalmente atrasa o começo da aula, falando que está esperando mais alunos chegarem.
    \item A parte do conteúdo das aulas normalmente dura 40 minutos, seguida de um período de resolução de listas de exercícios. É comum o professor sair da sala, citando um intervalo, e ficar em torno de 40 minutos fora de sala.
    \item As listas de exercícios (não foram passadas muitas até agora) em sua maioria explicitam no enunciado qual a linguagem a ser desenvolvido. Em sua maioria a linguagem é java, são poucos os exercícios que devem ser resolvidos em python.
    \item Ele não sabe tirar dúvidas de alunos e parece não conhecer a linguagem (python). Nos slides, parece que apenas traduziu códigos em java para python.
    \item A correção dos trabalhos é baseada apenas no código rodando, as vezes com perguntas para os alunos, sem ver o código. O trabalho não é inteiramente visto. Os critérios de avaliação não são explicitados para os alunos.
    \item Já teve casos de, em um trabalho, os alunos perguntarem os requisitos em duas aulas diferentes. Os requisitos ditos foram diferentes nas duas aulas, indicando que ele não tem isso definido.
    \item Alguns trabalhos foram passados em uma aula e nunca mais citados, como se não realmente existissem, e não foram cobrados.
    \item O plano de ensino não está sendo seguido em sua totalidade. Alguns conteúdos ele não passou, outros passou em menos tempo do que o definido pelo plano de ensino. Foi citado que ele não passará mais conteúdos, então os ainda não passados do plano de ensino não serão passados.
    \item Alguns conteúdos foram passados errados (exemplo: aula de json).
    \item No período das aulas em que ele está na sala, mas não passando matéria, fica conversando com alguns alunos, muitas vezes sobre trading.
    \item Na P1, foi cobrado interface gráfica, porém só uma aula sobre o assunto foi dada.
    \item Existe receio em tirar dúvidas com o professor. Em algumas ocasiões, alguns alunos perguntaram sobre o que cairia na prova. Enquanto para uns ele falou que seria todo o conteúdo passado até o momento, para outros ele explicitou em pontos definidos o que cairia.
    \item As aulas são leitura de slides, basicamente. Existem alguns erros, tanto na explicação dele quanto nos slides. Não parece conhecer muito de python.
    \item Na P2, chegou atrasado para o começo da prova, nas duas turmas. Em uma das turmas, não teve computadores suficientes para a turma toda, então parte dos alunos teve que ficar esperando até ele conseguir outra sala para que fizessem a prova.
    \item A P2 foi inteiramente sobre listas e dicionários, conteúdo teoricamente cobrado na P2 de POOI. Foram 4 questões de fácil resolução, que poderiam ser escritas em 2~3 linhas, com geração de números aleatórios e print de alguns desses números.
    \item Tiveram trabalhos com temas fora de assuntos que os alunos deveriam conhecer (assuntos que não são passados nem na matéria nem em nenhum outro lugar do curso). Não existiu uma explicação básica ou sequer um material de apoio para que os alunos pudessem fazer o trabalho focando no aspecto de POOII. (Mais especificamente, os trabalhos de ações e de roleta.)
\end{itemize}
\presentes {André F.S. Fernandes, Arthur Mesquita Pickcius, Bernardo S. Farias, Gabriel Holstein Meireles, Hans Buss Heielemann, Helena Kunz Aires, João Vitor Maia, José Luiz de Souza, Lucas Cavalcante de Sousa, Matheus D.C. Roque, Mikael Saraiva, Paloma Cione}

\end{document}
