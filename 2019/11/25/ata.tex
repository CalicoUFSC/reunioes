\documentclass{ata-calico}
\usepackage{indentfirst}
\pagenumbering{arabic}
\begin{document}

\maketitle

\pauta{Informes/Repasses}
\begin{itemize}
\item A ata de posse ainda não foi lançada.
\item Só teve uma resposta no questionário sobre os votos nulos no moodle.
\item Vai ter CEB dia 26/11 às 12h para discutir o regimento eleitoral do DCE.
\item DCE publicou uma nota contra a criminalização das entidades, CALICO assinou.
\item Vai ter Amnésia dia 30.
\item Alguém precisa puxar a organização da SECCOM, que já ta acabando o semestre.
\item Precisamos ver pessoas para ficarem responsáveis pelo Trote Integrado, duas pessoas possivelmente.
\item Está acontecendo vacinação contra sarampo em frente à reitoria, vai do dia 25 ao dia 29 às 21h.
\item Precisamos marcar a limpeza semestral da sede.
\item Vai ter Bento dia 05/12 às 18h30 no container da carvoeira.
\end{itemize}

\pauta{Reabertura da varanda do CETEC}
\begin{itemize}
\item A varanda tem que ser reaberta, mas é necessário melhorar o estado dela, deixar ela mais amigável.
\item Na ata do CETEC, foi escrito que iam trancar a porta para reduzir o fluxo de pessoas. Não sabemos quem tem a chave, mas a principio são poucas pessoas. Faria mais sentido ter pelo menos um representante de cada centro acadêmico com uma cópia da chave.
\item O CETEC não pode tomar uma decisão tão impactante quanto essa sem consultar as bases antes. É válido questionarmos a validade dessa decisão. A justificativa deles, de ficar sujo o lugar, não faz sentido. Temos que mudar a cultura do espaço, não restringir o acesso.
\item É preciso pensar em pontos para serem levantados na reunião do CETEC. Pensar em formas de retomar o espaço, mudar a cultura do local. Isso vai além da varanda, mas inicialmente seria bom pensar nesse espaço em específico.
\item Qualquer coisa que colocarmos ali tem o risco de ser depredado. Parte da revitalização passa por tornar o espaço agradável. Temos que ver como puxar essa revitalização. É mais limpar e garantir que a galera utilize mais espaço, além de conversar com qualquer pessoa que esteja fumando ou sujando o lugar.
\item Na reunião com o bar do CETEC, eles também mostraram interesse em revitalizar. Poderia ser interessante fazer uma parceria com eles.
\item Um jeito de garantir que as pessoas cuidem mais do espaço é deixar a luz acesa. Se deixar limpo e convidativo, vão acabar cuidando mais.
\item Talvez se utilizarmos mais o espaço, fazer as reuniões lá, acabe ajudando. Também é possível colocar cadeiras e tomadas para que seja mais convidativo.
\item Temos que falar com os outros centros acadêmicos para que entendam que a responsabilidade também é deles de ocupar o espaço. Talvez sugerir que as reuniões do CETEC sejam lá, antes ou depois da revitalização.
\item Precisamos pensar em outro espaço para as pessoas fumarem, propôr outro lugar que usa atualmente a varanda e entrar em um consenso com eles.
\item Podemos nos comprometer como entidade a lidar com o pessoal. O resto do CETEC não se mostra aberto para dialogar com as pessoas. O problema é que nem sempre estamos disponíveis. Um jeito é nos mostrarmos dispostos, mas não necessariamente nos comprometermos.
\item Quanto à limpeza, podemos colocar lixeiras lá. O chão fica sujo não só por causa de bebida, mas também da chuva. Não tem como não sujar o chão quando o espaço é usado.
\end{itemize}

\pauta{Reunião do CETEC}
\begin{itemize}
\item As atas devem ser publicadas para todos, sem demora. Que não sejam discutidas questões que não estão em pauta, ou pelo menos não deliberar. Sobre a reunião em si, pedir que seja em círculo e bem coordenada. Pedir para respeitarem ordem de fala.
\end{itemize}

\presentes {Arthur Pickius, Lucas Sousa, Luis Oswaldo Ganoza, Helena Aires, Cauê Baasch, Mikael Saraiva, Paloma Cione}

\end{document}
