\documentclass{ata-calico}
\usepackage{indentfirst}
\pagenumbering{arabic}
\begin{document}

\maketitle

\pauta{Informes/Repasses}
\begin{itemize}
\item Existe conflito de horário em duas matérias da terceira fase: INE5409 (Cálculo Numérico para Computação) e INE5410 (Programação Concorrente). Já foi avisado na secretaria o problema.
\item Há algumas semanas, foi pedido a inclusão de Libras I e II como optativas do curso. O pedido ainda está parado no NDE, não sabemos em que estágio, e não temos acesso às atas deles.
\item Dia 05/12 aconteceu o Bento, em torno de 30 pessoas apareceram no inicio do evento.
\item A Amnésia teve em torno de 1600 pessoas. Foi uma edição tranquila, sem assaltos ou brigas.
\item O Planejamento Estratégico ficou para dia 11/12, das 18h até 24h.
\end{itemize}

\pauta{Compra da Amnésia}
\begin{itemize}
\item A Fenda, produtor de eventos que organiza a festa do curso, quer vender parte dos direitos sobre a festa para as outras entidades que já a compõe, reduzindo sua participação e ganhos. A ideia é que a Fenda não continue mais com a parte majoritária da festa, para que as decisões não  precisem necessariamente passar por ela.
\item Existem duas propostas: a Fenda ficar com 100\% do lucro desse semestre ou 80\% nesse e 40\% no próximo. Depois de pago, a empresa ficaria com 20\% do lucro da festa.
\item Foi levantado a questão de talvez usar um valor fixo em vez de porcentagem do lucro. Os outros envolvidos na compra não se sentem confortáveis com a proposta de valor fixo, por isso foi descartada a ideia.
\item Precisaria de mais pessoas e mais trabalho depois da compra, 3 ou 4 pessoas do CALICO apenas para trabalhar na comissão organizadora. Seria um aumento considerável de trabalho.
\item Em questão financeira, a proposta melhor para o centro acadêmico é a de 80\%, pois assim teríamos algum lucro esse semestre ainda.
\item A Fenda ainda iria bancar o investimento inicial, com o centro acadêmico dando entre 1500 à 2000 reais para ajudar na festa. Mesmo que não fosse continuar assim, depois de algumas edições seria mais fácil bancar o investimento inicial, já que estaríamos ganhando uma parte maior dos lucros.
\item A festa tem uma parceria com o local, então o pagamento se dá por uma porcentagem do lucro, não por aluguel fixo. A festa tem alguma ajuda do lugar no dia, também, caso falte alguma bebida ou com infraestrutura necessária não inclusa no contrato. Caso a festa não dê lucro, o local é pago 4000 reais como forma de seguro.
\item O que é feito atualmente é um contrato fixo, renovado anualmente e assinado por um membro da diretoria do CALICO. Que redige o contrato é a Fenda.
\item A A5 (atlética de computação e sistemas, que também compõe a festa), já aceitou a proposta.
\end{itemize}

\textbf{Encaminhamento:} O CALICO aceitará a proposta de 80\% nesse semestre e 40\% no próximo para a Fenda.

\presentes {Fábio Coelho, Patrick Machado, Cauê Baasch, Helena Aires, Arthur Pickcius, Lucas Sousa, Ricardo Jensen, Paloma Cione}

\end{document}
