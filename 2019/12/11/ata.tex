\documentclass{ata-calico}
\usepackage{indentfirst}
\pagenumbering{arabic}
\begin{document}

\maketitle

\pauta{Informes/Repasses}
\begin{itemize}
\item Foi aprovada no colegiado do departamento uma nova resolução de espaço físico.
\item Estava tendo discussão sore o regimento interno do departamento, mas foi adiado porque estava dando muito problema. Talvez a discussão seja retomada antes do início do semestre letivo.
\item Início das aulas está para 04/03/2020.
\end{itemize}

\pauta{Limpeza e Reforma da Sede}
\begin{itemize}
\item Existe muito lixo na sede do CALICO e o espaço está muito suo, é preciso ver o que fazer com tudo e organizar o que vai ficar. Até mesmo pensar em políticas de uso do espaço.
\item Quanto à limpeza, precisamos pensar em uma data para a faxina, possivelmente no próximo semestre. Precisamos também pensar em algumas medidas para ajudar a manter o espaço limpo. Talvez também pintar a parede, se desfazer dos livros e trocar alguns móveis, porque está tudo quebrando. Também precisa de uma rede nova.
\item Poderíamos fazer um inventário das coisas na faxina. Também é interessante comprar cadeiras novas.
\item Pensar em trocar os sofás, colocando coisas que ocupem menos espaço. Para se desfazer de qualquer coisa, precisa ser decidido em reunião aberta.
\item Tirar uma data para a limpeza e, nessa data, dar uma olhada em tudo e ver o que se desfazer já para se organizar para levar à reunião aberta. Tirar também todos os móveis e fazer a pintura da parede. Escolher alguém para fazer uma pesquisa de preços para móveis novos.
\item Talvez fazer grupos de limpeza da diretoria para ter limpezas periódicas do CALICO e não sobrecarregar uma pessoa só.
\item É difícil manter um inventário das coisas da sede, justamente pelo caráter aberto do espaço.
\item Poderíamos já listar agora as coisas que iríamos comprar, delimitar uma faixa de preço e ai a comissão que ficaria responsável pela pesquisa parte daí. Precisaríamos de um armário, assentos e uma rede.
\item Seria bom evitar ao máximo compensado.
\item Marcar uma reunião para definir exatamente os itens que vão ser comprados, lugar e tudo mais.
\item Se a reunião for feita online, alguém precisa estar no CALICO para conferir as medidas certinho.
\item Precisa de um tempo entre a reunião e a limpeza, pra dar tempo dos móveis chegarem.
\end{itemize}

\textbf{Encaminhamento:} Na semana do dia 03 de fevereiro vai ter reunião online pra decidir a compra dos móveis.

\textbf{Encaminhamento:} Na semana do dia 17 de fevereiro vai ser a limpeza.

\textbf{Encaminhamento:} Comissão de procura de preço: Fábio e Roque.

\pauta{Calourada 2020.1}
\begin{itemize}
\item Calourada é a semana destinada ao centro acadêmico interagir com os calouros, normalmente a primeira do semestre. Isso varia bastante de centro para centro. Até uns anos atrás, no curso tinha a linguicinha, o emplacamento e o trote sujo. No último ano teve algumas mudanças. O centro acadêmico não só parou de se responsabilizar pelo emplacamento como colocou outro evento. Essa disputa que aconteceu foi prejudicial aos calouros.
\item O semestre começa dia 04 de março, quarta-feira, com a prova de pré-cálculo sendo dia 28/02.
\item Proposta de bar no dia 28 para depois da prova de pré-cálculo, trilha dia 02 ou 03 (logo antes de começar as aulas).
\item Tentar um espaço na aula de Introdução para apresentar o centro acadêmico.
\item Fazer a linguicinha no primeiro dia, 04, para fazer afronte ao emplacamento. Isso pode ser problemático, tentar primeiro um diálogo, até porque o grupo que está puxando dessa vez é mais progressista. Tivemos já duas experiências:
  só deixar rolar, que não funcionou, e entrar em embate direto, que também não
  funcionou. Podemos falar com o grupo que está puxando agora para fazer alguma
  coisa legal pode ser um começo de mudança da cultura do curso.
 \item Só conversar com eles pode não ser benéfico, porque pode não mudar a postura
  e cultura do curso.
\item Esse semestre que vem provavelmente não vai ter veterano problemático, inclusive o
CALICO  momento dos calouros criarem vínculos. A cultura também começa de algum
  lugar, deixar isso pode ser interessante pra começar a mudar isso. Também é
  responsabilidade do centro acadêmico de acompanhar essas iniciativas e
  aproveitar as oportunidades que nos são dadas.
\item O perfil de entrada e a cultura vigente que ditam como vão ser os próximos
  semestres. A possibilidade de depois voltar a cultura vigente é bem real,
  precisamos cortar o problema pela raiz.
\item Esse é o momento de pegar esse grupo de pessoas pra tentar acompanhar, caso
  depois volte a gente intervem como necessário.
\item A curto prazo, a gente se opor e fazer propostas alternativas é prejudicial.
  A gente cria uma divisão no curso que, embora tenha pontos positivos, não é
  interessante mostrar aos calouros na primeira semana, isso afasta eles do
  centro acadêmico. Talvez seja melhor pensar nas coisas a curto e médio prazo,
  não temos muito como pensar direito a longo prazo.
\item Não podemos endossar ou ajudar no trote, como centro acadêmico.
\item Precisamos agora mesmo durante as férias falar com a professora de introdução
  para ter a primeira aula inteira, exclusividade do centro acadêmico. É
  desrespeitoso usar o mesmo espaço para apresentar todas as entidades.
  Distribuir o manual do calouro, talvez camisetas se tiver. Talvez falar do
  DCE e dos centros acadêmicos, da eleição do DCE que vai acontecer no começo
  do semestre que vem.
\item Também talvez explicar coisas do curso, a experiência real do curso, alguma
  atividade de apresentação e introdução. Dar um abarcado geral no manual,
  também.
\item Pensar em quando fazer noite de jogos, Chicco e talvez uma primeira reunião
  do CALICO na primeira semana.
\item É importante não criar mais eventos, nesse momento, mas fortalecer o que já tem.
\item Talvez fazer um tour pela UFSC, talvez como uma parte prática da apresentação
  em Introdução. Fazer reunião aberta do CALICO na primeira semana, talvez.
\item Pode ser problemático o centro acadêmico ter só um evento e os veteranos terem dois. Uma
  ideia é estarmos nos apropriando de algumas das ideias deles. Talvez falar
  para eles jogarem para a segunda semana os eventos deles e dominar a
  primeira. Ai eles podem manter a programação original, mas só com um shift.
  Ai poderíamos estar presentes na primeira semana inteira, também.
\item Os eventos de recepção e integração são de responsabilidade do CALICO, não
  dos veteranos.
\item Ter abertura para o diálogo ajuda a criar uma narrativa a nosso favor ao
  longo do semestre caso eles decidam manter as datas deles.  Manter os eventos
  quarta, quinta e sexta para que o CALICO possa apresentar todas suas áreas de
  atuação.
\item Os veteranos vão aceitar mudar as coisas de boas. Avisar com antecedência garante que eles possam alterar as datas.
\item A ideia do tour é passar por diversas áreas do campus, incluindo CFH, bosque,
  DCE, reitoria.
\item Fazer o manual do calouro. Melhorar ou começar do zero. Tirar responsáveis para a melhoria do manual e procurar alternativas para a
  impressão.
\item Talvez fazer o Chicco temático, já que cai em uma sexta-feira 13.
\end{itemize}

\textbf{Encaminhamento:} Manual do calouro: Paloma.

\textbf{Encaminhamento:} Definir uma data para a Noite de Jogos.

\textbf{Encaminhamento:} Calendário de eventos da Calourada:
\begin{center}
\begin{tabular}{|c|c|c|c|c|c|} \hline
     Fevereiro & \multicolumn{5}{|c|}{Março} \\ \hline
     28 & 02-03 & 04 & 05 & 06 & 13 \\ \hline
     Bar pós pré-cálculo & Trilha & Linguicinha & Tour pela UFSC & Reunião & Chicco \\ \hline
\end{tabular}
\end{center}

\pauta{Eventos de Integração}
\begin{itemize}
\item O CA é responsável pela integração dos alunos do curso.
\item Ponto foi proposta porque foi proposto garantir uma continuação da integração
  ao longo do semestre e não só na calourada.
\item O Bento e a noite de jogos são exemplos de eventos.
\item Noite de jogos deveria acontecer com mais frequência, mas isso faz com que
  perca o caráter de ser da calourada
\item Surge a proposta de que a noite de jogos seja feita uma vez por mês ou uma no começo e uma na
  metade pro final do ano.
\item Tirar um perfil das pessoas que estão participando dos eventos para
  direcionar os perfis. Não necessariamente os eventos de integração deveriam
  ser periódicos e sim de acordo com a procura. Fazer espaços de ensino como
  integração.
\item Surge a possibilidade de fazer uma enquete pra saber que tipo de evento pode
  ser feito.
\item Eventos de esportes podem ser feitos pelo CA, mesmo com a existência das
  atléticas.
\item Fazer mais eventos acadêmicos, talvez ajudando a SECCOM ou novos eventos.
\item A enquete sugerida não seria feita só para os calouros, mas para também
  pensar na grande maioria dos estudantes de computação que são a maioria no
  curso. Seria para ter novas ideias.
\item Com relação à eventos acadêmicos, existem diversas pessoas ligadas à isso, como PET e caravela.
\item É pontuado que talvez as pessoas não vão nos eventos por causa dos horários.
\item Reforçada a ideia de fazer eventos de esportes. Pra além disso fazer eventos
  de jogos sedentários, como sinuca, ping-pong, CS e mais.
\item Já existem jogos sedentários na computação, porém é responsabilidade da
  atlética.
\item Foi pontuada o problema de se fazer o eventos por responsabilidade da
  atlética.
\item Se a gente acha que é importante a existência desses eventos que não
  acontecem, deveríamos fazer.
\item A gente pode manter o diálogo, mas não é necessário que esperemos por eles
  nem peçamos a permissão.
\item Se a ideia é reviver o JOSE (jogos sedentários de CCO), que é da
  responsabilidade deles, seria melhor falar com eles parar garantir que não
  serão criados antagonismos.
\item É importante que sejamos "políticos" e que falemos com eles. Falar com eles:
  se eles forem fazer então que façam, se não quiserem, então nos fazemos.
\item É importante que o centro acadêmico não faça eventos que são responsabilidade da atlética
  para que não sejam confundidas as responsabilidades da atlética e do CALICO, porque
  é importante saber a diferença da associação esportiva e da entidade
  representativa.
\item Foi pontuado que seria bom fazer um evento que não fosse tão atrelado ao
  álcool.
\item O JOSE é bastante baseado no álcool e o que não é álcool já é feito na noite
  de jogos.
\item A ideia de jogos sedentáŕios é diferente da noite de jogos, por ser baseado
  em um campeonato, que possua mais competição.
\item A problemática de jogos competitivos e a necessidade de colocar pessoas umas
  contra as outras. Possibilidade de jogos onde todos competem entre si por um objetivo em comum
  (jogos cooperativos). Volta para eventos esportivos, talvez em conjunto com as pessoas do CDS.
\item Importante tirar datas fixas para eventos, nem que os eventos sejam tirados
  mais pra frente.
\item Chamar mais cursos para fazer parte dos eventos de integração.
\item Balancear os eventos de formação com integração
\item A ideia dos eventos quando eles surgiram era fazer regularmente.
\item Fazer eventos conforme surgirem as necessidades. Proposta de fazer mensal ou quinzenal.
\item Um mês é tempo demais e quinze dias é demais. Três semanas surge como
  possibilidade.
\item Fazer os eventos mensalmente e garantir eventos mais espontâneos.
\item Um mês passa rápido, porém o semestre só tem 4 meses.
\item Não ter periodicidade porque podemos negociar nosso calendário conforme a agenda dos calouros.
\end{itemize}

\textbf{Encaminhamento:} Fazer eventos mensalmente.

\textbf{Encaminhamento:} Contatar outros centros acadêmicos e organizar alguns eventos com mais cursos.

\textbf{Encaminhamento:} Fazer eventos esportivos.

\textbf{Encaminhamento:} Falar com a atlética para garantir que o JOSE aconteça.

\pauta{Campanhas e Espaços Formativos}
\begin{itemize}
\item Interessante ser feito com coisas factíveis, relacionadas à realidade e à
  atualidade, como o evento de debate sobre a eleição presidencial.
\item Trazer debates em questões polarizadas para dar espaço para que o diálogo
  aconteça.
\item Precisamos de eventos de formação política não só de atualidade, mas também
  de compreensão da sociedade. É importante ter coisas da atualidade, mas
  também ter algo progressivo para aqueles interessados.
\item É importante estarmos investindo mais em formação política, entendendo que a
  faculdade tem interesse de formar não só trabalhadores, mas também cidadãos e
  acadêmicos. Também entendendo que as matérias voltadas para isso não dão
  conta e que o ambiente do CTC tem um repúdio a isso, então o centro acadêmico
  precisa dar uma abertura para o assunto para os estudantes.
\item Três frentes: universidade popular e extensão universitária; flexibilização
  das relações trabalhistas, precarização do trabalho e leis trabalhistas num
  geral; soberania nacional e produção de ciência e tecnologia. Como formatos,
  poderia ser rodas de conversa, mesas redondas e curso.
\item Talvez fique um pouco pesado abranger tudo assim, poderíamos talvez esmiuçar
  os pontos para ficar entendível. Precisamos pensar o formato, se pegamos um
  dos pontos ou se pensamos nos três em paralelo.
\item Iríamos nos apropriar de textos já escritos, não elaborar as coisas.
\item O que tem no CTC não é uma ausência de política, mas sim um lado já. Não tem
  necessidade de ficar mediando, pois acaba deixando o lado que trazemos
  como vilões.
\item O problema em ir construindo algo juntos é que não é a nossa área, então a
  ideia seria estudar coisas já existentes, não estabelecer coisas do zero. Os
  formatos iriam estabelecer a forma de como fazer as coisas. Quando queremos
  falar de algo mais profundo que não temos muito conhecimento, precisamos de
  um elemento como um curso.
\item Debates não são necessariamente proveitosos pois um dos lados já está
  estabelecido, enquanto o outro fica marginalizado, mas talvez seja algo bom
  por questões estratégicas.
\item Debate funciona quando queremos introduzir a ideia de uma maneira um pouco
  mais leve.
\item A primeira coisa a se pensar com eventos para leitura de textos é a adesão.
  Queremos quebrar uma bolha, investindo em pessoas que estão "não tendo
  política". Fazer algo mais horizontal, mesmo que seja menos profundo, é muito
  mais produtivo quando queremos puxar pessoas que estão engatinhando
  politicamente.
\item Talvez puxar menos pessoas, mas sem prejudicar tanto o debate pode ser mais
  produtivo. Precisamos pensar qual o nosso objetivo, se é qualificar um grupo
  menor com mais embasamento ou se é atrair uma massa maior de estudantes com
  menos qualificação. Talvez a ideia seja realmente os dois.
\item Vendo principalmente a situação política do próximo ano, é importante
  massificar esse debate. Mas também é importante ter pessoas com um
  embasamento político forte para poder fomentar o debate.
\item Pensando em seguir as duas linhas, com que frequência faríamos esses eventos
  é algo a se pensar.
\item Poderíamos também estar reagindo melhor à conjuntura e fazendo eventos sobre
  assuntos que estão populares no momento. Talvez pensar em uma frequência
  menor para os eventos mais pesados em conteúdo.
\item Algo que talvez possa ser trazido mais facilmente de forma mais ampla é sobre
  leis trabalhistas, principalmente considerando a realidade do curso. É um
  assunto muito mais tangível, muito mais fácil de trazer para o curso.
\item Seria muito interessante conseguir fazer uma boa progressão entre os
  semestres. Primeiro ter formações mais básicas, no próximo semestre pegar
  coisas um pouco mais aprofundadas.
\item Fazer três eventos pontuais, um no começo e dois no meio do semestre. É muito
  importante se preparar pro semestre que vem, pra se arrumar conforme a
  conjuntura. É bom ter em mente que talvez acabe sendo preciso remanejar tudo
  completamente.
\item Fazer um sobre universidade popular, um sobre ciência e tecnologia e um sobre
  algum ponto trabalhista (talvez uberização do trabalho).
\item Ver o quanto conseguiremos envolver os calouros nos eventos. Talvez seja uma
  boa ter algo mais progressivo para engajar mais as pessoas.
\item Precisamos pensar em pontos que vão realmente atrair as pessoas.
\item Perigosa a lógica de precisar atrair as pessoas. Necessário despertar o
  interesse das pessoas, desenvolver como algo natural.
\item Espaços de integração devem ser usados como espaço de abertura para se
  conversar com as pessoas para despertar o interesse.  Precisamos definir qual
  o público alvo, acredito que nosso foco deva ser os calouros.
\item Se nosso foco é chamar os calouros, imagino que pelo menos no primeiro evento
  queremos atrair principalmente eles.
\item Pensei em usar o espaço da primeira aula de introdução para fazer uma mesa
  (30 min ~ 40 min) sobre a história do movimento estudantil.  Iniciar na
  produção de ciência e tecnologia e progredir pra universidade popular.
\item Gostei da ideia de pegar a primeira aula de introdução, mas não podemos nos
  perder, acho interessante falar sobre o CA e suas atribuições primeiramente e
  depois falar sobre as outras entidades e movimento estudantil. Gostei da
  progressão. Por último falar sobre a precarização do trabalho.  Temos que dar
  a cara para as pessoas do curso, mostrar quem constrói o CA. Concordo com a
  proposta (acima) e acho que podemos partir para os encaminhamentos. Jogar o
  tema de precarização para o semestre seguinte, fazer o curso CFS no primeiro
  semestre.  Durante reuniões ordinárias discutir pontos para qualificar a
  diretoria.
\end{itemize}

\pauta{Política Financeira}
\begin{itemize}
\item Defendo que continuemos fazendo os eventos com contribuição espontânea,
  pessoas tem que ter consciência que precisam contribuir com a entidade.
  Precisamos de mais pessoas para contribuir com a Amnésia. Precisaremos de
  mais estudantes da base para contribuir com o CALICO.  Vender picolé no RU.
  Venda de copos, camisetas e adesivos com a identidade da entidade. Talvez
  transformar um dos eventos de integração em um HH junto com o DCE.
\item Antes de política financeira devemos pensar sobre política. Sou contra festas
  universitárias, acho que são um problema, por não serem acessíveis à todas as
  pessoas. Mesmo que seja uma festa progressista pessoas passam por situações
  desconfortáveis. Acho que precisamos depender menos de renda de festas.
  Acredito que o melhor tipo de festa universitária é HH, por ser acessível,
  cada um paga o que consome, ambiente aberto.  Devemos pensar outras formas de
  ganhar dinheiro, talvez HH apesar do risco, por Florianópolis estar perigosa e etc.
\item Concordo que festa universitária é problemática. Acho q o HH aqui depende de
  conversar com outros cursos que já fazem HH, para trazer mais pessoas para um
  HH no CTC.
\item HH no cetec já tivemos lucro no passado e deu certo, mas tivemos que cobrar
  ingresso. Já um HH no DCE seria mais acessível.  Cetec tem limite de pessoas
  devido ao espaço restrito.
\item Muito trabalho arrecadar dinheiro com comida.
\item Pensar ao longo do semestre em HHs, seja no cetec, DCE ou algum outro curso.
  Acho que deixamos de lado questões com comida.  Com relação a venda de
  camisetas e etc, acredito que não seja política financeira, devemos fazer
  pelo menor preço possível e não para lucrar com os produtos.
\end{itemize}

\pauta{Programa da Chapa}
\begin{itemize}
\item Mural na entrada do INE. Notificações no elevador e na sede do CALICO.
  Deveríamos passar nas salas de aula fazer informes.  Nas primeiras reuniões
  definir representantes para os colegiados e do cetec. Defender que atas do
  colegiado possam ser divulgadas.  Representantes têm a obrigação de trazer o
  retorno das reuniões para o restante da diretoria.  Pensar em reativar a
  ouvidoria. Questão de estar indo atrás da estatística de evasão. Defender e
  reivindicar espaços de estudos no CTC.  Estar articulando com outros cursos de
  computação a criação da executiva nacional de computação, talvez organizar
  movimento estudantil, discussões estaduais.
\item Ainda não temos a ata de posse. Acho importante a partir do semestre que vem
  tocar todas as tarefas. Precisamos combater a inércia inicial.
\item Fazer um Calendar da diretoria do Calico para marcar coisas específicas para
  a gestão.
\item Sobre o mural do INE e elevador podemos tocar nas férias, a executiva
  nacional também.
\end{itemize}

\presentes {Fábio Coelho, Cauê Baasch, Hans Buss, Matheus Roque, Mikael Saraiva, Luis Ganoza, Lucas Sousa, Helena Aires, Paloma Cione}

\end{document}
