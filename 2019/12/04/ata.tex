\documentclass{ata-calico}
\usepackage{indentfirst}
\pagenumbering{arabic}
\begin{document}

\maketitle

\pauta{Informes/Repasses}
\begin{itemize}
\item Sobre a varanda, ficou decidido na reunião do CETEC que seria dado um voto de confiança ao CALICO. Semestre que vem ela vai ser reaberta, com o centro acadêmico cuidando do diálogo com o pessoal e fomento do uso do local.
\item Dia 28/11 teve uma audiência pública com a reitoria para expor a situação do orçamento depois que voltou a verba. Todos os cortes estão sendo revertidos. Estão delegando a cada setor para ver se precisa ou não reabrir as vagas cortadas. A reitoria está positiva para o próximo semestre, achando que a PLOA que corta 40\% do orçamento não vai passar, não estão se preparando para caso não passe.
\item O RU da trindade vai fechar por algum período nas férias. O do CCA vai ficar aberto, tirando feriados. Vai ficar aberto para todo mundo, não só para isentos, porém não vai ter transporte para o local.
\item Dia 05 vai ter uma reunião do DCE às 12h sobre a audiência e para avaliar a gestão desse ano.
\item As eleições da atlética estão acontecendo pelo moodle, com chapa única.
\item Centro acadêmico da elétrica e eletrônica estão se fundindo e em processo eleitoral. Existe uma chapa única com 27 pessoas na nominata.
\item Estavam acontecendo CEBs para discutir o regimento eleitoral do DCE para as próximas eleições. Como nas últimas sessões não teve quorum mínimo, foi decidido que vai ser retomado no começo do semestre seguinte.
\item A Amnésia vai ser vendida e precisamos comprar ela, se não ela pode acabar. A empresa responsável (Fenda) não está conseguindo se dedicar o suficiente para a porcentagem que tem da festa atualmente. O plano é vender parte do que tem e reduzir o trabalho. Existem duas propostas: pagar 100\% do lucro desse semestre ou 80\% nesse. A Fenda ainda ficaria responsável pelos contratos. Atualmente possui 50\% da festa e faz toda a contratação, sendo o sócio majoritário. Depois da venda, passaria a ter 20\%.
\end{itemize}

\pauta{Limpeza do espaço físico}
\begin{itemize}
\item É preciso limpar a sede do CALICO. Não vai ser muito difícil, pois durante o semestre o local foi sendo limpo, mas é preciso mover móveis e organizar as coisas que tem. Também vai ser o momento para ver do que a gente se desfaz.
\item Talvez seja possível doar os livros, talvez vender para algum sebo, mas algumas coisas só podem sair se tiver algum substituto, como o armário.
\item Seria interessante fazer antes das férias acabarem se for feito um chamado aberto para os estudantes. Se não for chamar, o melhor seria fazer na primeira ou última semana de férias.
\item O espaço físico normalmente começa a ser usado antes do início do semestre, então limpar no começo das férias é uma opção melhor.
\end{itemize}

\textbf{Encaminhamento:} Será decidida uma data no planejamento estratégico

\pauta{Resposta ao incidente da Amnésia}
\begin{itemize}
\item Um dos estudantes do curso foi na festa usando um vestido, porém teve que passar pela revista feminina e foi impedido de usar o banheiro masculino, mesmo indicando que se identificava com o gênero masculino.
\item Foi discutido o assunto na reunião da festa. A contratação dos seguranças segue seguinte ordem: festa contrata o espaço, espaço contrata a empresa de segurança. A comissão organizadora da festa se mostrou contra o CALICO redigir uma nota, pois pode acabar prejudicando a festa futuramente na contratação do espaço.
\item Uma opção seria citar apenas que foi em alguma festa, sem citar especificidade de local ou que foi na Amnésia. Dessa forma, o centro acadêmico iria emitir um posicionamento, mas sem prejudicar a festa.
\item Precisamos ver como isso vai afetar nossas relações futuras com as pessoas e empresas envolvidas. O que podemos fazer é pedir uma reunião com o espaço, em nome do centro acadêmico. Isso, porém, pode trazer o mesmo desgaste com o local que uma nota traria.
\item É preciso ver qual o propósito de publicar uma nota sobre. O problema é mais fundo que uma situação específica, é algo estrutural da sociedade.
\end{itemize}

\textbf{Encaminhamento:} Será publicada uma nota, porém sem especificar local, festa ou indivíduo. Responsáveis: Fábio Coelho, Luis Ganoza, Cauê Baasch, Matheus Roque, Helena Aires.

\textbf{Encaminhamento:} Haverá uma tentativa de contato com o espaço para que o centro acadêmico discuta sobre o assunto com o lugar. Responsável: Matheus Roque.

\pauta{Novo regimento do INE e representações discentes}
\begin{itemize}
\item O departamento está pautando um novo regimento. Nisso, também se propõe mudar a quantidade de representações discentes, passando para 2 representantes por curso ao invés dos 5 divididos entre computação e sistemas (como é atualmente). Também existem 3 câmaras, 2 delas com 1 representação discente e outra, a de pesquisa e extensão, que não tem representação. A reunião vai ser dia 10/12 às 13h45.
\item Precisamos ver se é possível publicar  o regimento para os estudantes, conseguir ter a discussão no curso inteiro.
\item Precisamos ver o porquê de isso estar sendo puxado agora, justamente quando conseguimos nomear representantes para o colegiado depois de anos. Deveríamos estar pautando uma representação mais igualitária, não aceitar uma diminuição de representação. Mesmo se não pensarmos nisso, o estatuto da UFSC prevê 1/5 de representação discente. Devemos estar preparados caso não estejam dispostos a ceder, mostrar que estão tentando podar a representação. É preciso pensar em uma representação diferente, que preveja discentes em todas as câmaras.
\item Existe um ponto no estatuto que prevê votos para professores que não sejam parte da representação, mas não para os alunos. Podemos ver de pedir a alteração desse ponto também.
\end{itemize}

\textbf{Encaminhamento:} Conseguir representação discente equivalente a 1/5 e representação na câmara de pesquisa e extensão.
\textbf{Encaminhamento:} Pedir esclarecimento do funcionamento do colegiado, em questão de indicação para representantes e funcionamento geral das reuniões.

\pauta{Funcionamento da entidade nas férias e data do planejamento estratégico}
\begin{itemize}
\item Existe a possibilidade de fazer no começo ou no fim das férias, com opções de horários sendo: 10h-16h, 13h-19h ou 18h-24h. A reunião vai ser demorada, provavelmente ocupando um período inteiro.
\item Seria bom já fazer no começo das férias para que as coisas possam ser organizadas durante as férias. Além disso, é preciso garantir que pelo menos toda a diretoria possa participar.
\end{itemize}

\textbf{Encaminhamento:} Será aberto um doodle com opções de sexta dia 06/12 até sábado 14/12, com três opções de horário por dia (10h-16h, 13h-19h e 18h-24h).

\pauta{Política de permissão de acesso à sede da entidade}
\begin{itemize}
\item Sempre existe o problema de que as pessoas não tomam conta do espaço, é preciso pensar quem tem acesso. Atualmente quem tem acesso é a diretoria atual, a diretoria anterior, algumas pessoas da atlética e alguns calouros.
\item Não temos problemas reais, como roubos ou mobília estragada, o único problema é a falta de cuidado - principalmente com a sujeira. Levando isso em conta, talvez não tenha problema liberar o acesso para outras pessoas.
\item Deveria ser incentivado também o uso da varanda, não só da sede.
\item Poderia ser delimitado um número fixo de pessoas que teriam acesso à sede sem fazer parte de uma entidade do curso.
\item Esse ponto é algo que pode ser avaliado considerando a necessidade, a cada semestre, com a confiança na pessoa sendo o critério.
\end{itemize}

\textbf{Encaminhamento:} Descadastrar as carteirinhas e mudar a senha no planejamento estratégico.

\presentes {Matheus Roque, Cauê Baasch, Fábio Coelho, Teo Haezer, Mikael Saraiva, Hans Buss, Lucas Sousa, Helena Aires, Luis Oswaldo Ganoza}

\end{document}
