\documentclass{ata-calico}
\usepackage{indentfirst}
\pagenumbering{arabic}
\begin{document}

\maketitle

\pauta{Informes e Repasses}
\itemize
\item Luis diz que nessa semana UDESC abriu processo contra pessoas que organizaram a greve desse mês. Foi redigida uma carta em apoio à comunidade da UDESC contra esse processo, que foi assinada pelo CALICO.

\item A dupla representação foi aprovada, o que quer dizer que um mesmo aluno pode estar em mais de um órgão de colegiado. Temos que falar com o CASIN para indicações para câmara de ensino.

\item Amanhã à noite será a Amnésia!

\item Amanhã acontecerá a Criptainha, haverá várias atividades, oficinas e palestras relacionadas à segurança digital e ativismo com tecnologia. Mais informações no canal de notícias.

\pauta{Elaboração de ofício para professor de ES1}
Existe um descontentamento geral do curso sobre a maneira como a média de ES1 é feita. Já foi feito um pedido por professores na câmara de ensino para que o professor altere o plano de ensino, mas ele se recusou a acatar o pedido. Cauê conversou com o coordenador do curso e foi sugerida a elaboração de um ofício. Luis comenta como os dois professores com quem tivemos reuniões foram abertos à conversa, e podemos inicialmente tentar conversar com ele antes de abrir um processo.

Caso a média seja alterada dificilmente será a média simples, já que ela não cumpre o objetivo que ele propõe de que alunos que sejam carregados no trabalho não consigam passar com as notas ruins nas provas, que é o momento de avaliação individual. Na maneira como a média é calculada agora, um 5 no trabalho resultaria na necessidade de 12 na prova para recuperá-la.

Também levantado outros problemas na maneira como a média é feita, existe um multiplicador que sobe de valor de acordo com as entregas, porém os requisitos para que ele suba são subjetivos. No final, a nota é multiplicada pelo valor do multiplicador. Um exemplo é a entrega final, que soma 0.7 no multiplicador, mas uma entrega final completa é algo subjetivo. Além disso, existem cálculos bastante complexos que acabam ajudando os alunos na prova e que podem inclusive remover a necessidade de fazer uma das provas, e, nesse caso, a nota de uma das outras tem seu valor dobrado.

É sugerido que já levemos sugestões de médias para reunião com ele. Acabou sendo discutidas maneiras de conseguir avaliar os alunos individualmente, como apresentação individual do trabalho.

Antes de tomar alguma ação precisamos conversar com os estudantes que passaram na  disciplina, assim como alunos que ainda cursarão, para ter uma melhor opinião sobre essas propostas. O problema disso é a demora, uma vez que já estamos no final do semestre, é sugerida então uma pesquisa online para levantar a opinião dos alunos, para que a conversa com o Ricardo possa acontecer na semana que vem.
\newline

\textbf{Encaminhamentos:} Será feita pesquisa online e será marcada uma reunião com o professor.

\pauta{Relatório de Estágios}
O sistema de estágios possui quatro campos para relatórios. O coordenador de estágios não gosta da maneira simples do relatório lá feito e exige um relatório mais completo, de no mínimo 8 folhas, e com mais alguns critérios. Muitos alunos demonstraram descontentamento com o relatório exigido devido a sua extensão.

Foi feita uma reunião com o Cancian, ele justificou que possui vários objetivos utilizando esse relatório detalhado, como garantir que o aluno não está servindo apenas como mão de obra barata, entender as tecnologias existentes no mercado, e garantir que as atividades são adequadas para um estudante de ciências da computação. Outra reclamação feita é sigilo da empresa, a qual o professor disse que se dispõe a assinar um termo de confidencialidade, mas ele precisa avaliar as atividades que foram elaboradas pelos alunos.

Questionada a extensão exigida, ele diz que podem ser colocadas imagens e outras coisas que fazem com que seja mais fácil completar o mínimo de páginas requerido. É dito, porém, que a possibilidade de preencher com imagens não é mencionada no "enunciado", além da quantidade excessiva de de folhas que vão fora, já que é necessário imprimir 4 cópias do relatório.

Durante a reunião, foi pedido que ele seja mais claro com os objetivos de cada campo do relatório.

Das reclamações não abordadas na reunião, permanecem a maneira como a média é feita, que é igual à média de modelagem e simulação, e usar como avaliação textos do tipo "como se portar no ambiente de trabalho".
\newline

\textbf{Encaminhamentos:} Continuar discutindo o tema.

\presentes {Cauê Baasch, Mikael Saraiva, Luis Oswaldo, João Trombeta, Lucas Cavalcante de Sousa, Paloma Cione, Caio Oliveira, Helena Aires}
\end{document}
