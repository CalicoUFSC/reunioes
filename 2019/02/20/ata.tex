\documentclass{ata-calico}

\begin{document}

\maketitle

\pauta{Identidade visual}
Cauê menciona como já discutimos a necessidade de remodelar a identidade visual, desde o logo até coisas como postagens no facebook. Isso foi acordado mas não houve encaminhamentos, Paulo Lefol ofereceu preços para fazer e Cauê comenta que podemos falar com Nícholas, que costumava fazer trabalhos de design para o CA. Caio diz que não ficará pronto até o início do semestre por ter apenas uma semana de antecedência. Paulo Lefol pontua que é importante para os calouros estar pronto no inicio do semestre por ser uma mudança difícil de entender durante o semestre, e causar impacto em calouros. Otto e Caio acreditam que não terá tanta diferença se a identidade visual ficar pronta durante o início do semestre. Otto levanta o ponto se é possível fazer algo bom em apenas duas semanas, Caio diz que acredita que não e que é uma correria desnecessária, Paulo, que trabalha com design, comenta como conversou com uma amiga do seu trabalho e acham ser possível. Will diz que por sua experiência isso é um processo demorado, divergências ocorrem e isso demora, mas caso seja delegado um pequeno grupo para ser responsável talvez seja possível.

Thales concorda que é preciso uma imagem nova desde o início por ser o momento de maior interação entre os estudantes e o impacto será maior, mas é preciso que a nova identidade seja boa. Cauê diz como pessoal ainda não viram nossa mudança, é necessário que precisamos mudar de maneira grande e abrupta, se demorar 2 semanas melhor, se for um mês, bola para frente. Alexandre diz que vale a pena priorizar como essa gestão propõe mudanças e isso já mostra como estamos fazendo algo. Paulo diz que caso ele pegue isso, já possui um planejamento, amanhã já começaria a trabalhar nisso para semana que vem já trazer propostas de novas imagens, e na seguinte ter uma proposta final para que fique pronto dia 11. Paulo pede quem será escolhido e Caio diz que primeiro são necessárias propostar, Paulo diz que cobraria 700 por ser seu primeiro trabalho acrescido ao curto tempo, o CALICO poderia descontar desse valor os 100 reais que deve a ele por causa da CONEB. O trabalho incluiria pensamento da marca, criação e aplicação da marca, excluindo a parte de manutenção.

Caio diz que pediu proposta do Nícholas e esperaremos a resposta dele. 

\pauta{Aula de discreta}
Cauê diz que são duas turmas de discreta e precisaremos nos dividir. Caio diz que pelo menos com o Santana deveríamos conversar com antecedência e sugere que é melhor deixar para quarta-feira. É discutido se pediremos um pouco da aula de discreta ou deixaremos para aula de introdução onde todos os calouros estarão juntos.

Cauê mandará e-mail para os professores e iremos dar boas vindas na aula e entregar os manuais, porém as camisetas só serão distribuídas na quarta-feira.

\pauta{Revitalização do Jardim do INE}
O círculo na frente do INE que deveria estar escrito INE é uma oportunidade de juntar pessoas e trocar terra, plantar coisas, fazer o "INE" ser legível. Precisamos primeiro de permissão, o que Will disse que consegue e-mail do coordenador de departamento. Caio comenta que é necessário habilidade e competência. Mikael comenta que é um problema ter as coisas para fazer, é dito que conseguímos Seis se ofereceu para pegar terra.

Cauê fica responsável e é aceitado que isso deve ser algo do CALICO, Mikael ofereceu ajuda, essa semana será pedido a permissão e será feito após o início das aulas.

\pauta{Trilhas}
Cauê diz que surgiu a ideia no grupo de calouros de realizar trilhas, a primeira já com data marcada para o próximo sábado, e pergunta se é algo a ser aceito nas atividades do CALICO, para que tenhamos mais maneiras diferenciadas de recepcionar os calouros. Paulo diz que já viu isso acontecer e normalmente é feito pelas atléticas, ele é a favor mas trás a pergunta de se isso não é uma responsabilidade da atlética. Will diz que não é uma atribuição da atlética e não há problema nenhum, uma vez que estamos trabalhando pela integração e bem estar do estudantes. Lucas pergunta se existe um motivo do por quê seria atividade da atlética, Paulo responde que por se tratar de algo físico e integração. Paloma, Will e Helena sugerem uma parceria, Cauê comenta que não é necessário, pois isso seria algo de integração da computação e Will concorda já que a atlética é é composta também por Sistemas.

É aceito que isso é se torne um evento do CA, sendo responsáveis: Cauê, Paloma e Paulo.

\pauta{Repasses do DCE}
Paulo comenta como existem muitas pessoas em situação irregular morando no PAEP e existiram sugestões de mandar pedido de aumento da moradia estudantil. Estão pensando em como retirar alunos irregulares de forma não truculenta, uma vez que também existem diversos estudantes morando lá. Estão sendo realizadas propostas de solução, priorizando bem estar e segurança dos estudantes.

Cauê comenta que DCE organiza a semana da calourada.

\pauta{SECCOM}
A professora tutora do PET já possuía interesse em retirar a semana acadêmica do PET. João Paulo diz que já está discutindo nomes para a organização, e comenta como a estrutura atual da organização afasta pessoas voluntárias novas, e conclui dizendo que isso é assunto para reuniões futuras. 

Paulo acharia legal que o CALICO não organizasse a SECCOM como o PET fazia, mas que um papel interessante a ser assumido seria ajudar a criar a comissão organizadora e não assumir o papel de organização em si. Evandro concorda que membros do CA devem participar mas como uma entidade a parte, e que já deveríamos nos organizar nas frontes e encontrar um professor responsável pelo projeto. 

Caio comenta da primeira vez que CALICO participou da SECCOM, e o fato de que professora tutora tinha que aprovar absolutamente tudo, o processo foi bastante travado. É necessário que ninguém faça esse papel para não travar as ações tomadas pelo grupo como aconteceu anteriormente. Evandro garante que a tutora não participará da organização e Lucas de Souza adiciona que além disso a intenção é ser algo auto-gerido, mas, precisamos de um professor responsável para situações como projetos da FEESC que necessita tal professor.

Paulo comenta que se o CALICO tivesse CNPJ próprio não precisaríamos mais do projeto da FEESC, Trombeta comenta como regularizar o CNPJ não é uma prioridade atual do CA e Caio concorda, dizendo que além disso é um processo demorado.  Otto levanta o ponto de como é feita a Amnésia e se não poderíamos ter uma relação similar, Paulo e Caio comentam como ela é feita por uma empresa e o CA entra como sócio. Cauê diz então que precisamos achar um professor para assumir o projeto nessa edição e nas próximas pensar como transferir isso para CNPJ do CALICO. 

Thales comenta a preocupação de utilizar o CNPJ do CA, podendo em algum momento acontecer alguma rixa na hora de criar uma comissão e então o evento estar ligado diretamente ao CA. Paulo discorda e diz que a comissão organizadora seria independente e, caso essa rixa acontecesse, a comissão organizadora teria independência suficiente para pedir ajuda de professores. Caio responde que antigamente a SECCOM já estava atrelada ao PET. Luis diz que se a SECCOM quer ser uma entidade a parte ela precisa de um CNPJ próprio, e isso não é uma ação a ser tomada agora. Precisamos no momento achar um professor responsável e criar as comissões, e a decisão disso fica com a comissão. Evandro diz que é possível montar regras e um estatuto para definir a função do professor, mas é discutido que não é difícil encontrar um professor fácil de lidar e que deixaria a responsabilidade para a comissão. 

João Paulo já possui nomes interessados para a comissão, são eles: Vanessa Cunha, Arthur Pixels, e Helena. Não há nomes ainda para: financeiro, infraestrutura e contatos empresariais. Will comenta conversar com pessoal de sistemas, o que é concordado.

João Paulo, Luis, Lucas Sousa, Mikael e João Gabriel se dispuseram a formar a comissão organizadora, assim como contatar o CA de sistemas.



\presentes{Cauê Baasch, João Gabriel Trombeta, Luis Oswaldo Ganoza, Caio Pereira Oliveira, Paloma Cione, Paulo Default, João Paulo T., Mikael Saraiva, Hans, Thales Zirbel, Gustavo Kundlatsch, William Kreamer, Lucas Coquiho, Helena Aires, Alexandre Muller, Salomão, Lucas Verdade, Otto Pires, Thayssa, Marcos Tomaszewski, Evandro Chagas}

\end{document}
