\documentclass{ata-calico}

\begin{document}

\maketitle

\pauta{Estatuto}
Caio diz que tem um pedaço do estatuto feito e que gostaria de terminar essa semana ou semana que vem e enviar para que os outros leiam para futura discussão e iniciar o processo para alterá-lo.

\pauta{Recepção dos Calouros}

MATRICULA: É preciso de alguém para acompanhar calouros na sala no processo de matrícula via cotas. O CALICO pode enviar dois membros para acompanhá-los. Além disso é preciso ficar membros do CA no INE para acompanhar os calouros que estejam fazendo matrículas. Será enviado uma pessoa para acompanhar as matrículas via cotas e oferecer suporte, enquanto outras permanecem no INE. Cauê, Paloma, Marcos e João Paulo se ofereceram para esta função.\newline

EVENTOS DE RECEPÇÃO - MATRICULA: Foi dita a possibilidade de alugar o CETEC no primeiro ou último dia de matrículas para confraternização. Foi levantado que teria pouca presença. 10 calouros foram no bar semestre passado, CETEC seria mais atrativo por oferecer mais do que bebida. Caio comenta, porém, que não é permitido vender bebidas alcóolicas no CETEC, o que diminuiria o evento. Paloma comenta a possibilidade de fazer jogos e depois bar, mas é concordado que é complicado controlar bebida. Ao ver a planilha do CETEC diz indisponível porém não foi reservado, seria necessário conversar com o CETEC sobre isso bem como burlar a regra de não poder alugar o CETEC em dias de semana. Tendo toda discussão em vista, foi concordado que é inviável alugar o CETEC e que bar é o suficiente.  Cauê comenta de calouros que chegam de manhã e menciona a possibilidade de apresentar a UFSC para eles. Será feito um bar após a prova de pré-cálculo (9/3), que provavelmente terá mais presença.\newline

EVENTOS DE RECEPÇÃO - PRIMEIRA SEMANA:
Será feita a linguicinha no primeiro dia, às 18:00. Na quarta-feira ocorre a apresentação do trote integrado. Luís e Cauê acham que é necessário marcar eventos em quarta ou sexta, junto com emplacamento e trote sujo, enquanto Caio, Paulo e Paloma são contra, por causar atrito desnecessário. Continua discussão sobre o quão construtivo é o ambiente do trote. Luis mostra como existem calouros não interessados no trote e seria bom fazer eventos para eles, Caio concorda mas que não deve ser feito no momento do trote para gerar conflito, o que seria ruim inclusive para a imagem do CALICO. Cauê compara aspectos da recepção da arquitetura como apadrinhamento, Paulo diz que isso já é feito pelos veteranos durante o emplacamento e que fazer algo durante o trote distanciaria calouros e mais eventos podem ser feitos em outras datas. É comentado do envolvimento do CA em emplacamento e trote, e todos concordam que ele não deve se envolver. É discutido como devem ter individuos para policiar, com o contraponto de que isso por si só seria oposição ao trote. No final é concordado que o CALICO irá organizar dois eventos: linguicinha e apresentação do trote integrado.\newline

SEGUNDA SEMANA: Na segunda semana o DCE tem um calendário planejado. Terça e quinta terão eventos para calouros, alguns deles são: aula magna e sessão de filmes. A ideia é não criar eventos que gerem conflitos com estes. Na sexta-feira será feito o CHICCO. É discutido que a noite de jogos deve ser feita ainda na segunda semana, sendo escolhida a segunda-feira para fazê-la em uma sala de aula, das 18:00 às 22:00.\newline

EVENTOS NO DECORRER DO SEMESTRE:
Caio comenta Dojo, que já foi feito em outros semestres e deu bastante certo, que poderia ser feito na segunda metade do semestre para que os calouros já tenham um conhecimento. Paulo comenta de fazer antes das provas de  POO, o que é concordado.
Cauê sugere meet-ups, também fala da possibilidade de apresentação de laboratórios. Caio concorda com a estrutura do evento mas discorda de laboratórios uma vez que o público é muito heterogêneo, sugere temas como "async". Como voluntários ficam Cauê e Caio.
Luis comenta a necessidade de divulgar o que é desenvolvido dentro da UFSC, como nos laboratórios. João Paulo comenta como a SECCOM poderia se envolver nesse evento. João Paulo, Cauê, Paulo e Luis.
Meet-ups e eventos de laboratório acontecendo uma vez por mês de maneira intercalada.
Cauê comenta como CA pode dar apoio a outros grupos como GameDev e maratona, principalmente com divulgação.
Também é falado sobre divulgar eventos que ocorrem na região, verificar se UFSC consegue auxiliar com transporte.É discutido o que o CALICO deveria ajudar financeiramente para esse tipo de viagem para torná-las mais acessíveis, mas o custo seria muito alto.
Luis/Cauê falaram sobre fazer eventos para discussão sobre saúde mental. Paulo comenta a possibilidade de conversar com cursos de psicologia e HU. Caio sugere que esses eventos devem ser feitos a nível de CTC e não apenas computação, o que foi concordado. Foi decidido que até o decorrer do semestre deve ser feito uma palestra e um círculo de discussão.\newline

EVENTOS POLÍTICOS:
Luis diz que é necessário discussões sobre a função da Universidade, temas como extensão.
Paulo fala sobre eventos sobre inclusão como a falta de negros e mulheres. Ele diz também a necessidade de procurar o motivo de mulheres que antes participavam do CA terem saído, verificar se não foi opressão de genêro ou algo relacionado, Caio comenta que estava aqui e sabe que elas saíram por falta de tempo, troca de curso ou para dedicar-se mais à graduação. Luis e Caio comenta a falta de público e como apenas pessoas afetadas iriam comparecer a esses eventos.
Cauê menciona a apresentação de software livre.
Possivelmente com rodas de conversa, convidando pessoas para puxar o assunto e iniciar interação. Através dessas conversas discutir temas previamente sugeridos como extensão e software livre. Um por mês.\newline

MANUAL DO CALOURO:
Cauê, Paulo e Luis se dispuseram a criar um protótipo de manual para apresentar para o grupo.\newline

OUVIDORIA DO CALICO:
Caio gostaria de reviver a ouvidoria do CALICO, através de um formulário anônimo onde pessoas poderiam realizar denúncias. É concordado. Para esse fica atribuído João Gabriel e Paloma.\newline

"TODO" DAS PROPOSTAS:
É mencionado verificar o peso discente na votação de coordenador do curso.
É necessário discutir a reforma do currículo.\newline

IDENTIDADE VISUAL:
Refazer logo do CALICO. Paulo se dispôs a conversar com professor do laboratório em que trabalha (Logo) para que seja feito lá.
Cauê comenta de ter pelo menos um protótipo de camiseta para mostrar para os calouros. Já foi feito em outros semestres concursos para decidir camiseta, Caio é contra pela pouca participação e participações zoeiras, Cauê já tem uma ideia de modelo e apoia concurso, no final é acordado abrir um concurso por duas semanas com regras para que não aconteçam zoeiras, votação encabeçada por Caio. Luis menciona como alunos em situação socioeconomica vulnerável não teria dinheiro para comprar e seriam excluídos, é então dada a ideia de dar a camiseta para pessoas que comprovem, por exemplo, através do selo de passe de isento. Caio prefere continuar cobrando 20 de todos ao invés de 25 e subsidiar camiseta para estudantes de baixa renda. Cauê sugere ao invés de subsidiar o custo inteiro da camiseta seja cobrado um valor menor, 5 reais. Cauê menciona e é acordado que isso deve ser dicutido com o curso já que se trata do dinheiro do CA. Caio também sugere cobrar 25 por cada camiseta e distribuir de graça para calouros.\newline

AULA DE DISCRETA:
É concordado que seria bom o CALICO se apresentar logo na primeira aula de discreta e distribuir o manual do calouro.\newline

ARRECADAÇÃO:
Luis sugere que deveriam haver mais eventos para arrecadação monetária que não seja exclusivamente festas privadas. Sem ideias no momento, seria um ponto em aberto para ver como é feito em outros CAs.
Cauê sugere um croudfunding, Luis é contra por parecer uma solução individual e Caio considera pedir migalhas.\newline

CALENDARIO:\newline
05/02 - Bar no Desembargador.\newline
09/03 - Bar, local a confirmar.\newline
11/03 18:00- Linguicinha.\newline
13/03 - Apresentação do trote integrado.\newline
18/03 18:00-22:00 - Noite de jogos.\newline
19/03 - Eventos do DCE.\newline
21/03 - Eventos do DCE.\newline
22/03 - CHICCO, local a confirmar.\newline
??/04 - Primeiro meet-up.\newline
??/?? - Dojos.\newline
??/?? - Meet-up.\newline
??/?? - Exposição de laboratórios\newline
??/?? - Palestra sobre saúde mental.
??/?? - Círculo de discussão sobre saúde mental.\newline
??/?? - Círculo de discussão (software livre, extensão)\newline

\presentes{Cauê Baasch, João Gabriel Trombeta, Luis Oswaldo Ganoza, Caio Pereira Oliveira, Paloma Cione, Paulo Default, João Paulo T., Marcos Tomaszewski}

\end{document}
