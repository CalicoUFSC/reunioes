\documentclass{ata-calico}
\usepackage{indentfirst}
\begin{document}

\maketitle
\pauta {Repasses da reunião do CETEC}

\textbf{Limpeza}: Caio diz que temos que levar 45 reais para a limpeza do semestre inteiro.

\textbf{Feira de cursos}: Feira organizada pelos CA's do CTC, para apresentar aos alunos de ensino médio da comunidade os cursos da UFSC, tendo em vista o nível de evasão dos cursos do CTC. Patrick diz que poderíamos juntar professores e laboratórios. Caio diz que essa feira aconteceu ano passado, mas foi num sábado e não teve um público grande. Caio diz que o ideal seria fazer isso em um dia da semana, e arrumar transportes para trazer os alunos das escolas.

\textbf{Vandalismo}: Caio diz que já houve muitos atos de vandalismo no CETEC. E, apesar de ter câmeras, não há muito o que se fazer sobre as pessoas fumando maconha na varanda, além de mostrar os cartazes e pedir pra eles se retirarem.

\textbf{Porta}: A porta do CETEC está quebrada, ficando aberta 24h por dia. O CETEC aprovou que o Artur comprasse uma controladora e arrumasse a porta.\newline

\textbf{Encaminhamentos}: Comissão organizadora atual da feira de cursos: Seis, Luis, Caue, Hans, Mikael, Yoshi e André.


\pauta {Repasses da reunião do Caravela HC}
Cauê comenta que o Caravela irá sofrer mais pressão do departamento, por estar ocupando uma sala e por supostamente ter tirado os bolsistas do PET. Diz que pretende levar essa pauta para o colegiado, e está também investigando outras entidades estudantis para ver como poderia oficializar o Caravela, dizendo que pretende levar uma pauta para a reunião do colegiado. Patrick indaga qual seria a exata pauta para o colegiado. Cauê diz que precisa enfatizar a importância, representação e diferença do PET. Caio diz que provavelmente não teríamos tempo para convocar a pauta antes da próxima reunião de colegiado. Luis diz que na próxima reunião será votado o coordenador do curso, não tendo tempo para mais nenhuma outra pauta.

Patrick pergunta se o Jean não poderia assumir o Caravela como uma entidade própria e Mikael comenta que, em reunião, Jean disse que não queria ter responsabilidade nenhuma, podendo apenas ceder o espaço e os materias que há dentro. Caio comenta como uma entidade estudantil precisa de um professor coordenador, pois para ter um vínculo do espaço físico com o departamento, precisa-se de alguém do departamento. Patrick indica a Jerusa, e todos comentam que ela é muito atarefada. 

Caio comenta que a Pixel é a entidade que se assemelha mais com o Caravela, então poderíamos perguntar como eles se organizaram.
Cauê relata que Jean recomendou que o Calico fizesse uma nota sobre Caravela, podendo também ser responsável por tal. Caio diz que essa era ideia inicial, mas foi apenas um acordo verbal. Seis indaga se o Caravela não poderia ser uma "equipe de competição"

Luis comenta que não importa se o Calico apoia ou não, mas que precisamos de um professore responsável, pois sem um não conseguiremos nada.

Cauê diz que mudar de espaço também seria uma opção.\newline

\textbf{Encaminhamentos}: Organizar melhor a pauta do colegiado, criando uma comissão do Caravela com o apoio do Calico.


\pauta {Repasses da conversa com o Cancian}
Luis relata que foi conversar com o Cancian, que disse que o Calico estava sem a representação discente necessária, pois não foram entregues os documentos. Caio disse que o documento foi entregue, mas nao publicado oficialmente ainda, por isso ele não estava ciente. 

O coordernador comentou também da reforma do currículo, dizendo que o Calico precisa ajudá-lo. Patrick indaga como poderíamos ajudar nisso. Luis diz tanto no sistema de correlação de matérias, quanto fazendo pressão nos professores. Caio disse que Calico não só deve como já ajudou a reforma. Cancian disse que precisa de mais ajuda e não sabe com quem procurar. Caio diz que precisamos de mais pessoas, mas que o Calico está disposto a ajudar.\newline

\textbf{Encaminhamentos}: Luis irá perguntar, da próxima vez que encontrar o Cancian, quantas exatas cadeiras temos no colegiado do departamento, tanto pra computação (Calico) quanto pra sistemas (Casin).


\pauta {TDC}
The Developers Conferece é um evento de tecnologia, com entradas relativamentes caras, tendo várias opções de "trilhas". Patrick fala que poderíamos entrar em contato com a organização para conseguirmos descontos de estudante. JVM comenta que houve um caso de racismo na edição passada. Luis disse que poderíamos pesquisar isso a fundo, vendo como a organização reagiu a isso. JVM diz que um palestrante tentou se aplicar e recebeu um não como resposta por ser negro (por uma pessoa anônima) e a organização não fez nada a respeito. Patrick recomenda contatar a TDC e exigir uma resposta oficial.\newline


\textbf{Encaminhamentos}: Contatar a TDC e pesquisar sobre o caso de racismo. E, dependendo da resposta, pedir o desconto.

\pauta {Imagem do Calico}
Seis comenta que as pessoas que frequentam o Calico podem ter uma imagem ruim de nós (gestão), pois algumas pessoas que o frequentam não são muito respeitosas (gritos, som alto, ar no 16).

Patrick diz que isso é subjetivo e que não podemos negar pessoas a irem pra lá e que poderíamos, no máximo, botar cartazes e exigir o respeito de todos, como por exemplo, uso da caixa de som e regulagem do ar. Cauê diz que poderíamos criar um formulário pedindo para que as pessoas se expressassem ao se sentirem desconfortáveis e Paloma diz que não é necessário, podendo soar forçado. Mikael diz que poderíamos apenas criar regras de boa convivência. Luis fala para mudarmos para "O que você acha que podemos melhorar?".\newline


\textbf{Encaminhamentos}: Fazer formulários para feedback do Calico e criar cartazes de boa convivência.

\pauta {Instagram}
Patrick diz que precisamos criar um instagram do Calico, mas que o e-mail já está cadastrado no da Amnesia, precisando criar outro. Paloma comenta que criou um canal no telegram, para divulgar as noticias e que precisamos criar um no WhatsApp também.\newline

\textbf{Encaminhamentos}: Caio irá criar o instagram e Paloma o canal no WhatsApp.


\presentes{Paloma Cione, Luis Oswaldo, Caio Pereira, Patrick Machado, Seis, Mikael Saraiva, Hans, João Vitor Maia, Leonardo Yoshi, Gabriel Meireles, André Fernandes}

\end{document}
