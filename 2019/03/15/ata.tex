\documentclass{ata-calico}

\begin{document}

\maketitle

\pauta{Multirão de revitalização do jardim do INE}
Mikael explica que é necessário descrever exatamente o que será feito para que o departamento encaminhe para a segurança. Também é preciso falar com o NEAmb para pedir ajuda e determinar data. Mikael diz ainda que precisaremos de equipamentos, que precisaremos pro NEAmb. Luís e Caio lembram que Cauê e Mikael já estão trabalhando nisso e não precisamos resolver agora, só é preciso ver se o CA precisa ajudar com alguma ferramenta ou algo do tipo.\newline

 \textbf{Encaminhamentos: A comissão pedirá autorização para o centro e ajuda para o NEAmb.}

\pauta{Identidade visual do CALICO}
Patrick diz que conversaram com Sheldon que fez perguntas sobre a história do CALICO e a imagem que gostaríamos de passar. Ele passará orçamento via e-mail, ficará pronto em certa de um mês.\newline

\textbf{Encaminhamentos: Será esperado o retorno do Sheldon.}

\pauta{Equivalência da matéria de Distribuída}
Caio diz que aparentemente as matérias de sistemas e computação não são equivalentes, mas os conteúdos, provas e quantidades de horas são iguais. Caio acha que é algo fácil de fazer, sendo preciso enviar um ofício para coordenação e discutir tal pauta no colegiado, e então dificilmente isso será barrado.\newline

 \textbf{Encaminhamentos: Caio irá conversar com o coordenador e verá a necessidade do ofício. Se for necessário, ele fará.}

\pauta{Cadeira do CTC}
Caio diz que foi feita uma comissão para avaliar a quantidade de cadeiras a serem ocupadas por discentes. Atualmente existem 9, mas a discussão é para que isso seja aumentado. Patrick diz que existem 3 propostas atualmente e que serão reduzidas para duas. Caio diz que a pós-graduação não possui cadeiras, está sendo tentado evitar atrelar cadeiras extras à APG, por esta ser distante do CTC.

Uma das propostas é aumentar o número de cadeiras e as adicionais seriam reservadas para a pós, todas indicadas pelo CETEC. Outra seria aumentar cadeiras e tanto estas quanto mais algumas serem reservadas para a pós, ou por fim aumentar cadeiras e colocar elas para a pós, mas ligadas à APG. Lucas questiona por que o Calico atualmente possui 2 das 9 cadeiras atuais, Caio explica que a distribuição é feita pelo CETEC, e conseguimos uma a mais por não faltar reuniões e por sua atividade.\newline

 \textbf{Encaminhamentos: Patrick fica responsável por encaminhar para o Calico novas propostas da comissão.}

\pauta{Acessibilidade das aulas}
Luis explica como temos um calouro surdo, e isso nos fez perceber como nós (e a UFSC como um todo) tem dificuldades em acessibilidade. Luis mandou mensagem para o aluno pedindo se ele gostaria de algum tipo de ajuda nessa questão e ainda não foi respondido, caso ele responda positivamente o Calico deverá ajudá-lo. Mikael sugere uma reunião extraordinária assim que ele responder, Luis diz que não é necessário, apenas deixar um grupo pronto para isso.\newline

\textbf{Encaminhamentos: Luis e Mikael ficarão responsáveis por tomar ações e avisar assim que o calouro der uma resposta.}

\pauta{Parceria SumOne}

Caio diz que a empresa "SumOne" entrou em contato oferecendo parceria para elaborar eventos. Luis questiona quem são e o que fazem. Caio diz que é uma empresa na Madre Benvenuta que desenvolve software para empreendedores locais, voltado para o marketing. Patrick questiona que tipo de evento, Caio diz que ainda está em aberto. Luis questiona que tipo de auxilio uma empresa dessas pode dar para futuros cientistas da computação e Patrick diz que a empresa ensina bastante seus funcionários, podendo estes fazer palestras, minicursos, etc. Luis esclarece que a pergunta é que se as vagas na empresa são voltadas para Ciências da Computação ou Sistemas, Patrick e Caio diz que são temas complicados porque ambos estão no mercado de trabalho, Luis diz que quer saber se as vagas abertas estão nos nichos exclusivos de computação ou na área em comum com sistemas, Caio responde que eles possuem sim vagas em várias áreas. 

Luis indaga se eles têm como trazer temas de interesse para os alunos de computação, Caio diz que sim e que nós podemos inclusive propor temas e ter liberdade de escolher. 

Paloma pergunta que tipo de parceria eles desejam e Caio diz que eles são voltados para a realização de eventos. Paloma diz que representamos também a seção de alunos que desejam empreender e que isso é interessante para eles. Mikael comenta como minicursos atraem bastante pessoas, como aconteceu na SECCOM. Caio diz que palestra chama atenção de acordo com o tema mas que prefere minicursos e Luis diz que por ser interesse deles é melhor que eles ofereçam o que eles conseguem disponibilizar.\newline

\textbf{Encaminhamentos: Caio conversará com a empresa para ver o que eles podem oferecer e comunicará ao CA.}

\pauta{Seminários nas aulas de Introdução a Computação}
Cauê comenta como a aula de Introd. é bastante fluida e que podemos conversar com alunos interessados que gostam do que fazem para demonstrar aos alunos mais partes do nosso curso, o que poderia ser mais interessante do que professores. 

Patrick diz que é preciso ter cuidado com quem escolheremos. Caio comenta que Plentz tem um cronograma de aula e que não devemos passar por cima dele. Luis diz que podemos fazer a proposta e, se isso não for viável nesse semestres, podemos conversar para semestres seguintes.

Quanto a questão de escolher quem falar, é simples, apenas conversamos e entrevistamos. Lucas diz que o PET já tentou fazer isso e que Plentz disse que não poderia acontecer para esse semestre, por ela ter o cronograma, mas que semestre que vem pareceria viável. Mikael sugeriu já oferecer para semestre que vem e começar ir atrás de alunos para os seminários. Caio acha inviável chamar alunos, pois deveríamos chamar especialistas da área. 

Caio comenta que temos mais pautas para fechar e o tempo é curto, sendo isso discutido posteriormente.\newline

\textbf{Encaminhentos: Voltar a discutir essa pauta em reuniões futuras.}

\pauta{Eleição de Coordenador de curso}

Caio explica que a votação era feita em cédula e depois era calculado o peso. O curso não tem regimento que dita como a eleição deveria ser feita, então segue a regra de maior escopo, que define que a eleição do coordenador do curso é votada em reunião do colegiado. 

Luis questiona o por quê de não termos regimento e pede como devemos resolver isso. Caio diz que isso é decidido em colegiado e que teremos representantes até a próxima reunião. Cauê pergunta se teremos os candidatos com antecedência, Caio diz que sim e responde que a direção de centro deve emitir uma circular anunciando a reunião do colegiado, onde será feita a votação, e os professores poderão criar chapas/se candidatar. Luis diz que devemos discutir em quem os representantes discentes votarão, mesmo que tenhamos pouco peso. Caio explica que, na verdade, temos mais peso por ser uma única pessoa votando e menos professores do geral.\newline

\textbf{Encaminhamentos: Esperaremos o repasse dos nossos representantes sobre quem serão os candidatos após o envio do memorando. Será discutido novamente quando tivermos mais informações.}

\presentes{João Gabriel Trombeta, Luis Oswaldo Ganoza, Mikael Saraiva, Lucas Sousa, Paloma Zankely, Caio Oliveira, Matheus Roque, Cauê, Hans}

\end{document}
